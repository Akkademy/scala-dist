\def\exercise{
   \def\theresult{Exercise~\thesection.\arabic{result}}
   \refstepcounter{result}
   \trivlist\item[\hskip
   \labelsep{\bf \theresult}]}
\def\endexercise{\endtrivlist}
 
\newcommand{\rewriteby}[1]{\mbox{\tab\tab\rm(#1)}}

\chapter{\label{chap:example-one}A First Example}

As a first example, here is an implementation of Quicksort in Scala.

\begin{lstlisting}
def sort(xs: Array[Int]) {
  def swap(i: Int, j: Int) {
    val t = xs(i); xs(i) = xs(j); xs(j) = t
  }
  def sort1(l: Int, r: Int) {
    val pivot = xs((l + r) / 2)
    var i = l; var j = r
    while (i <= j) {
      while (xs(i) < pivot) i += 1
      while (xs(j) > pivot) j -= 1
      if (i <= j) { 
        swap(i, j)
        i += 1
        j -= 1
      }
    } 
    if (l < j) sort1(l, j)
    if (j < r) sort1(i, r)
  }
  sort1(0, xs.length - 1)
}
\end{lstlisting}

The implementation looks quite similar to what one would write in Java
or C.  We use the same operators and similar control structures.
There are also some minor syntactical differences. In particular:
\begin{itemize}
\item
Definitions start with a reserved word. Function definitions start
with \code{def}, variable definitions start with \code{var} and
definitions of values (i.e. read only variables) start with \code{val}.
\item
The declared type of a symbol is given after the symbol and a colon.
The declared type can often be omitted, because the compiler can infer
it from the context.
\item
Array types are written \code{Array[T]} rather than \code{T[]}, 
and array selections are written \code{a(i)} rather than \code{a[i]}.
\item
Functions can be nested inside other functions. Nested functions can
access parameters and local variables of enclosing functions. For
instance, the name of the array \code{xs} is visible in functions
\code{swap} and \code{sort1}, and therefore need not be passed as a
parameter to them.
\end{itemize}
So far, Scala looks like a fairly conventional language with some
syntactic peculiarities. In fact it is possible to write programs in a
conventional imperative or object-oriented style. This is important
because it is one of the things that makes it easy to combine Scala
components with components written in mainstream languages such as
Java, C\# or Visual Basic.

However, it is also possible to write programs in a style which looks
completely different. Here is Quicksort again, this time written in
functional style.

\begin{lstlisting}
def sort(xs: Array[Int]): Array[Int] =
  if (xs.length <= 1) xs
  else {
    val pivot = xs(xs.length / 2)
    Array.concat(
      sort(xs filter (pivot >)),
           xs filter (pivot ==),
      sort(xs filter (pivot <)))
  }
\end{lstlisting}

The functional program captures the essence of the quicksort algorithm
in a concise way:
\begin{itemize}
\item If the array is empty or consists of a single element, 
      it is already sorted, so return it immediately.
\item If the array is not empty, pick an an element in the middle of
      it as a pivot.
\item Partition the array into two sub-arrays containing elements that
are less than, respectively greater than the pivot element, and a
third array which contains elements equal to pivot.
\item Sort the first two sub-arrays by a recursive invocation of
the sort function.\footnote{This is not quite what the imperative algorithm does;
the latter partitions the array into two sub-arrays containing elements
less than or greater or equal to pivot.}
\item The result is obtained by appending the three sub-arrays together.
\end{itemize}
Both the imperative and the functional implementation have the same
asymptotic complexity -- $O(N;log(N))$ in the average case and
$O(N^2)$ in the worst case. But where the imperative implementation
operates in place by modifying the argument array, the functional
implementation returns a new sorted array and leaves the argument
array unchanged. The functional implementation thus requires more
transient memory than the imperative one.

The functional implementation makes it look like Scala is a language
that's specialized for functional operations on arrays. In fact, it is
not; all of the operations used in the example are simple library
methods of a {\em sequence} class \code{Seq[T]} which is part of the
standard Scala library, and which itself is implemented in
Scala. Because arrays are instances of \verb@Seq@ all sequence
methods are available for them.

In particular, there is the method \code{filter} which takes as
argument a {\em predicate function}. This predicate function must map
array elements to boolean values. The result of \code{filter} is an
array consisting of all the elements of the original array for which the
given predicate function is true. The \code{filter} method of an object
of type \code{Array[T]} thus has the signature

\begin{lstlisting}
def filter(p: T => Boolean): Array[T]
\end{lstlisting}

Here, \code{T => Boolean} is the type of functions that take an element
of type \code{t} and return a \code{Boolean}.  Functions like
\code{filter} that take another function as argument or return one as
result are called {\em higher-order} functions.

Scala does not distinguish between identifiers and operator names. An
identifier can be either a sequence of letters and digits which begins
with a letter, or it can be a sequence of special characters, such as
``\code{+}'', ``\code{*}'', or ``\code{:}''.  Any identifier can
be used as an infix operator in Scala.  The binary operation $E;op;E'$
is always interpreted as the method call $E.op(E')$. This holds also
for binary infix operators which start with a letter. Hence, the
expression ~\lstinline@xs filter (pivot >)@~ is equivalent to the
method call ~\lstinline@xs.filter(pivot >)@.

In the quicksort program, \code{filter} is applied three times to an
anonymous function argument.  The first argument, \code{pivot >},
represents a function that takes an argument $x$ and returns the value
~\lstinline@pivot > $x$@. This is an example of a {\em partially applied
function}.  Another, equivalent way to write this function which makes
the missing argument explicit is ~\lstinline@x => pivot > x@.
The function is anonymous, i.e.\ it is not defined with a name. The
type of the \code{x} parameter is omitted because a Scala compiler can
infer it automatically from the context where the function is used. To
summarize, \code{xs.filter(pivot >)} returns a list consisting
of all elements of the list \code{xs} that are smaller than
\code{pivot}.

Looking again in detail at the first, imperative implementation of
Quicksort, we find that many of the language constructs used in the
second solution are also present, albeit in a disguised form.

For instance, ``standard'' binary operators such as \code{+},
\code{-}, or \code{<} are not treated in any special way. Like
\code{append}, they are methods of their left operand. Consequently,
the expression \code{i + 1} is regarded as the invocation
\code{i.+(1)} of the \code{+} method of the integer value \code{x}.
Of course, a compiler is free (if it is moderately smart, even expected)
to recognize the special case of calling the \code{+} method over
integer arguments and to generate efficient inline code for it.

For efficiency and better error diagnostics the \code{while} loop is a
primitive construct in Scala. But in principle, it could have just as
well been a predefined function. Here is a possible implementation of it:
\begin{lstlisting}
def While (p: => Boolean) (s: => Unit) {
  if (p) { s ; While(p)(s) }
}
\end{lstlisting}
The \code{While} function takes as first parameter a test function,
which takes no parameters and yields a boolean value. As second
parameter it takes a command function which also takes no parameters
and yields a result of type \lstinline@Unit@. \code{While} invokes the
command function as long as the test function yields true.

Scala's \lstinline@Unit@ type roughly corresponds to \lstinline@void@
in Java; it is used whenever a function does not return an interesting
result. In fact, because Scala is an expression-oriented language,
every function returns some result. If no explicit return expression
is given, the value \lstinline@()@, which is pronounced ``unit'', is
assumed.  This value is of type \lstinline@Unit@.
Unit-returning functions are also called {\em procedures}.  
Here's a more
``expression-oriented'' formulation of the \lstinline@swap@ function
in the first implementation of quicksort, which makes this explicit:
\begin{lstlisting}
def swap(i: Int, j: Int) {
  val t = xs(i); xs(i) = xs(j); xs(j) = t
  ()
}
\end{lstlisting}
The result value of this function is simply its last expression -- a
\lstinline@return@ keyword is not necessary. Note that functions
returning an explicit value always need an ``='' before their
body or defining expression.  

\chapter{Programming with Actors and Messages}
\label{chap:example-auction}

Here's an example that shows an application area for which Scala is
particularly well suited. Consider the task of implementing an
electronic auction service. We use an Erlang-style actor process
model to implement the participants of the auction. Actors are
objects to which messages are sent. Every actor has a ``mailbox'' of
its incoming messages which is represented as a queue. It can work
sequentially through the messages in its mailbox, or search for
messages matching some pattern.

\begin{lstlisting}[style=floating,label=fig:simple-auction-msgs,caption=Message
    Classes for an Auction Service]
import scala.actors.Actor

abstract class AuctionMessage
case class Offer(bid: Int, client: Actor)  extends AuctionMessage
case class Inquire(client: Actor)          extends AuctionMessage

abstract class AuctionReply
case class  Status(asked: Int, expire: Date) extends AuctionReply
case object BestOffer                        extends AuctionReply
case class  BeatenOffer(maxBid: Int)         extends AuctionReply
case class  AuctionConcluded(seller: Actor, client: Actor) 
                                             extends AuctionReply
case object AuctionFailed                    extends AuctionReply
case object AuctionOver                      extends AuctionReply
\end{lstlisting}

For every traded item there is an auctioneer actor that publishes
information about the traded item, that accepts offers from clients
and that communicates with the seller and winning bidder to close the
transaction. We present an overview of a simple implementation
here.

As a first step, we define the messages that are exchanged during an
auction. There are two abstract base classes
\code{AuctionMessage} for messages from clients to the auction
service, and \code{AuctionReply} for replies from the service to the
clients.  For both base classes there exists a number of cases, which
are defined in Figure~\ref{fig:simple-auction-msgs}.

\begin{lstlisting}[style=floating,label=fig:simple-auction,caption=Implementation of an Auction Service]
class Auction(seller: Actor, minBid: Int, closing: Date) extends Actor {
  val timeToShutdown = 36000000  // msec
  val bidIncrement = 10
  def act() {
    var maxBid = minBid - bidIncrement
    var maxBidder: Actor = null
    var running = true
    while (running) {
      receiveWithin ((closing.getTime() - new Date().getTime())) {
        case Offer(bid, client) =>
          if (bid >= maxBid + bidIncrement) { 
            if (maxBid >= minBid) maxBidder ! BeatenOffer(bid)
            maxBid = bid; maxBidder = client; client ! BestOffer
          } else {
            client ! BeatenOffer(maxBid)
          }
        case Inquire(client) =>
          client ! Status(maxBid, closing)
        case TIMEOUT =>
          if (maxBid >= minBid) {
            val reply = AuctionConcluded(seller, maxBidder)
            maxBidder ! reply; seller ! reply
          } else {
            seller ! AuctionFailed
          }
          receiveWithin(timeToShutdown) {
            case Offer(_, client) => client ! AuctionOver
            case TIMEOUT => running = false
          }
      }
    }
  } 
}
\end{lstlisting}

For each base class, there are a number of {\em case classes} which
define the format of particular messages in the class. These messages
might well be ultimately mapped to small XML documents. We expect
automatic tools to exist that convert between XML documents and
internal data structures like the ones defined above.

Figure~\ref{fig:simple-auction} presents a Scala implementation of a
class \code{Auction} for auction actors that coordinate the bidding
on one item. Objects of this class are created by indicating
\begin{itemize}
\item a seller actor which needs to be notified when the auction is over,
\item a minimal bid,
\item the date when the auction is to be closed.
\end{itemize}
The behavior of the actor is defined by its \code{act} method. That method
repeatedly selects (using \code{receiveWithin}) a message and reacts to it,
until the auction is closed, which is signaled by a \code{TIMEOUT}
message. Before finally stopping, it stays active for another period
determined by the \code{timeToShutdown} constant and replies to
further offers that the auction is closed.

Here are some further explanations of the constructs used in this
program:
\begin{itemize}
\item
The \code{receiveWithin} method of class \code{Actor} takes as
parameters a time span given in milliseconds and a function that
processes messages in the mailbox. The function is given by a sequence
of cases that each specify a pattern and an action to perform for
messages matching the pattern. The \code{receiveWithin} method selects
the first message in the mailbox which matches one of these patterns
and applies the corresponding action to it.
\item
The last case of \code{receiveWithin} is guarded by a
\code{TIMEOUT} pattern. If no other messages are received in the meantime, this
pattern is triggered after the time span which is passed as argument
to the enclosing \code{receiveWithin} method. \code{TIMEOUT} is a
special message, which is triggered by the \code{Actor} implementation itself.
\item
Reply messages are sent using syntax of the form
\code{destination ! SomeMessage}. \code{!} is used here as a
binary operator with an actor and a message as arguments. This is
equivalent in Scala to the method call
\code{destination.!(SomeMessage)}, i.e. the invocation of
the \code{!} method of the destination actor with the given message as
parameter.
\end{itemize}
The preceding discussion gave a flavor of distributed programming in
Scala. It might seem that Scala has a rich set of language constructs
that support actor processes, message sending and receiving,
programming with timeouts, etc. In fact, the opposite is true. All the
constructs discussed above are offered as methods in the library class
\code{Actor}. That class is itself implemented in Scala, based on the underlying 
thread model of the host language (e.g. Java, or .NET).
The implementation of all features of class \code{Actor} used here is
given in Section~\ref{sec:actors}.

The advantages of the library-based approach are relative simplicity
of the core language and flexibility for library designers. Because
the core language need not specify details of high-level process
communication, it can be kept simpler and more general. Because the
particular model of messages in a mailbox is a library module, it can
be freely modified if a different model is needed in some
applications.  The approach requires however that the core language is
expressive enough to provide the necessary language abstractions in a
convenient way. Scala has been designed with this in mind; one of its
major design goals was that it should be flexible enough to act as a
convenient host language for domain specific languages implemented by
library modules. For instance, the actor communication constructs
presented above can be regarded as one such domain specific language,
which conceptually extends the Scala core.

\chapter{\label{chap:simple-funs}Expressions and Simple Functions}

The previous examples gave an impression of what can be done with
Scala.  We now introduce its constructs one by one in a more
systematic fashion. We start with the smallest level, expressions and
functions.

\section{Expressions And Simple Functions}

A Scala system comes with an interpreter which can be seen as a fancy
calculator. A user interacts with the calculator by typing in
expressions. The calculator returns the evaluation results and their
types. For example:

\begin{lstlisting}
scala> 87 + 145
unnamed0: Int = 232

scala> 5 + 2 * 3
unnamed1: Int = 11

scala> "hello" + " world!"
unnamed2: java.lang.String = hello world!
\end{lstlisting}
It is also possible to name a sub-expression and use the name instead
of the expression afterwards:
\begin{lstlisting}
scala> def scale = 5
scale: Int

scala> 7 * scale
unnamed3: Int = 35
\end{lstlisting}
\begin{lstlisting}
scala> def pi = 3.141592653589793
pi: Double

scala> def radius = 10
radius: Int

scala> 2 * pi * radius
unnamed4: Double = 62.83185307179586
\end{lstlisting}
Definitions start with the reserved word \code{def}; they introduce a
name which stands for the expression following the \code{=} sign.  The
interpreter will answer with the introduced name and its type.

Executing a definition such as \code{def x = e} will not evaluate the
expression \code{e}.  Instead \code{e} is evaluated whenever \code{x}
is used. Alternatively, Scala offers a value definition 
\code{val x = e}, which does evaluate the right-hand-side \code{e} as part of the
evaluation of the definition. If \code{x} is then used subsequently,
it is immediately replaced by the pre-computed value of
\code{e}, so that the expression need not be evaluated again.
 
How are expressions evaluated? An expression consisting of operators
and operands is evaluated by repeatedly applying the following
simplification steps.
\begin{itemize}
\item pick the left-most operation
\item evaluate its operands
\item apply the operator to the operand values.
\end{itemize}
A name defined by \code{def}\ is evaluated by replacing the name by the
(unevaluated) definition's right hand side. A name defined by \code{val} is
evaluated by replacing the name by the value of the definitions's
right-hand side.  The evaluation process stops once we have reached a
value. A value is some data item such as a string, a number, an array,
or a list.

\example
Here is an evaluation of an arithmetic expression.
\begin{lstlisting}
$\,\,\,$   (2 * pi) * radius
$\rightarrow$  (2 * 3.141592653589793) * radius
$\rightarrow$  6.283185307179586 * radius
$\rightarrow$  6.283185307179586 * 10
$\rightarrow$  62.83185307179586
\end{lstlisting}
The process of stepwise simplification of expressions to values is
called {\em reduction}.

\section{Parameters}

Using \code{def}, one can also define functions with parameters. For example:
\begin{lstlisting}
scala> def square(x: Double) = x * x
square: (Double)Double

scala> square(2)
unnamed0: Double = 4.0

scala> square(5 + 3)
unnamed1: Double = 64.0

scala> square(square(4))
unnamed2: Double = 256.0

scala> def sumOfSquares(x: Double, y: Double) = square(x) + square(y)
sumOfSquares: (Double,Double)Double

scala> sumOfSquares(3, 2 + 2)
unnamed3: Double = 25.0
\end{lstlisting}

Function parameters follow the function name and are always enclosed
in parentheses.  Every parameter comes with a type, which is indicated
following the parameter name and a colon.  At the present time, we
only need basic numeric types such as the type \code{scala.Double} of
double precision numbers. Scala defines {\em type aliases} for some
standard types, so we can write numeric types as in Java. For instance
\code{double} is a type alias of \code{scala.Double} and \code{int} is
a type alias for \code{scala.Int}.

Functions with parameters are evaluated analogously to operators in
expressions. First, the arguments of the function are evaluated (in
left-to-right order). Then, the function application is replaced by
the function's right hand side, and at the same time all formal
parameters of the function are replaced by their corresponding actual
arguments.

\example\ 
 
\begin{lstlisting}
$\,\,\,$   sumOfSquares(3, 2+2)
$\rightarrow$  sumOfSquares(3, 4)
$\rightarrow$  square(3) + square(4)
$\rightarrow$  3 * 3 + square(4)
$\rightarrow$  9 + square(4)
$\rightarrow$  9 + 4 * 4
$\rightarrow$  9 + 16
$\rightarrow$  25
\end{lstlisting}

The example shows that the interpreter reduces function arguments to
values before rewriting the function application.  One could instead
have chosen to apply the function to unreduced arguments. This would
have yielded the following reduction sequence:
\begin{lstlisting}
$\,\,\,$   sumOfSquares(3, 2+2)
$\rightarrow$  square(3) + square(2+2)
$\rightarrow$  3 * 3 + square(2+2)
$\rightarrow$  9 + square(2+2)
$\rightarrow$  9 + (2+2) * (2+2)
$\rightarrow$  9 + 4 * (2+2)
$\rightarrow$  9 + 4 * 4
$\rightarrow$  9 + 16
$\rightarrow$  25
\end{lstlisting}

The second evaluation order is known as \emph{call-by-name},
whereas the first one is known as \emph{call-by-value}.  For
expressions that use only pure functions and that therefore can be
reduced with the substitution model, both schemes yield the same final
values.  

Call-by-value has the advantage that it avoids repeated evaluation of
arguments. Call-by-name has the advantage that it avoids evaluation of
arguments when the parameter is not used at all by the function.
Call-by-value is usually more efficient than call-by-name, but a
call-by-value evaluation might loop where a call-by-name evaluation
would terminate. Consider:
\begin{lstlisting}
scala> def loop: Int = loop
loop: Int

scala> def first(x: Int, y: Int) = x
first: (Int,Int)Int
\end{lstlisting}
Then \code{first(1, loop)} reduces with call-by-name to \code{1},
whereas the same term reduces with call-by-value repeatedly to itself,
hence evaluation does not terminate.
\begin{lstlisting}
$\,\,\,$   first(1, loop)
$\rightarrow$  first(1, loop)
$\rightarrow$  first(1, loop)
$\rightarrow$  ...
\end{lstlisting}
Scala uses call-by-value by default, but it switches to call-by-name evaluation
if the parameter type is preceded by \code{=>}.

\example\ 
 
\begin{lstlisting}
scala> def constOne(x: Int, y: => Int) = 1
constOne: (Int,=> Int)Int

scala> constOne(1, loop)
unnamed0: Int = 1

scala> constOne(loop, 2)               // gives an infinite loop.
^C                                     // stops execution with Ctrl-C
\end{lstlisting}

\section{Conditional Expressions}

Scala's \code{if-else} lets one choose between two alternatives.  Its
syntax is like Java's \code{if-else}. But where Java's \code{if-else}
can be used only as an alternative of statements, Scala allows the
same syntax to choose between two expressions. That's why Scala's
\code{if-else} serves also as a substitute for Java's conditional
expression \code{ ... ? ... : ...}.

\example\ 

\begin{lstlisting}
scala> def abs(x: Double) = if (x >= 0) x else -x
abs: (Double)Double
\end{lstlisting}
Scala's boolean expressions are similar to Java's; they are formed
from the constants
\code{true} and
\code{false}, comparison operators, boolean negation \code{!} and the
boolean operators $\,$\code{&&}$\,$ and $\,$\code{||}.

\section{\label{sec:sqrt}Example: Square Roots by Newton's Method}

We now illustrate the language elements introduced so far in the
construction of a more interesting program. The task is to write a
function
\begin{lstlisting}
def sqrt(x: Double): Double = ...
\end{lstlisting}
which computes the square root of \code{x}.

A common way to compute square roots is by Newton's method of
successive approximations. One starts with an initial guess \code{y}
(say: \code{y = 1}). One then repeatedly improves the current guess
\code{y} by taking the average of \code{y} and \code{x/y}.  As an
example, the next three columns indicate the guess \code{y}, the
quotient \code{x/y}, and their average for the first approximations of
$\sqrt 2$.
\begin{lstlisting}
1            2/1 = 2               1.5
1.5          2/1.5 = 1.3333        1.4167
1.4167       2/1.4167 = 1.4118     1.4142
1.4142       ...                   ...

$y$            $x/y$                   $(y + x/y)/2$
\end{lstlisting}
One can implement this algorithm in Scala by a set of small functions,
which each represent one of the elements of the algorithm.  

We first define a function for iterating from a guess to the result:
\begin{lstlisting}
def sqrtIter(guess: Double, x: Double): Double =
  if (isGoodEnough(guess, x)) guess
  else sqrtIter(improve(guess, x), x)
\end{lstlisting}
Note that \code{sqrtIter} calls itself recursively.  Loops in
imperative programs can always be modeled by recursion in functional
programs. 

Note also that the definition of \code{sqrtIter} contains a return
type, which follows the parameter section. Such return types are
mandatory for recursive functions. For a non-recursive function, the
return type is optional; if it is missing the type checker will
compute it from the type of the function's right-hand side. However,
even for non-recursive functions it is often a good idea to include a
return type for better documentation.

As a second step, we define the two functions called by
\code{sqrtIter}: a function to \code{improve} the guess and a
termination test \code{isGoodEnough}. Here is their definition.
\begin{lstlisting}
def improve(guess: Double, x: Double) =
  (guess + x / guess) / 2

def isGoodEnough(guess: Double, x: Double) =
  abs(square(guess) - x) < 0.001
\end{lstlisting}

Finally, the \code{sqrt} function itself is defined by an application
of \code{sqrtIter}.
\begin{lstlisting}
def sqrt(x: Double) = sqrtIter(1.0, x)
\end{lstlisting}

\begin{exercise} The \code{isGoodEnough} test is not very precise for small
numbers and might lead to non-termination for very large ones (why?).
Design a different version of \code{isGoodEnough} which does not have
these problems.
\end{exercise}

\begin{exercise} Trace the execution of the \code{sqrt(4)} expression.
\end{exercise}

\section{Nested Functions}

The functional programming style encourages the construction of many
small helper functions. In the last example, the implementation
of \code{sqrt} made use of the helper functions \code{sqrtIter},
\code{improve} and \code{isGoodEnough}. The names of these functions
are relevant only for the implementation of \code{sqrt}. We normally
do not want users of \code{sqrt} to access these functions directly.

We can enforce this (and avoid name-space pollution) by including
the helper functions within the calling function itself:
\begin{lstlisting}
def sqrt(x: Double) = {
  def sqrtIter(guess: Double, x: Double): Double =
    if (isGoodEnough(guess, x)) guess
    else sqrtIter(improve(guess, x), x)
  def improve(guess: Double, x: Double) =
    (guess + x / guess) / 2
  def isGoodEnough(guess: Double, x: Double) =
    abs(square(guess) - x) < 0.001
  sqrtIter(1.0, x)
}
\end{lstlisting}
In this program, the braces \code{\{ ... \}} enclose a {\em block}.
Blocks in Scala are themselves expressions.  Every block ends in a
result expression which defines its value.  The result expression may
be preceded by auxiliary definitions, which are visible only in the
block itself.

Every definition in a block must be followed by a semicolon, 
which separates this definition from subsequent definitions or the result
expression. However, a semicolon is inserted implicitly at the end of
each line, unless one of the following conditions is
true.
\begin{enumerate}
\item
Either the line in question ends in a word such as a period 
or an infix-operator which would not be legal as the end of an expression.
\item
Or the next line begins with a word that cannot start a expression.
\item
Or we are inside parentheses \prog{$($...$)$} or brackets \prog{}, 
because these cannot contain multiple statements anyway.
\end{enumerate}
Therefore, the following are all legal:
\begin{lstlisting}
def f(x: Int) = x + 1;
f(1) + f(2)

def g1(x: Int) = x + 1
g(1) + g(2)

def g2(x: Int) = {x + 1};  /* `;' mandatory */ g2(1) + g2(2)

def h1(x) = 
  x + 
  y
h1(1) * h1(2)

def h2(x: Int) = (
  x     // parentheses mandatory, otherwise a semicolon
  + y   // would be inserted after the `x'.
)
h2(1) / h2(2)
\end{lstlisting}
Scala uses the usual block-structured scoping rules. A name defined in
some outer block is visible also in some inner block, provided it is
not redefined there. This rule permits us to simplify our
\code{sqrt} example. We need not pass \code{x} around as an additional parameter of
the nested functions, since it is always visible in them as a
parameter of the outer function \code{sqrt}. Here is the simplified code:
\begin{lstlisting}
def sqrt(x: Double) = {
  def sqrtIter(guess: Double): Double =
    if (isGoodEnough(guess)) guess
    else sqrtIter(improve(guess))
  def improve(guess: Double) =
    (guess + x / guess) / 2
  def isGoodEnough(guess: Double) =
    abs(square(guess) - x) < 0.001
  sqrtIter(1.0)
}
\end{lstlisting}

\section{Tail Recursion}

Consider the following function to compute the greatest common divisor
of two given numbers.

\begin{lstlisting}
def gcd(a: Int, b: Int): Int = if (b == 0) a else gcd(b, a % b)
\end{lstlisting}

Using our substitution model of function evaluation, 
\code{gcd(14, 21)} evaluates as follows:

\begin{lstlisting}
$\,\,$      gcd(14, 21)  
$\rightarrow\!$     if (21 == 0) 14 else gcd(21, 14 % 21)
$\rightarrow\!$     if (false) 14 else gcd(21, 14 % 21)
$\rightarrow\!$     gcd(21, 14 % 21)
$\rightarrow\!$     gcd(21, 14)
$\rightarrow\!$     if (14 == 0) 21 else gcd(14, 21 % 14)
$\rightarrow$ $\rightarrow$  gcd(14, 21 % 14)
$\rightarrow\!$     gcd(14, 7)
$\rightarrow\!$     if (7 == 0) 14 else gcd(7, 14 % 7)
$\rightarrow$ $\rightarrow$  gcd(7, 14 % 7)
$\rightarrow\!$     gcd(7, 0)
$\rightarrow\!$     if (0 == 0) 7 else gcd(0, 7 % 0)
$\rightarrow$ $\rightarrow$  7
\end{lstlisting}

Contrast this with the evaluation of another recursive function, 
\code{factorial}:

\begin{lstlisting}
def factorial(n: Int): Int = if (n == 0) 1 else n * factorial(n - 1)
\end{lstlisting}

The application \code{factorial(5)} rewrites as follows:
\begin{lstlisting}
$\,\,\,$       factorial(5)
$\rightarrow$      if (5 == 0) 1 else 5 * factorial(5 - 1)
$\rightarrow$      5 * factorial(5 - 1)
$\rightarrow$      5 * factorial(4)
$\rightarrow\ldots\rightarrow$  5 * (4 * factorial(3))
$\rightarrow\ldots\rightarrow$  5 * (4 * (3 * factorial(2)))
$\rightarrow\ldots\rightarrow$  5 * (4 * (3 * (2 * factorial(1))))
$\rightarrow\ldots\rightarrow$  5 * (4 * (3 * (2 * (1 * factorial(0))))
$\rightarrow\ldots\rightarrow$  5 * (4 * (3 * (2 * (1 * 1))))
$\rightarrow\ldots\rightarrow$  120
\end{lstlisting}
There is an important difference between the two rewrite sequences:
The terms in the rewrite sequence of \code{gcd} have again and again
the same form. As evaluation proceeds, their size is bounded by a
constant. By contrast, in the evaluation of factorial we get longer
and longer chains of operands which are then multiplied in the last
part of the evaluation sequence.

Even though actual implementations of Scala do not work by rewriting
terms, they nevertheless should have the same space behavior as in the
rewrite sequences. In the implementation of \code{gcd}, one notes that
the recursive call to \code{gcd} is the last action performed in the
evaluation of its body. One also says that \code{gcd} is
``tail-recursive''. The final call in a tail-recursive function can be
implemented by a jump back to the beginning of that function. The
arguments of that call can overwrite the parameters of the current
instantiation of \code{gcd}, so that no new stack space is needed.
Hence, tail recursive functions are iterative processes, which can be
executed in constant space.

By contrast, the recursive call in \code{factorial} is followed by a
multiplication.  Hence, a new stack frame is allocated for the
recursive instance of factorial, and is deallocated after that
instance has finished. The given formulation of the factorial function
is not tail-recursive; it needs space proportional to its input
parameter for its execution.

More generally, if the last action of a function is a call to another
(possibly the same) function, only a single stack frame is needed for
both functions. Such calls are called ``tail calls''. In principle,
tail calls can always re-use the stack frame of the calling function.
However, some run-time environments (such as the Java VM) lack the
primitives to make stack frame re-use for tail calls efficient.  A
production quality Scala implementation is therefore only required to
re-use the stack frame of a directly tail-recursive function whose
last action is a call to itself.  Other tail calls might be optimized
also, but one should not rely on this across implementations.

\begin{exercise} Design a tail-recursive version of
\code{factorial}.
\end{exercise}

\chapter{\label{chap:first-class-funs}First-Class Functions}

A function in Scala is a ``first-class value''. Like any other value,
it may be passed as a parameter or returned as a result.  Functions
which take other functions as parameters or return them as results are
called {\em higher-order} functions. This chapter introduces
higher-order functions and shows how they provide a flexible mechanism
for program composition.

As a motivating example, consider the following three related tasks:
\begin{enumerate}
\item
Write a function to sum all integers between two given numbers \code{a} and \code{b}:
\begin{lstlisting}
def sumInts(a: Int, b: Int): Int =
  if (a > b) 0 else a + sumInts(a + 1, b)
\end{lstlisting}
\item 
Write a function to sum the squares of all integers between two given numbers 
\code{a} and \code{b}:
\begin{lstlisting}
def square(x: Int): Int = x * x
def sumSquares(a: Int, b: Int): Int =
  if (a > b) 0 else square(a) + sumSquares(a + 1, b)
\end{lstlisting}
\item
Write a function to sum the powers $2^n$ of all integers $n$ between
two given numbers \code{a} and \code{b}:
\begin{lstlisting}
def powerOfTwo(x: Int): Int = if (x == 0) 1 else 2 * powerOfTwo(x - 1)
def sumPowersOfTwo(a: Int, b: Int): Int =
  if (a > b) 0 else powerOfTwo(a) + sumPowersOfTwo(a + 1, b)
\end{lstlisting}
\end{enumerate}
These functions are all instances of
\(\sum^b_a f(n)\) for different values of $f$. 
We can factor out the common pattern by defining a function \code{sum}:
\begin{lstlisting}
def sum(f: Int => Int, a: Int, b: Int): Int =
  if (a > b) 0 else f(a) + sum(f, a + 1, b)
\end{lstlisting}
The type \code{Int => Int} is the type of functions that
take arguments of type \code{Int} and return results of type
\code{Int}. So \code{sum} is a function which takes another function as
a parameter. In other words, \code{sum} is a {\em higher-order}
function.

Using \code{sum}, we can formulate the three summing functions as
follows.
\begin{lstlisting}
def sumInts(a: Int, b: Int): Int = sum(id, a, b)
def sumSquares(a: Int, b: Int): Int = sum(square, a, b)
def sumPowersOfTwo(a: Int, b: Int): Int = sum(powerOfTwo, a, b)
\end{lstlisting}
where
\begin{lstlisting}
def id(x: Int): Int = x
def square(x: Int): Int = x * x
def powerOfTwo(x: Int): Int = if (x == 0) 1 else 2 * powerOfTwo(x - 1)
\end{lstlisting}

\section{Anonymous Functions}

Parameterization by functions tends to create many small functions. In
the previous example, we defined \code{id}, \code{square} and
\code{power} as separate functions, so that they could be 
passed as arguments to \code{sum}.

Instead of using named function definitions for these small argument
functions, we can formulate them in a shorter way as {\em anonymous
functions}. An anonymous function is an expression that evaluates to a
function; the function is defined without giving it a name. As an
example consider the anonymous square function:
\begin{lstlisting}
  (x: Int) => x * x
\end{lstlisting}
The part before the arrow `\code{=>}' are the parameters of the function,
whereas the part following the `\code{=>}' is its body. 
For instance, here is an anonymous function which multiples its two arguments.
\begin{lstlisting}
  (x: Int, y: Int) => x * y
\end{lstlisting}
Using anonymous functions, we can reformulate the first two summation
functions without named auxiliary functions:
\begin{lstlisting}
def sumInts(a: Int, b: Int): Int = sum((x: Int) => x, a, b)
def sumSquares(a: Int, b: Int): Int = sum((x: Int) => x * x, a, b)
\end{lstlisting}
Often, the Scala compiler can deduce the parameter type(s) from the
context of the anonymous function in which case they can be omitted.
For instance, in the case of \code{sumInts} or \code{sumSquares}, one
knows from the type of \code{sum} that the first parameter must be a
function of type \code{Int => Int}.  Hence, the parameter type
\code{Int} is redundant and may be omitted. If there is a single
parameter without a type, we may also omit the parentheses around it:
\begin{lstlisting}
def sumInts(a: Int, b: Int): Int = sum(x => x, a, b)
def sumSquares(a: Int, b: Int): Int = sum(x => x * x, a, b)
\end{lstlisting}

Generally, the Scala term
\code{(x}$_1$\code{: T}$_1$\code{, ..., x}$_n$\code{: T}$_n$\code{) => E} 
defines a function which maps its parameters
\code{x}$_1$\code{, ..., x}$_n$ to the result of the expression \code{E}
(where \code{E} may refer to \code{x}$_1$\code{, ..., x}$_n$).  Anonymous
functions are not essential language elements of Scala, as they can
always be expressed in terms of named functions. Indeed, the 
anonymous function
\begin{lstlisting}
(x$_1$: T$_1$, ..., x$_n$: T$_n$) => E
\end{lstlisting}
is equivalent to the block
\begin{lstlisting}
{ def f (x$_1$: T$_1$, ..., x$_n$: T$_n$) = E ; f _ }
\end{lstlisting}
where \code{f} is fresh name which is used nowhere else in the program.
We also say, anonymous functions are ``syntactic sugar''.

\section{Currying}

The latest formulation of the summing functions is already quite
compact. But we can do even better. Note that
\code{a} and \code{b} appear as parameters and arguments of every function
but they do not seem to take part in interesting combinations. Is
there a way to get rid of them?

Let's try to rewrite \code{sum} so that it does not take the bounds
\code{a} and \code{b} as parameters:
\begin{lstlisting}
def sum(f: Int => Int): (Int, Int) => Int = {
  def sumF(a: Int, b: Int): Int =
    if (a > b) 0 else f(a) + sumF(a + 1, b)
  sumF
}
\end{lstlisting}
In this formulation, \code{sum} is a function which returns another
function, namely the specialized summing function \code{sumF}. This
latter function does all the work; it takes the bounds \code{a} and
\code{b} as parameters, applies \code{sum}'s function parameter \code{f} to all
integers between them, and sums up the results. 

Using this new formulation of \code{sum}, we can now define:
\begin{lstlisting}
def sumInts  =  sum(x => x)
def sumSquares  =  sum(x => x * x)
def sumPowersOfTwo  =  sum(powerOfTwo)
\end{lstlisting}
Or, equivalently, with value definitions:
\begin{lstlisting}
val sumInts  =  sum(x => x)
val sumSquares  =  sum(x => x * x)
val sumPowersOfTwo  =  sum(powerOfTwo)
\end{lstlisting}

\code{sumInts}, \code{sumSquares}, and \code{sumPowersOfTwo} can be
applied like any other function. For instance,
\begin{lstlisting}
scala> sumSquares(1, 10) + sumPowersOfTwo(10, 20)
unnamed0: Int = 267632001
\end{lstlisting}
How are function-returning functions applied? As an example, in the expression
\begin{lstlisting}
sum(x => x * x)(1, 10) ,
\end{lstlisting}
the function \code{sum} is applied to the squaring function 
\code{(x => x * x)}. The resulting function is then 
applied to the second argument list, \code{(1, 10)}.

This notation is possible because function application associates to the left.
That is, if $\mbox{args}_1$ and $\mbox{args}_2$ are argument lists, then 
\bda{lcl}
f(\mbox{args}_1)(\mbox{args}_2) & \ \ \mbox{is equivalent to}\ \ & (f(\mbox{args}_1))(\mbox{args}_2)
\eda
In our example, \code{sum(x => x * x)(1, 10)} is equivalent to the
following expression:
\code{(sum(x => x * x))(1, 10)}.

The style of function-returning functions is so useful that Scala has
special syntax for it. For instance, the next definition of \code{sum}
is equivalent to the previous one, but is shorter:
\begin{lstlisting}
def sum(f: Int => Int)(a: Int, b: Int): Int =
  if (a > b) 0 else f(a) + sum(f)(a + 1, b)
\end{lstlisting}
Generally, a curried function definition 
\begin{lstlisting}
def f (args$_1$) ... (args$_n$) = E
\end{lstlisting}
where $n > 1$ expands to
\begin{lstlisting}
def f (args$_1$) ... (args$_{n-1}$) = { def g (args$_n$) = E ; g }
\end{lstlisting}
where \code{g} is a fresh identifier. Or, shorter, using an anonymous function:
\begin{lstlisting}
def f (args$_1$) ... (args$_{n-1}$) = ( args$_n$ ) => E .
\end{lstlisting}
Performing this step $n$ times yields that
\begin{lstlisting}
def f (args$_1$) ... (args$_n$) = E
\end{lstlisting}
is equivalent to
\begin{lstlisting}
def f = (args$_1$) => ... => (args$_n$) => E .
\end{lstlisting}
Or, equivalently, using a value definition:
\begin{lstlisting}
val f = (args$_1$) => ... => (args$_n$) => E .
\end{lstlisting}
This style of function definition and application is called {\em
currying} after its promoter, Haskell B.\ Curry, a logician of the
20th century, even though the idea goes back further to Moses
Sch\"onfinkel and Gottlob Frege.

The type of a function-returning function is expressed analogously to
its parameter list. Taking the last formulation of \code{sum} as an example,
the type of \code{sum} is \code{(Int => Int) => (Int, Int) => Int}.
This is possible because function types associate to the right. I.e.
\begin{lstlisting}
T$_1$ => T$_2$ => T$_3$       $\mbox{is equivalent to}$     T$_1$ => (T$_2$ => T$_3$)
\end{lstlisting}


\begin{exercise}
1. The \code{sum} function uses a linear recursion. Can you write a
tail-recursive one by filling in the ??'s?

\begin{lstlisting}
def sum(f: Int => Double)(a: Int, b: Int): Double = {
  def iter(a, result) = {
    if (??) ??
    else iter(??, ??)
  }
  iter(??, ??)
}
\end{lstlisting}
\end{exercise}

\begin{exercise}
Write a function \code{product} that computes the product of the
values of functions at points over a given range.
\end{exercise}

\begin{exercise}
Write \code{factorial} in terms of \code{product}.
\end{exercise}

\begin{exercise}
Can you write an even more general function which generalizes both
\code{sum} and \code{product}?
\end{exercise}

\section{Example: Finding Fixed Points of Functions}

A number \code{x} is called a {\em fixed point} of a function \code{f} if
\begin{lstlisting}
f(x) = x .
\end{lstlisting}
For some functions \code{f} we can locate the fixed point by beginning
with an initial guess and then applying \code{f} repeatedly, until the
value does not change anymore (or the change is within a small
tolerance). This is possible if the sequence
\begin{lstlisting}
x, f(x), f(f(x)), f(f(f(x))), ...
\end{lstlisting}
converges to fixed point of $f$. This idea is captured in
the following ``fixed-point finding function'':
\begin{lstlisting}
val tolerance = 0.0001
def isCloseEnough(x: Double, y: Double) = abs((x - y) / x) < tolerance
def fixedPoint(f: Double => Double)(firstGuess: Double) = {
  def iterate(guess: Double): Double = {
    val next = f(guess)
    if (isCloseEnough(guess, next)) next
    else iterate(next)
  }
  iterate(firstGuess)
}
\end{lstlisting}
We now apply this idea in a reformulation of the square root function.
Let's start with a specification of \code{sqrt}:
\begin{lstlisting}
sqrt(x)  =  $\mbox{the {\sl y} such that}$  y * y = x
         =  $\mbox{the {\sl y} such that}$  y = x / y
\end{lstlisting}
Hence, \code{sqrt(x)} is a fixed point of the function \code{y => x / y}.
This suggests that \code{sqrt(x)} can be computed by fixed point iteration:
\begin{lstlisting}
def sqrt(x: double) = fixedPoint(y => x / y)(1.0)
\end{lstlisting}
But if we try this, we find that the computation does not
converge. Let's instrument the fixed point function with a print
statement which keeps track of the current \code{guess} value:
\begin{lstlisting}
def fixedPoint(f: Double => Double)(firstGuess: Double) = {
  def iterate(guess: Double): Double = {
    val next = f(guess)
    println(next)
    if (isCloseEnough(guess, next)) next
    else iterate(next)
  }
  iterate(firstGuess)
}
\end{lstlisting}
Then, \code{sqrt(2)} yields:
\begin{lstlisting}
  2.0
  1.0
  2.0
  1.0
  2.0
  ...
\end{lstlisting}
One way to control such oscillations is to prevent the guess from changing too much. 
This can be achieved by {\em averaging} successive values of the original sequence:
\begin{lstlisting}
scala> def sqrt(x: Double) = fixedPoint(y => (y + x/y) / 2)(1.0)
sqrt: (Double)Double

scala> sqrt(2.0)
  1.5
  1.4166666666666665
  1.4142156862745097
  1.4142135623746899
  1.4142135623746899
\end{lstlisting}
In fact, expanding the \code{fixedPoint} function yields exactly our 
previous definition of fixed point from Section~\ref{sec:sqrt}.

The previous examples showed that the expressive power of a language
is considerably enhanced if functions can be passed as arguments.  The
next example shows that functions which return functions can also be
very useful.

Consider again fixed point iterations. We started with the observation
that $\sqrt(x)$ is a fixed point of the function \code{y => x / y}.
Then we made the iteration converge by averaging successive values.
This technique of {\em average damping} is so general that it
can be wrapped in another function.
\begin{lstlisting}
def averageDamp(f: Double => Double)(x: Double) = (x + f(x)) / 2
\end{lstlisting}
Using \code{averageDamp}, we can reformulate the square root function
as follows.
\begin{lstlisting}
def sqrt(x: Double) = fixedPoint(averageDamp(y => x/y))(1.0)
\end{lstlisting}
This expresses the elements of the algorithm as clearly as possible.

\begin{exercise} Write a function for cube roots using \code{fixedPoint} and 
\code{averageDamp}.
\end{exercise}

\section{Summary}

We have seen in the previous chapter that functions are essential
abstractions, because they permit us to introduce general methods of
computing as explicit, named elements in our programming language.
The present chapter has shown that these abstractions can be combined
by higher-order functions to create further abstractions.  As
programmers, we should look out for opportunities to abstract and to
reuse. The highest possible level of abstraction is not always the
best, but it is important to know abstraction techniques, so that one
can use abstractions where appropriate.

\section{Language Elements Seen So Far}

Chapters~\ref{chap:simple-funs} and \ref{chap:first-class-funs} have
covered Scala's language elements to express expressions and types
comprising of primitive data and functions.  The context-free syntax
of these language elements is given below in extended Backus-Naur
form, where `\code{|}' denotes alternatives, \code{[...]} denotes
option (0 or 1 occurrence), and \lstinline@{...}@ denotes repetition
(0 or more occurrences).

\subsection*{Characters}

Scala programs are sequences of (Unicode) characters. We distinguish the
following character sets:
\begin{itemize}
\item
whitespace, such as `\code{ }', tabulator, or newline characters,
\item
letters `\code{a}' to `\code{z}', `\code{A}' to `\code{Z}',
\item
digits \code{`0'} to `\code{9}',
\item
the delimiter characters

\begin{lstlisting}
.    ,    ;    (    )    {    }    [    ]    \    $\mbox{\tt "}$    '
\end{lstlisting}

\item
operator characters, such as `\code{#}' `\code{+}',
`\code{:}'. Essentially, these are printable characters which are
in none of the character sets above.
\end{itemize}

\subsection*{Lexemes:}

\begin{lstlisting}
ident    =  letter {letter | digit}
         |  operator { operator }
         |  ident '_' ident
literal  =  $\mbox{``as in Java''}$
\end{lstlisting}

Literals are as in Java. They define numbers, characters, strings, or
boolean values.  Examples of literals as \code{0}, \code{1.0e10}, \code{'x'},
\code{"he said \"hi!\""}, or \code{true}.

Identifiers can be of two forms. They either start with a letter,
which is followed by a (possibly empty) sequence of letters or
symbols, or they start with an operator character, which is followed
by a (possibly empty) sequence of operator characters.  Both forms of
identifiers may contain underscore characters `\code{_}'. Furthermore,
an underscore character may be followed by either sort of
identifier. Hence, the following are all legal identifiers:
\begin{lstlisting}
x     Room10a     +     --     foldl_:     +_vector
\end{lstlisting}
It follows from this rule that subsequent operator-identifiers need to
be separated by whitespace. For instance, the input
\code{x+-y} is parsed as the three token sequence \code{x}, \code{+-},
\code{y}. If we want to express the sum of \code{x} with the
negated value of \code{y}, we need to add at least one space,
e.g. \code{x+ -y}.

The \verb@$@ character is reserved for compiler-generated
identifiers; it should not be used in source programs. %$

The following are reserved words, they may not be used as identifiers:
\begin{lstlisting}
abstract    case        catch       class       def    
do          else        extends     false       final    
finally     for         if          implicit    import      
match       new         null        object      override    
package     private     protected   requires    return      
sealed      super       this        throw       trait
try         true        type        val         var         
while       with        yield
_    :    =    =>    <-    <:    <%     >:    #    @
\end{lstlisting}

\subsection*{Types:}

\begin{lstlisting}
Type          =  SimpleType | FunctionType
FunctionType  =  SimpleType '=>' Type | '(' [Types] ')' '=>' Type
SimpleType    =  Byte | Short | Char | Int | Long | Float | Double |
                 Boolean | Unit | String
Types         =  Type {`,' Type}
\end{lstlisting}

Types can be:
\begin{itemize}
\item number types \code{Byte}, \code{Short}, \code{Char}, \code{Int}, \code{Long}, \code{Float} and \code{Double} (these are as in Java),
\item the type \code{Boolean} with values \code{true} and \code{false},
\item the type \code{Unit} with the only value \lstinline@()@,
\item the type \code{String},
\item function types such as \code{(Int, Int) => Int} or \code{String => Int => String}.
\end{itemize}

\subsection*{Expressions:}

\begin{lstlisting}
Expr         = InfixExpr | FunctionExpr | if '(' Expr ')' Expr else Expr
InfixExpr    = PrefixExpr | InfixExpr Operator InfixExpr
Operator     = ident
PrefixExpr   = ['+' | '-' | '!' | '~' ] SimpleExpr
SimpleExpr   = ident | literal | SimpleExpr '.' ident | Block
FunctionExpr = (Bindings | Id) '=>' Expr
Bindings     = `(' Binding {`,' Binding} `)'
Binding      = ident [':' Type]
Block        = '{' {Def ';'} Expr '}'
\end{lstlisting}

Expressions can be:
\begin{itemize}
\item
identifiers such as \code{x}, \code{isGoodEnough}, \code{*}, or \code{+-},
\item
literals, such as \code{0}, \code{1.0}, or \code{"abc"},
\item
field and method selections, such as \code{System.out.println},
\item
function applications, such as \code{sqrt(x)}, 
\item
operator applications, such as \code{-x} or \code{y + x},
\item
conditionals, such as \code{if (x < 0) -x else x},
\item
blocks, such as \lstinline@{ val x = abs(y) ; x * 2 }@,
\item
anonymous functions, such as \code{x => x + 1} or \code{(x: Int, y: Int) => x + y}.
\end{itemize}

\subsection*{Definitions:}

\begin{lstlisting}
Def          =  FunDef  |  ValDef
FunDef       =  'def' ident {'(' [Parameters] ')'} [':' Type] '=' Expr
ValDef       =  'val' ident [':' Type] '=' Expr
Parameters   =  Parameter {',' Parameter}
Parameter    =  ident ':' ['=>'] Type
\end{lstlisting}
Definitions can be:
\begin{itemize}
\item
function definitions such as \code{def square(x: Int): Int = x * x},
\item
value definitions such as \code{val y = square(2)}.
\end{itemize}

\chapter{Classes and Objects}
\label{chap:classes}

Scala does not have a built-in type of rational numbers, but it is
easy to define one, using a class. Here's a possible implementation.

\begin{lstlisting}
class Rational(n: Int, d: Int) {
  private def gcd(x: Int, y: Int): Int = {
    if (x == 0) y
    else if (x < 0) gcd(-x, y)
    else if (y < 0) -gcd(x, -y)
    else gcd(y % x, x)
  }
  private val g = gcd(n, d)

  val numer: Int = n/g
  val denom: Int = d/g
  def +(that: Rational) =
    new Rational(numer * that.denom + that.numer * denom,
                 denom * that.denom)
  def -(that: Rational) =
    new Rational(numer * that.denom - that.numer * denom, 
                 denom * that.denom)
  def *(that: Rational) =
    new Rational(numer * that.numer, denom * that.denom)
  def /(that: Rational) =
    new Rational(numer * that.denom, denom * that.numer)
}
\end{lstlisting}
This defines \code{Rational} as a class which takes two constructor
arguments \code{n} and \code{d}, containing the number's numerator and
denominator parts.  The class provides fields which return these parts
as well as methods for arithmetic over rational numbers.  Each
arithmetic method takes as parameter the right operand of the
operation. The left operand of the operation is always the rational
number of which the method is a member.

\paragraph{Private members}
The implementation of rational numbers defines a private method
\code{gcd} which computes the greatest common denominator of two
integers, as well as a private field \code{g} which contains the
\code{gcd} of the constructor arguments. These members are inaccessible
outside class \code{Rational}. They are used in the implementation of
the class to eliminate common factors in the constructor arguments in
order to ensure that numerator and denominator are always in
normalized form.

\paragraph{Creating and Accessing Objects}
As an example of how rational numbers can be used, here's a program
that prints the sum of all numbers $1/i$ where $i$ ranges from 1 to 10.
\begin{lstlisting}
var i = 1
var x = new Rational(0, 1)
while (i <= 10) {
  x += new Rational(1, i)
  i += 1
}
println("" + x.numer + "/" + x.denom)
\end{lstlisting}
The \code{+} takes as left operand a string and as right operand a
value of arbitrary type. It returns the result of converting its right
operand to a string and appending it to its left operand. 
  
\paragraph{Inheritance and Overriding}
Every class in Scala has a superclass which it extends.  
\comment{Excepted is
only the root class \code{Object}, which does not have a superclass,
and which is indirectly extended by every other class.  }
If a class
does not mention a superclass in its definition, the root type
\code{scala.AnyRef} is implicitly assumed (for Java implementations,
this type is an alias for \code{java.lang.Object}. For instance, class
\code{Rational} could equivalently be defined as
\begin{lstlisting}
class Rational(n: Int, d: Int) extends AnyRef {
  ... // as before
}
\end{lstlisting}
A class inherits all members from its superclass. It may also redefine
(or: {\em override}) some inherited members. For instance, class
\code{java.lang.Object} defines
a method
\code{toString} which returns a representation of the object as a string:
\begin{lstlisting}
class Object {
  ...
  def toString: String = ...
}
\end{lstlisting}
The implementation of \code{toString} in \code{Object}
forms a string consisting of the object's class name and a number. It
makes sense to redefine this method for objects that are rational
numbers:
\begin{lstlisting}
class Rational(n: Int, d: Int) extends AnyRef {
  ... // as before
  override def toString = "" + numer + "/" + denom
}
\end{lstlisting}
Note that, unlike in Java, redefining definitions need to be preceded
by an \code{override} modifier.

If class $A$ extends class $B$, then objects of type $A$ may be used
wherever objects of type $B$ are expected. We say in this case that
type $A$ {\em conforms} to type $B$.  For instance, \code{Rational}
conforms to \code{AnyRef}, so it is legal to assign a \code{Rational}
value to a variable of type \code{AnyRef}:
\begin{lstlisting}
var x: AnyRef = new Rational(1, 2)
\end{lstlisting}

\paragraph{Parameterless Methods}
%Also unlike in Java, methods in Scala do not necessarily take a
%parameter list. An example is \code{toString}; the method is invoked
%by simply mentioning its name. For instance:
%\begin{lstlisting}
%val r = new Rational(1,2)
%System.out.println(r.toString);      // prints``1/2''
%\end{lstlisting}
Unlike in Java, methods in Scala do not necessarily take a
parameter list. An example is the \code{square} method below. This
method is invoked by simply mentioning its name. 
\begin{lstlisting}
class Rational(n: Int, d: Int) extends AnyRef {
  ... // as before
  def square = new Rational(numer*numer, denom*denom)
}
val r = new Rational(3, 4)
println(r.square)           // prints``9/16''*
\end{lstlisting}
That is, parameterless methods are accessed just as value fields such
as \code{numer} are. The difference between values and parameterless
methods lies in their definition. The right-hand side of a value is
evaluated when the object is created, and the value does not change
afterwards. A right-hand side of a parameterless method, on the other
hand, is evaluated each time the method is called.  The uniform access
of fields and parameterless methods gives increased flexibility for
the implementer of a class. Often, a field in one version of a class
becomes a computed value in the next version. Uniform access ensures
that clients do not have to be rewritten because of that change.

\paragraph{Abstract Classes}

Consider the task of writing a class for sets of integer numbers with
two operations, \code{incl} and \code{contains}. \code{(s incl x)}
should return a new set which contains the element \code{x} together
with all the elements of set \code{s}. \code{(s contains x)} should
return true if the set \code{s} contains the element \code{x}, and
should return \code{false} otherwise. The interface of such sets is
given by:  
\begin{lstlisting}
abstract class IntSet {
  def incl(x: Int): IntSet
  def contains(x: Int): Boolean
}
\end{lstlisting}
\code{IntSet} is labeled as an \emph{abstract class}. This has two
consequences.  First, abstract classes may have {\em deferred} members
which are declared but which do not have an implementation. In our
case, both \code{incl} and \code{contains} are such members. Second,
because an abstract class might have unimplemented members, no objects
of that class may be created using \code{new}. By contrast, an
abstract class may be used as a base class of some other class, which
implements the deferred members.

\paragraph{Traits}

Instead of \code{abstract class} one also often uses the keyword
\code{trait} in Scala. Traits are abstract classes that are meant to
be added to some other class. This might be because a trait adds some
methods or fields to an unknown parent class.  For instance, a trait
\code{Bordered} might be used to add a border to a various graphical
components. Another usage scenario is where the trait collects
signatures of some functionality provided by different classes, much
in the way a Java interface would work.

Since \code{IntSet} falls in this category, one can alternatively
define it as a trait:
\begin{lstlisting}
trait IntSet {
  def incl(x: Int): IntSet
  def contains(x: Int): Boolean
}
\end{lstlisting}

\paragraph{Implementing Abstract Classes}

Let's say, we plan to implement sets as binary trees.  There are two
possible forms of trees. A tree for the empty set, and a tree
consisting of an integer and two subtrees. Here are their
implementations.

\begin{lstlisting}
class EmptySet extends IntSet {
  def contains(x: Int): Boolean = false
  def incl(x: Int): IntSet = new NonEmptySet(x, new EmptySet, new EmptySet)
}
\end{lstlisting}

\begin{lstlisting}
class NonEmptySet(elem: Int, left: IntSet, right: IntSet) extends IntSet {
  def contains(x: Int): Boolean =
    if (x < elem) left contains x
    else if (x > elem) right contains x
    else true
  def incl(x: Int): IntSet =
    if (x < elem) new NonEmptySet(elem, left incl x, right)
    else if (x > elem) new NonEmptySet(elem, left, right incl x)
    else this
}
\end{lstlisting}
Both \code{EmptySet} and \code{NonEmptySet} extend class
\code{IntSet}.  This implies that types \code{EmptySet} and
\code{NonEmptySet} conform to type \code{IntSet} -- a value of type \code{EmptySet} or \code{NonEmptySet} may be used wherever a value of type \code{IntSet} is required.

\begin{exercise} Write methods \code{union} and \code{intersection} to form
the union and intersection between two sets.
\end{exercise}

\begin{exercise} Add a method 
\begin{lstlisting}
def excl(x: Int)
\end{lstlisting}
to return the given set without the element \code{x}. To accomplish this,
it is useful to also implement a test method
\begin{lstlisting}
def isEmpty: Boolean
\end{lstlisting}
for sets.
\end{exercise}

\paragraph{Dynamic Binding}

Object-oriented languages (Scala included) use \emph{dynamic dispatch}
for method invocations.  That is, the code invoked for a method call
depends on the run-time type of the object which contains the method.
For example, consider the expression \code{s contains 7} where
\code{s} is a value of declared type \code{s: IntSet}. Which code for
\code{contains} is executed depends on the type of value of \code{s} at run-time.
If it is an \code{EmptySet} value, it is the implementation of \code{contains} in class \code{EmptySet} that is executed, and analogously for \code{NonEmptySet} values. 
This behavior is a direct consequence of our substitution model of evaluation.
For instance,
\begin{lstlisting}
    (new EmptySet).contains(7) 

->  $\rewriteby{by replacing {\sl contains} by its body in class {\sl EmptySet}}$

    false
\end{lstlisting}
Or,
\begin{lstlisting}
    new NonEmptySet(7, new EmptySet, new EmptySet).contains(1)

->  $\rewriteby{by replacing {\sl contains} by its body in class {\sl NonEmptySet}}$

    if (1 < 7) new EmptySet contains 1
    else if (1 > 7) new EmptySet contains 1
    else true

->  $\rewriteby{by rewriting the conditional}$

    new EmptySet contains 1

->  $\rewriteby{by replacing {\sl contains} by its body in class {\sl EmptySet}}$

    false .
\end{lstlisting}

Dynamic method dispatch is analogous to higher-order function
calls. In both cases, the identity of code to be executed is known
only at run-time. This similarity is not just superficial. Indeed,
Scala represents every function value as an object (see
Section~\ref{sec:functions}).


\paragraph{Objects}

In the previous implementation of integer sets, empty sets were
expressed with \code{new EmptySet}; so a new object was created every time
an empty set value was required. We could have avoided unnecessary
object creations by defining a value \code{empty} once and then using
this value instead of every occurrence of \code{new EmptySet}. For example:
\begin{lstlisting}
val EmptySetVal = new EmptySet
\end{lstlisting}
One problem with this approach is that a value definition such as the
one above is not a legal top-level definition in Scala; it has to be
part of another class or object. Also, the definition of class
\code{EmptySet} now seems a bit of an overkill -- why define a class of objects, 
if we are only interested in a single object of this class? A more
direct approach is to use an {\em object definition}. Here is
a more streamlined alternative definition of the empty set:
\begin{lstlisting}
object EmptySet extends IntSet {
  def contains(x: Int): Boolean = false
  def incl(x: Int): IntSet = new NonEmptySet(x, EmptySet, EmptySet)
}
\end{lstlisting}
The syntax of an object definition follows the syntax of a class
definition; it has an optional extends clause as well as an optional
body. As is the case for classes, the extends clause defines inherited
members of the object whereas the body defines overriding or new
members.  However, an object definition defines a single object only
it is not possible to create other objects with the same structure
using \code{new}.  Therefore, object definitions also lack constructor
parameters, which might be present in class definitions.

Object definitions can appear anywhere in a Scala program; including
at top-level.  Since there is no fixed execution order of top-level
entities in Scala, one might ask exactly when the object defined by an
object definition is created and initialized. The answer is that the
object is created the first time one of its members is accessed. This
strategy is called {\em lazy evaluation}.

\paragraph{Standard Classes}

\todo{include picture}

Scala is a pure object-oriented language. This means that every value
in Scala can be regarded as an object.  In fact, even primitive types
such as \code{int} or \code{boolean} are not treated specially. They
are defined as type aliases of Scala classes in module \code{Predef}:
\begin{lstlisting}
type boolean = scala.Boolean
type int = scala.Int
type long = scala.Long
...
\end{lstlisting}
For efficiency, the compiler usually represents values of type
\code{scala.Int} by 32 bit integers, values of type
\code{scala.Boolean} by Java's booleans, etc.  But it converts these
specialized representations to objects when required, for instance
when a primitive \code{Int} value is passed to a function with a
parameter of type \code{AnyRef}.  Hence, the special representation of
primitive values is just an optimization, it does not change the
meaning of a program.

Here is a specification of class \code{Boolean}.
\begin{lstlisting}
package scala
abstract class Boolean {
  def && (x: => Boolean): Boolean
  def || (x: => Boolean): Boolean
  def !                 : Boolean

  def == (x: Boolean)   : Boolean
  def != (x: Boolean)   : Boolean
  def <  (x: Boolean)   : Boolean
  def >  (x: Boolean)   : Boolean
  def <= (x: Boolean)   : Boolean
  def >= (x: Boolean)   : Boolean
}
\end{lstlisting}
Booleans can be defined using only classes and objects, without
reference to a built-in type of booleans or numbers. A possible
implementation of class \code{Boolean} is given below.  This is not
the actual implementation in the standard Scala library. For
efficiency reasons the standard implementation uses built-in
booleans.
\begin{lstlisting}
package scala
abstract class Boolean {
  def ifThenElse(thenpart: => Boolean, elsepart: => Boolean)

  def && (x: => Boolean): Boolean  =  ifThenElse(x, false)
  def || (x: => Boolean): Boolean  =  ifThenElse(true, x)
  def !                 : Boolean  =  ifThenElse(false, true)

  def == (x: Boolean)   : Boolean  =  ifThenElse(x, x.!)
  def != (x: Boolean)   : Boolean  =  ifThenElse(x.!, x)
  def <  (x: Boolean)   : Boolean  =  ifThenElse(false, x)
  def >  (x: Boolean)   : Boolean  =  ifThenElse(x.!, false)
  def <= (x: Boolean)   : Boolean  =  ifThenElse(x, true)
  def >= (x: Boolean)   : Boolean  =  ifThenElse(true, x.!)
}
case object True extends Boolean {
  def ifThenElse(t: => Boolean, e: => Boolean) = t
}
case object False extends Boolean {
  def ifThenElse(t: => Boolean, e: => Boolean) = e
}
\end{lstlisting}
Here is a partial specification of class \code{Int}.

\begin{lstlisting}
package scala
abstract class Int extends AnyVal {
  def toLong: Long
  def toFloat: Float
  def toDouble: Double

  def + (that: Double): Double
  def + (that: Float): Float
  def + (that: Long): Long
  def + (that: Int): Int         // analogous for -, *, /, %

  def << (cnt: Int): Int         // analogous for >>, >>>

  def & (that: Long): Long
  def & (that: Int): Int         // analogous for |, ^

  def == (that: Double): Boolean
  def == (that: Float): Boolean
  def == (that: Long): Boolean   // analogous for !=, <, >, <=, >=
}
\end{lstlisting}

Class \code{Int} can in principle also be implemented using just
objects and classes, without reference to a built in type of
integers. To see how, we consider a slightly simpler problem, namely
how to implement a type \code{Nat} of natural (i.e. non-negative)
numbers. Here is the definition of an abstract class \code{Nat}:
\begin{lstlisting}
abstract class Nat {
  def isZero: Boolean
  def predecessor: Nat
  def successor: Nat
  def + (that: Nat): Nat
  def - (that: Nat): Nat
}
\end{lstlisting}
To implement the operations of class \code{Nat}, we define a sub-object
\code{Zero} and a subclass \code{Succ} (for successor). Each number
\code{N} is represented as \code{N} applications of the \code{Succ}
constructor to \code{Zero}:
\[
\underbrace{\mbox{\sl new Succ( ... new Succ}}_{\mbox{$N$ times}}\mbox{\sl (Zero) ... )}
\]
The implementation of the \code{Zero} object is straightforward:
\begin{lstlisting}
object Zero extends Nat {
  def isZero: Boolean = true
  def predecessor: Nat = error("negative number")
  def successor: Nat = new Succ(Zero)
  def + (that: Nat): Nat = that
  def - (that: Nat): Nat = if (that.isZero) Zero
                           else error("negative number")
}
\end{lstlisting}

The implementation of the predecessor and subtraction functions on
\code{Zero} throws an \code{Error} exception, which aborts the program
with the given error message.

Here is the implementation of the successor class:
\begin{lstlisting}
class Succ(x: Nat) extends Nat  {
  def isZero: Boolean = false
  def predecessor: Nat = x
  def successor: Nat = new Succ(this)
  def + (that: Nat): Nat = x + that.successor
  def - (that: Nat): Nat = if (that.isZero) this
                           else x - that.predecessor
}
\end{lstlisting}
Note the implementation of method \code{successor}. To create the
successor of a number, we need to pass the object itself as an
argument to the \code{Succ} constructor.  The object itself is
referenced by the reserved name \code{this}.   

The implementations of \code{+} and \code{-} each contain a recursive
call with the constructor argument as receiver. The recursion will
terminate once the receiver is the \code{Zero} object (which is
guaranteed to happen eventually because of the way numbers are formed).

\begin{exercise} Write an implementation \code{Integer} of integer numbers
The implementation should support all operations of class \code{Nat}
while adding two methods
\begin{lstlisting}
def isPositive: Boolean
def negate: Integer
\end{lstlisting}
The first method should return \code{true} if the number is positive. The second method should negate the number.
Do not use any of Scala's standard numeric classes in your
implementation. (Hint: There are two possible ways to implement
\code{Integer}. One can either make use the existing implementation of
\code{Nat}, representing an integer as a natural number and a sign.
Or one can generalize the given implementation of \code{Nat} to
\code{Integer}, using the three subclasses \code{Zero} for 0, 
\code{Succ} for positive numbers and \code{Pred} for negative numbers.)
\end{exercise}



\subsection*{Language Elements Introduced In This Chapter}

\textbf{Types:}
\begin{lstlisting}
Type         = ...  |  ident
\end{lstlisting}

Types can now be arbitrary identifiers which represent classes.

\textbf{Expressions:}
\begin{lstlisting}
Expr         = ...  |  Expr '.' ident  |  'new' Expr  |  'this'
\end{lstlisting}

An expression can now be an object creation, or
a selection \code{E.m} of a member \code{m}
from an object-valued expression \code{E}, or it can be the reserved name \code{this}.

\textbf{Definitions and Declarations:}
\begin{lstlisting}
Def          = FunDef  |  ValDef  |  ClassDef  |  TraitDef  |  ObjectDef
ClassDef     = ['abstract'] 'class' ident ['(' [Parameters] ')'] 
               ['extends' Expr] [`{' {TemplateDef} `}']
TraitDef     = 'trait' ident ['extends' Expr] ['{' {TemplateDef} '}']
ObjectDef    = 'object' ident ['extends' Expr] ['{' {ObjectDef} '}']
TemplateDef  = [Modifier] (Def | Dcl)
ObjectDef    = [Modifier] Def
Modifier     = 'private'  |  'override'
Dcl          = FunDcl  |  ValDcl
FunDcl       = 'def' ident {'(' [Parameters] ')'} ':' Type
ValDcl       = 'val' ident ':' Type
\end{lstlisting}

A definition can now be a class, trait or object definition such as
\begin{lstlisting}
class C(params) extends B { defs }
trait T extends B { defs }
object O extends B { defs }
\end{lstlisting}
The definitions \code{defs} in a class, trait or object may be
preceded by modifiers \code{private} or \code{override}.

Abstract classes and traits may also contain declarations. These
introduce {\em deferred} functions or values with their types, but do
not give an implementation. Deferred members have to be implemented in
subclasses before objects of an abstract class or trait can be created.

\chapter{Case Classes and Pattern Matching}

Say, we want to write an interpreter for arithmetic expressions.  To
keep things simple initially, we restrict ourselves to just numbers
and \code{+} operations. Such expressions can be represented as a class hierarchy, with an abstract base class \code{Expr} as the root, and two subclasses \code{Number} and
\code{Sum}. Then, an expression \code{1 + (3 + 7)} would be represented as
\begin{lstlisting}
new Sum(new Number(1), new Sum(new Number(3), new Number(7)))
\end{lstlisting}
Now, an evaluator of an expression like this needs to know of what
form it is (either \code{Sum} or \code{Number}) and also needs to
access the components of the expression.  The following
implementation provides all necessary methods.
\begin{lstlisting}
abstract class Expr {
  def isNumber: Boolean
  def isSum: Boolean
  def numValue: Int
  def leftOp: Expr
  def rightOp: Expr
}
class Number(n: Int) extends Expr {
  def isNumber: Boolean = true
  def isSum: Boolean = false
  def numValue: Int = n
  def leftOp: Expr = error("Number.leftOp")
  def rightOp: Expr = error("Number.rightOp")
}
class Sum(e1: Expr, e2: Expr) extends Expr {
  def isNumber: Boolean = false
  def isSum: Boolean = true
  def numValue: Int = error("Sum.numValue")
  def leftOp: Expr = e1
  def rightOp: Expr = e2
}
\end{lstlisting}
With these classification and access methods, writing an evaluator function is simple:
\begin{lstlisting}
def eval(e: Expr): Int = {
  if (e.isNumber) e.numValue
  else if (e.isSum) eval(e.leftOp) + eval(e.rightOp)
  else error("unrecognized expression kind")
}
\end{lstlisting}
However, defining all these methods in classes \code{Sum} and
\code{Number} is rather tedious. Furthermore, the problem becomes worse 
when we want to add new forms of expressions. For instance, consider
adding a new expression form
\code{Prod} for products. Not only do we have to implement a new class \code{Prod}, with all previous classification and access methods; we also have to introduce a
new abstract method \code{isProduct} in class \code{Expr} and
implement that method in subclasses \code{Number}, \code{Sum}, and
\code{Prod}. Having to modify existing code when a system grows is always problematic, since it introduces versioning and maintenance problems. 

The promise of object-oriented programming is that such modifications
should be unnecessary, because they can be avoided by re-using
existing, unmodified code through inheritance. Indeed, a more
object-oriented decomposition of our problem solves the problem.  The
idea is to make the ``high-level'' operation \code{eval} a method of
each expression class, instead of implementing it as a function
outside the expression class hierarchy, as we have done
before. Because \code{eval} is now a member of all expression nodes,
all classification and access methods become superfluous, and the implementation is simplified considerably:
\begin{lstlisting}
abstract class Expr {
  def eval: Int
}
class Number(n: Int) extends Expr {
  def eval: Int = n
}
class Sum(e1: Expr, e2: Expr) extends Expr {
  def eval: Int = e1.eval + e2.eval
}
\end{lstlisting}
Furthermore, adding a new \code{Prod} class does not entail any changes to existing code:
\begin{lstlisting}
class Prod(e1: Expr, e2: Expr) extends Expr {
  def eval: Int = e1.eval * e2.eval
}
\end{lstlisting}

The conclusion we can draw from this example is that object-oriented
decomposition is the technique of choice for constructing systems that
should be extensible with new types of data. But there is also another
possible way we might want to extend the expression example. We might
want to add new {\em operations} on expressions.  For instance, we might
want to add an operation that pretty-prints an expression tree to standard output.

If we have defined all classification and access methods, such an
operation can easily be written as an external function. Here is an
example:
\begin{lstlisting}
def print(e: Expr) {
  if (e.isNumber) Console.print(e.numValue)
  else if (e.isSum) {
    Console.print("(")
    print(e.leftOp)
    Console.print("+")
    print(e.rightOp)
    Console.print(")")
  } else error("unrecognized expression kind")
}
\end{lstlisting}
However, if we had opted for an object-oriented decomposition of
expressions, we would need to add a new \code{print} procedure
to each class:
\begin{lstlisting}
abstract class Expr {
  def eval: Int
  def print
}
class Number(n: Int) extends Expr {
  def eval: Int = n
  def print { Console.print(n) }
}
class Sum(e1: Expr, e2: Expr) extends Expr {
  def eval: Int = e1.eval + e2.eval
  def print {
    Console.print("(")
    print(e1)
    Console.print("+")
    print(e2)
    Console.print(")")
  }
}
\end{lstlisting}
Hence, classical object-oriented decomposition requires modification
of all existing classes when a system is extended with new operations.

As yet another way we might want to extend the interpreter, consider
expression simplification. For instance, we might want to write a
function which rewrites expressions of the form
\code{a * b + a * c} to \code{a * (b + c)}. This operation requires inspection of 
more than a single node of the expression tree at the same
time. Hence, it cannot be implemented by a method in each expression
kind, unless that method can also inspect other nodes. So we are
forced to have classification and access methods in this case. This
seems to bring us back to square one, with all the problems of
verbosity and extensibility.

Taking a closer look, one observes that the only purpose of the
classification and access functions is to {\em reverse} the data
construction process.  They let us determine, first, which sub-class
of an abstract base class was used and, second, what were the
constructor arguments. Since this situation is quite common, Scala has
a way to automate it with case classes. 

\section{Case Classes and Case Objects}

{\em Case classes} and {\em case objects} are defined like a normal
classes or objects, except that the definition is prefixed with the modifier
\code{case}.  For instance, the definitions
\begin{lstlisting}
abstract class Expr
case class Number(n: Int) extends Expr
case class Sum(e1: Expr, e2: Expr) extends Expr
\end{lstlisting}
introduce \code{Number} and \code{Sum} as case classes.
The \code{case} modifier in front of a class or object 
definition has the following effects.
\begin{enumerate}
\item Case classes implicitly come with a constructor function, with the same name as the class. In our example, the two functions
\begin{lstlisting}
def Number(n: Int) = new Number(n)
def Sum(e1: Expr, e2: Expr) = new Sum(e1, e2)
\end{lstlisting}
would be added. Hence, one can now construct expression trees a bit more concisely, as in
\begin{lstlisting}
Sum(Sum(Number(1), Number(2)), Number(3))
\end{lstlisting} 
\item Case classes and case objects 
implicitly come with implementations of methods
\code{toString}, \code{equals} and \code{hashCode}, which override the
methods with the same name in class \code{AnyRef}. The implementation
of these methods takes in each case the structure of a member of a
case class into account. The \code{toString} method represents an
expression tree the way it was constructed. So,
\begin{lstlisting}
Sum(Sum(Number(1), Number(2)), Number(3))
\end{lstlisting} 
would be converted to exactly that string, whereas the default
implementation in class \code{AnyRef} would return a string consisting
of the outermost constructor name \code{Sum} and a number.  The
\code{equals} methods treats two case members of a case class as equal
if they have been constructed with the same constructor and with
arguments which are themselves pairwise equal. This also affects the
implementation of \code{==} and \code{!=}, which are implemented in
terms of \code{equals} in Scala. So,
\begin{lstlisting}
Sum(Number(1), Number(2)) == Sum(Number(1), Number(2))
\end{lstlisting}
will yield \code{true}. If \code{Sum} or \code{Number} were not case
classes, the same expression would be \code{false}, since the standard
implementation of \code{equals} in class \code{AnyRef} always treats
objects created by different constructor calls as being different.
The \code{hashCode} method follows the same principle as other two
methods. It computes a hash code from the case class constructor name
and the hash codes of the constructor arguments, instead of from the object's
address, which is what the as the default implementation of \code{hashCode} does.
\item 
Case classes implicitly come with nullary accessor methods which
retrieve the constructor arguments.
In our example, \code{Number} would obtain an accessor method
\begin{lstlisting}
def n: Int
\end{lstlisting}
which returns the constructor parameter \code{n}, whereas \code{Sum} would obtain two accessor methods
\begin{lstlisting}
def e1: Expr, e2: Expr
\end{lstlisting}
Hence, if for a value \code{s} of type \code{Sum}, say, one can now
write \code{s.e1}, to access the left operand. However, for a value
\code{e} of type \code{Expr}, the term \code{e.e1} would be illegal
since \code{e1} is defined in \code{Sum}; it is not a member of the
base class \code{Expr}. 
So, how do we determine the constructor and access constructor
arguments for values whose static type is the base class \code{Expr}?
This is solved by the fourth and final particularity of case classes.
\item 
Case classes allow the constructions of {\em patterns} which refer to
the case class constructor.
\end{enumerate}

\section{Pattern Matching}

Pattern matching is a generalization of C or Java's \code{switch}
statement to class hierarchies. Instead of a \code{switch} statement,
there is a standard method \code{match}, which is defined in Scala's
root class \code{Any}, and therefore is available for all objects.
The \code{match} method takes as argument a number of cases. 
For instance, here is an implementation of \code{eval} using 
pattern matching.
\begin{lstlisting}
def eval(e: Expr): Int = e match {
  case Number(x) => x 
  case Sum(l, r) => eval(l) + eval(r) 
}
\end{lstlisting}
In this example, there are two cases. Each case associates a pattern
with an expression. Patterns are matched against the selector
values \code{e}.  The first pattern in our example,
\code{Number(n)}, matches all values of the form \code{Number(v)}, 
where \code{v} is an arbitrary value.  In that case, the {\em pattern
variable} \code{n} is bound to the value \code{v}. Similarly, the
pattern \code{Sum(l, r)} matches all selector values of form
\code{Sum(v}$_1$\code{, v}$_2$\code{)} and binds the pattern variables
\code{l} and \code{r} 
to \code{v}$_1$ and \code{v}$_2$, respectively. 

In general, patterns are built from
\begin{itemize}
\item Case class constructors, e.g. \code{Number}, \code{Sum}, whose arguments
      are again patterns,
\item pattern variables, e.g. \code{n}, \code{e1}, \code{e2},
\item the ``wildcard'' pattern \code{_},
\item literals, e.g. \code{1}, \code{true}, "abc", 
\item constant identifiers, e.g. \code{MAXINT}, \code{EmptySet}.
\end{itemize}
Pattern variables always start with a lower-case letter, so that they
can be distinguished from constant identifiers, which start with an
upper case letter.  Each variable name may occur only once in a
pattern. For instance, \code{Sum(x, x)} would be illegal as a pattern,
since the pattern variable \code{x} occurs twice in it.

\paragraph{Meaning of Pattern Matching}
A pattern matching expression 
\begin{lstlisting}
e match { case p$_1$ => e$_1$ ... case p$_n$ => e$_n$ }
\end{lstlisting}
matches the patterns $p_1 \commadots p_n$ in the order they
are written against the selector value \code{e}.
\begin{itemize}
\item
A constructor pattern $C(p_1 \commadots p_n)$ matches all values that
are of type \code{C} (or a subtype thereof) and that have been constructed with 
\code{C}-arguments matching patterns $p_1 \commadots p_n$.
\item 
A variable pattern \code{x} matches any value and binds the variable
name to that value.  
\item 
The wildcard pattern `\code{_}' matches any value but does not bind a name to that value. 
\item A constant pattern \code{C} matches a value which is
equal (in terms of \code{==}) to \code{C}.
\end{itemize}
The pattern matching expression rewrites to the right-hand-side of the
first case whose pattern matches the selector value. References to
pattern variables are replaced by corresponding constructor arguments.
If none of the patterns matches, the pattern matching expression is
aborted with a \code{MatchError} exception.

\example Our substitution model of program evaluation extends quite naturally to pattern matching, For instance, here is how \code{eval} applied to a simple expression is re-written:
\begin{lstlisting}
     eval(Sum(Number(1), Number(2)))

->   $\mbox{\tab\tab\rm(by rewriting the application)}$

     Sum(Number(1), Number(2)) match {
         case Number(n) => n
         case Sum(e1, e2) => eval(e1) + eval(e2)
     }

->   $\mbox{\tab\tab\rm(by rewriting the pattern match)}$

     eval(Number(1)) + eval(Number(2))

->   $\mbox{\tab\tab\rm(by rewriting the first application)}$

     Number(1) match {
         case Number(n) => n
         case Sum(e1, e2) => eval(e1) + eval(e2)
     } + eval(Number(2))

->   $\mbox{\tab\tab\rm(by rewriting the pattern match)}$

     1 + eval(Number(2))

->$^*$ 1 + 2 -> 3
\end{lstlisting}

\paragraph{Pattern Matching and Methods}
In the previous example, we have used pattern
matching in a function which was defined outside the class hierarchy
over which it matches.  Of course, it is also possible to define a
pattern matching function in that class hierarchy itself. For
instance, we could have defined
\code{eval} is a method of the base class \code{Expr}, and still have used pattern matching in its implementation:
\begin{lstlisting}
abstract class Expr { 
  def eval: Int = this match {
    case Number(n) => n
    case Sum(e1, e2) => e1.eval + e2.eval 
  } 
}
\end{lstlisting}

\begin{exercise} Consider the following definitions representing trees
of integers.  These definitions can be seen as an alternative
representation of \code{IntSet}:
\begin{lstlisting}
abstract class IntTree
case object EmptyTree extends IntTree
case class  Node(elem: Int, left: IntTree, right: IntTree) extends IntTree
\end{lstlisting}
Complete the following implementations of function \code{contains} and \code{insert} for 
\code{IntTree}'s.
\begin{lstlisting}
def contains(t: IntTree, v: Int): Boolean = t match { ...
  ...
}
def insert(t: IntTree, v: Int): IntTree = t match { ...
  ...
}
\end{lstlisting}
\end{exercise}

\paragraph{Pattern Matching Anonymous Functions}

So far, case-expressions always appeared in conjunction with a
\verb@match@ operation. But it is also possible to use
case-expressions by themselves. A block of case-expressions such as
\begin{lstlisting}
{ case $P_1$ => $E_1$ ... case $P_n$ => $E_n$ }
\end{lstlisting}
is seen by itself as a function which matches its arguments
against the patterns $P_1 \commadots P_n$, and produces the result of
one of $E_1 \commadots E_n$. (If no pattern matches, the function
would throw a \code{MatchError} exception instead).
In other words, the expression above is seen as a shorthand for the anonymous function
\begin{lstlisting}
(x => x match { case $P_1$ => $E_1$ ... case $P_n$ => $E_n$ })
\end{lstlisting}
where \code{x} is a fresh variable which is not used 
otherwise in the expression.

\chapter{Generic Types and Methods}

Classes in Scala can have type parameters. We demonstrate the use of
type parameters with functional stacks as an example. Say, we want to
write a data type of stacks of integers, with methods \code{push},
\code{top}, \code{pop}, and \code{isEmpty}. This is achieved by the
following class hierarchy:
\begin{lstlisting}
abstract class IntStack {
  def push(x: Int): IntStack = new IntNonEmptyStack(x, this)
  def isEmpty: Boolean
  def top: Int
  def pop: IntStack
}
class IntEmptyStack extends IntStack {
  def isEmpty = true
  def top = error("EmptyStack.top")
  def pop = error("EmptyStack.pop")
}
class IntNonEmptyStack(elem: Int, rest: IntStack) extends IntStack {
  def isEmpty = false
  def top = elem
  def pop = rest
}
\end{lstlisting}
Of course, it would also make sense to define an abstraction for a
stack of Strings. To do that, one could take the existing abstraction
for \code{IntStack}, rename it to \code{StringStack} and at the same
time rename all occurrences of type \code{Int} to \code{String}.

A better way, which does not entail code duplication, is to
parameterize the stack definitions with the element type.
Parameterization lets us generalize from a specific instance of a
problem to a more general one. So far, we have used parameterization
only for values, but it is available also for types. To arrive at a
{\em generic} version of \code{Stack}, we equip it with a type
parameter.
\begin{lstlisting}
abstract class Stack[A] {
  def push(x: A): Stack[A] = new NonEmptyStack[A](x, this)
  def isEmpty: Boolean
  def top: A
  def pop: Stack[A]
}
class EmptyStack[A] extends Stack[A] {
  def isEmpty = true
  def top = error("EmptyStack.top")
  def pop = error("EmptyStack.pop")
}
class NonEmptyStack[A](elem: A, rest: Stack[A]) extends Stack[A] {
  def isEmpty = false
  def top = elem
  def pop = rest
}
\end{lstlisting}
In the definitions above, `\code{A}' is a {\em type parameter} of
class \code{Stack} and its subclasses.  Type parameters are arbitrary
names; they are enclosed in brackets instead of parentheses, so that
they can be easily distinguished from value parameters.  Here is an
example how the generic classes are used:
\begin{lstlisting}
val x = new EmptyStack[Int]
val y = x.push(1).push(2)
println(y.pop.top)
\end{lstlisting}
The first line creates a new empty stack of \code{Int}'s. Note the
actual type argument \code{[Int]} which replaces the formal type
parameter \code{A}.

It is also possible to parameterize methods with types. As an example,
here is a generic method which determines whether one stack is a
prefix of another.
\begin{lstlisting}
def isPrefix[A](p: Stack[A], s: Stack[A]): Boolean = {
  p.isEmpty ||
  p.top == s.top && isPrefix[A](p.pop, s.pop)
}
\end{lstlisting}  
The method parameters are called {\em polymorphic}.  Generic methods are
also called {\em polymorphic}.  The term comes from the Greek, where it
means ``having many forms''.  To apply a polymorphic method such as
\code{isPrefix}, we pass type parameters as well as value parameters
to it. For instance,
\begin{lstlisting}
val s1 = new EmptyStack[String].push("abc")
val s2 = new EmptyStack[String].push("abx").push(s1.top)
println(isPrefix[String](s1, s2))
\end{lstlisting}

\paragraph{Local Type Inference}
Passing type parameters such as \code{[Int]} or \code{[String]} all
the time can become tedious in applications where generic functions
are used a lot. Quite often, the information in a type parameter is
redundant, because the correct parameter type can also be determined
by inspecting the function's value parameters or expected result type.
Taking the expression \code{isPrefix[String](s1, s2)} as an
example, we know that its value parameters are both of type
\code{Stack[String]}, so we can deduce that the type parameter must
be \code{String}. Scala has a fairly powerful type inferencer which
allows one to omit type parameters to polymorphic functions and
constructors in situations like these.  In the example above, one
could have written \code{isPrefix(s1, s2)} and the missing type argument
\code{[String]} would have been inserted by the type inferencer. 

\section{Type Parameter Bounds}

Now that we know how to make classes generic it is natural to
generalize some of the earlier classes we have written. For instance
class \code{IntSet} could be generalized to sets with arbitrary
element types. Let's try. The abstract class for generic sets is easily
written.
\begin{lstlisting}
abstract class Set[A] {
  def incl(x: A): Set[A]
  def contains(x: A): Boolean
}
\end{lstlisting}
However, if we still want to implement sets as binary search trees, we
encounter a problem. The \code{contains} and \code{incl} methods both
compare elements using methods \code{<} and \code{>}. For
\code{IntSet} this was OK, since type \code{Int} has these two
methods. But for an arbitrary type parameter \code{a}, we cannot
guarantee this. Therefore, the previous implementation of, say,
\code{contains} would generate a compiler error.
\begin{lstlisting}
  def contains(x: Int): Boolean =
    if (x < elem) left contains x
          ^ < $\mbox{\sl not a member of type}$ A.
\end{lstlisting}
One way to solve the problem is to restrict the legal types that can
be substituted for type \code{A} to only those types that contain methods
\code{<} and \code{>} of the correct types. There is a trait
\code{Ordered[A]} in the standard class library Scala which represents
values which are comparable (via \code{<} and \code{>}) to values of
type \code{A}. This trait is defined as follows:
\begin{lstlisting}
/** A class for totally ordered data. */
trait Ordered[A] {

  /** Result of comparing `this' with operand `that'.
   *  returns `x' where
   *  x < 0    iff    this < that
   *  x == 0   iff    this == that
   *  x > 0    iff    this > that
   */
  def compare(that: A): Int

  def <  (that: A): Boolean = (this compare that) <  0
  def >  (that: A): Boolean = (this compare that) >  0
  def <= (that: A): Boolean = (this compare that) <= 0
  def >= (that: A): Boolean = (this compare that) >= 0
  def compareTo(that: A): Int = compare(that)
}
\end{lstlisting}
We can enforce the comparability of a type by demanding
that the type is a subtype of \code{Ordered}. This is done by giving an
upper bound to the type parameter of \code{Set}:
\begin{lstlisting}
trait Set[A <: Ordered[A]] {
  def incl(x: A): Set[A]
  def contains(x: A): Boolean
}
\end{lstlisting}
The parameter declaration \code{A <: Ordered[A]} introduces \code{A} as a
type parameter which must be a subtype of \code{Ordered[A]}, i.e.\ its values
must be comparable to values of the same type.

With this restriction, we can now implement the rest of the generic
set abstraction as we did in the case of \code{IntSet}s before.

\begin{lstlisting}
class EmptySet[A <: Ordered[A]] extends Set[A] {
  def contains(x: A): Boolean = false
  def incl(x: A): Set[A] = new NonEmptySet(x, new EmptySet[A], new EmptySet[A])
}
\end{lstlisting}

\begin{lstlisting}
class NonEmptySet[A <: Ordered[A]]
        (elem: A, left: Set[A], right: Set[A]) extends Set[A] {
  def contains(x: A): Boolean =
    if (x < elem) left contains x
    else if (x > elem) right contains x
    else true
  def incl(x: A): Set[A] =
    if (x < elem) new NonEmptySet(elem, left incl x, right)
    else if (x > elem) new NonEmptySet(elem, left, right incl x)
    else this
}
\end{lstlisting}
Note that we have left out the type argument in the object creations
\code{new NonEmptySet(...)}. In the same way as for polymorphic methods,
missing type arguments in constructor calls are inferred from value
arguments and/or the expected result type.

Here is an example that uses the generic set abstraction. Let's first
create a subclass of \lstinline@Ordered@, like this:
\begin{lstlisting}
case class Num(value: Double) extends Ordered[Num] {
  def compare(that: Num): Int =
    if (this.value < that.value) -1
    else if (this.value > that.value) 1
    else 0
}
\end{lstlisting}
Then:
\begin{lstlisting}
val s = new EmptySet[Num].incl(Num(1.0)).incl(Num(2.0))
s.contains(Num(1.5))
\end{lstlisting}
This is OK, as type \code{Num} implements the trait \code{Ordered[Num]}.
However, the following example is in error.
\begin{lstlisting}
val s = new EmptySet[java.io.File]
                    ^ java.io.File $\mbox{\sl does not conform to type}$
                      $\mbox{\sl parameter bound}$ Ordered[java.io.File].
\end{lstlisting}
One probem with type parameter bounds is that they require
forethought: if we had not declared \lstinline@Num@ a subclass of
\lstinline@Ordered@, we would not have been able to use \lstinline@Num@
elements in sets. By the same token, types inherited from Java, 
such as \lstinline@Int@, \lstinline@Double@, or
\lstinline@String@ are not subclasses of \lstinline@Ordered@, so
values of these types cannot be used as set elements.

A more flexible design, which admits elements of these types, uses
{\em view bounds} instead of the plain type bounds we have seen so
far. The only change this entails in the example above is in the type
parameters:
\begin{lstlisting}
trait Set[A <% Ordered[A]] ...
class EmptySet[A <% Ordered[A]] ...
class NonEmptySet[A <% Ordered[A]] ...
\end{lstlisting}
View bounds \lstinline@<%@ are weaker than plain bounds \verb@<:@:
A view bounded type parameter clause \lstinline@[A <% T]@ only
specifies that the bounded type \lstinline@A@ must be {\em convertible} to
the bound type \lstinline@T@, using an implicit conversion.

The Scala library predefines implicit conversions for a number of
types, including the primitive types and \lstinline@String@.
Therefore, the redesign set abstraction can be instantiated with these
types as well. More explanations on implicit conversions and view
bounds are given in Section~\ref{sec:implicits}.

\section{Variance Annotations}\label{sec:first-arrays}

The combination of type parameters and subtyping poses some
interesting questions. For instance, should \code{Stack[String]} be a
subtype of \code{Stack[AnyRef]}? Intuitively, this seems OK, since a
stack of \code{String}s is a special case of a stack of
\code{AnyRef}s.  More generally, if \code{T} is a subtype of type \code{S}
then \code{Stack[T]} should be a subtype of \code{Stack[S]}. 
This property is called {\em co-variant} subtyping.

In Scala, generic types have by default non-variant subtyping. That
is, with \code{Stack} defined as above, stacks with different element
types would never be in a subtype relation. However, we can enforce
co-variant subtyping of stacks by changing the first line of the
definition of class \code{Stack} as follows.
\begin{lstlisting}
class Stack[+A] {
\end{lstlisting}
Prefixing a formal type parameter with a \code{+} indicates that
subtyping is covariant in that parameter. 
Besides \code{+}, there is also a prefix \code{-} which indicates
contra-variant subtyping. If \code{Stack} was defined \code{class
Stack[-A] ...}, then \code{T} a subtype of type \code{S} would imply
that \code{Stack[S]} is a subtype of \code{Stack[T]} (which in the
case of stacks would be rather surprising!).

In a purely functional world, all types could be co-variant. However,
the situation changes once we introduce mutable data. Consider the
case of arrays in Java or .NET. Such arrays are represented in Scala
by a generic class \code{Array}. Here is a partial definition of this
class.
\begin{lstlisting}
class Array[A] {
  def apply(index: Int): A
  def update(index: Int, elem: A)
}
\end{lstlisting}
The class above defines the way Scala arrays are seen from Scala user
programs. The Scala compiler will map this abstraction to the
underlying arrays of the host system in most cases where this
possible.

In Java, arrays are indeed covariant; that is, for reference types
\code{T} and \code{S}, if \code{T} is a subtype of \code{S}, then also
\code{Array[T]} is a subtype of \code{Array[S]}. This might seem
natural but leads to safety problems that require special runtime
checks. Here is an example:
\begin{lstlisting}
val x = new Array[String](1)
val y: Array[Any] = x
y(0) = new Rational(1, 2)  // this is syntactic sugar for
                           // y.update(0, new Rational(1, 2))
\end{lstlisting}
In the first line, a new array of strings is created. In the second
line, this array is bound to a variable \code{y}, of type
\code{Array[Any]}.  Assuming arrays are covariant, this is OK, since
\code{Array[String]} is a subtype of \code{Array[Any]}. Finally, in
the last line a rational number is stored in the array. This is also
OK, since type \code{Rational} is a subtype of the element type
\code{Any} of the array \code{y}. We thus end up storing a rational
number in an array of strings, which clearly violates type soundness. 

Java solves this problem by introducing a run-time check in the third
line which tests whether the stored element is compatible with the
element type with which the array was created. We have seen in the
example that this element type is not necessarily the static element
type of the array being updated. If the test fails, an
\code{ArrayStoreException} is raised.

Scala solves this problem instead statically, by disallowing the
second line at compile-time, because arrays in Scala have non-variant
subtyping. This raises the question how a Scala compiler verifies that
variance annotations are correct. If we had simply declared arrays
co-variant, how would the potential problem have been detected?

Scala uses a conservative approximation to verify soundness of
variance annotations.  A covariant type parameter of a class may only
appear in co-variant positions inside the class.  Among the co-variant
positions are the types of values in the class, the result types of
methods in the class, and type arguments to other covariant types. Not
co-variant are types of formal method parameters. Hence, the following
class definition would have been rejected
\begin{lstlisting}
class Array[+A] {
  def apply(index: Int): A
  def update(index: Int, elem: A)
                               ^ $\mbox{\sl covariant type parameter}$ A
                                 $\mbox{\sl appears in contravariant position.}$
}
\end{lstlisting}
So far, so good. Intuitively, the compiler was correct in rejecting
the \code{update} procedure in a co-variant class because \code{update}
potentially changes state, and therefore undermines the soundness of
co-variant subtyping. 

However, there are also methods which do not mutate state, but where a
type parameter still appears contra-variantly. An example is
\code{push} in type \code{Stack}. Again the Scala compiler will reject
the definition of this method for co-variant stacks.
\begin{lstlisting}
class Stack[+A] {
  def push(x: A): Stack[A] =
              ^ $\mbox{\sl covariant type parameter}$ A
                $\mbox{\sl appears in contravariant position.}$
\end{lstlisting}
This is a pity, because, unlike arrays, stacks are purely functional data
structures and therefore should enable co-variant subtyping. However,
there is a a way to solve the problem by using a polymorphic method
with a lower type parameter bound.

\section{Lower Bounds}

We have seen upper bounds for type parameters. In a type parameter
declaration such as \code{T <: U}, the type parameter \code{T} is
restricted to range only over subtypes of type \code{U}. Symmetrical
to this are lower bounds in Scala. In a type parameter declaration
\code{T >: S}, the type parameter \code{T} is restricted to range only
over {\em supertypes} of type \code{S}. (One can also combine lower and
upper bounds, as in \code{T >: S <: U}.)

Using lower bounds, we can generalize the \code{push} method in
\code{Stack} as follows.
\begin{lstlisting}
class Stack[+A] {
  def push[B >: A](x: B): Stack[B] = new NonEmptyStack(x, this)
\end{lstlisting}
Technically, this solves our variance problem since now the type
parameter \code{A} appears no longer as a parameter type of method
\code{push}. Instead, it appears as lower bound for another type
parameter of a method, which is classified as a co-variant position.
Hence, the Scala compiler accepts the new definition of \code{push}.

In fact, we have not only solved the technical variance problem but
also have generalized the definition of \code{push}.  Before, we were
required to push only elements with types that conform to the declared
element type of the stack. Now, we can push also elements of a
supertype of this type, but the type of the returned stack will change
accordingly. For instance, we can now push an \code{AnyRef} onto a
stack of \code{String}s, but the resulting stack will be a stack of
\code{AnyRef}s instead of a stack of \code{String}s!

In summary, one should not hesitate to add variance annotations to
your data structures, as this yields rich natural subtyping
relationships. The compiler will detect potential soundness
problems. Even if the compiler's approximation is too conservative, as
in the case of method \code{push} of class \code{Stack}, this will
often suggest a useful generalization of the contested method.

\section{Least Types}

Scala does not allow one to parameterize objects with types. That's
why we originally defined a generic class \code{EmptyStack[A]}, even
though a single value denoting empty stacks of arbitrary type would
do. For co-variant stacks, however, one can use the following idiom:
\begin{lstlisting}
object EmptyStack extends Stack[Nothing] { ... }
\end{lstlisting}
The bottom type \code{Nothing} contains no value, so the type
\code{Stack[Nothing]} expresses the fact that an \code{EmptyStack}
contains no elements. Furthermore, \code{Nothing} is a subtype of all
other types. Hence, for co-variant stacks, \code{Stack[Nothing]} is a
subtype of \code{Stack[T]}, for any other type \code{T}. This makes it
possible to use a single empty stack object in user code. For
instance:
\begin{lstlisting}
val s = EmptyStack.push("abc").push(new AnyRef())
\end{lstlisting}
Let's analyze the type assignment for this expression in detail.  The
\code{EmptyStack} object is of type \code{Stack[Nothing]}, which has a
method
\begin{lstlisting}
push[B >: Nothing](elem: B): Stack[B] .
\end{lstlisting}
Local type inference will determine that the type parameter \code{B}
should be instantiated to \code{String} in the application 
\code{EmptyStack.push("abc")}. The result type of that application is hence
\code{Stack[String]}, which in turn has a method
\begin{lstlisting}
push[B >: String](elem: B): Stack[B] .
\end{lstlisting}
The final part of the value definition above is the application of
this method to \code{new AnyRef()}. Local type inference will
determine that the type parameter \code{b} should this time be
instantiated to \code{AnyRef}, with result type \code{Stack[AnyRef]}.
Hence, the type assigned to value \code{s} is \code{Stack[AnyRef]}.

Besides \code{Nothing}, which is a subtype of every other type, there
is also the type \code{Null}, which is a subtype of
\code{scala.AnyRef}, and every class derived from it. The \code{null}
literal in Scala is the only value of that type. This makes
\code{null} compatible with every reference type, but not with a value
type such as \code{Int}.

We conclude this section with the complete improved definition of
stacks. Stacks have now co-variant subtyping, the \code{push} method
has been generalized, and the empty stack is represented by a single
object.
\begin{lstlisting}
abstract class Stack[+A] {
  def push[B >: A](x: B): Stack[B] = new NonEmptyStack(x, this)
  def isEmpty: Boolean
  def top: A
  def pop: Stack[A]
}
object EmptyStack extends Stack[Nothing] {
  def isEmpty = true
  def top = error("EmptyStack.top")
  def pop = error("EmptyStack.pop")
}
class NonEmptyStack[+A](elem: A, rest: Stack[A]) extends Stack[A] {
  def isEmpty = false
  def top = elem
  def pop = rest
}
\end{lstlisting}
Many classes in the Scala library are generic. We now present two
commonly used families of generic classes, tuples and functions. The
discussion of another common class, lists, is deferred to the next
chapter.

\section{Tuples}

Sometimes, a function needs to return more than one result. For
instance, take the function \code{divmod} which returns the integer quotient
and rest of two given integer arguments.  Of course, one can define a
class to hold the two results of \code{divmod}, as in:
\begin{lstlisting}
case class TwoInts(first: Int, second: Int)
def divmod(x: Int, y: Int): TwoInts = new TwoInts(x / y, x % y)
\end{lstlisting}
However, having to define a new class for every possible pair of
result types is very tedious. In Scala one can use instead a
generic class \lstinline@Tuple$2$@, which is defined as follows:
\begin{lstlisting}
package scala
case class Tuple2[A, B](_1: A, _2: B)
\end{lstlisting}
With \code{Tuple2}, the \code{divmod} method can be written as follows.
\begin{lstlisting}
def divmod(x: Int, y: Int) = new Tuple2[Int, Int](x / y, x % y)
\end{lstlisting}
As usual, type parameters to constructors can be omitted if they are
deducible from value arguments. There exist also tuple classes for
every other number of elements (the current Scala implementation
limits this to tuples of some reasonable number of elements).  

\comment{
Also, Scala defines an alias \code{Pair} for \code{Tuple2} (as well as
\code{Triple} for \code{Tuple3}).  With these conventions,
\code{divmod} can equivalently be written as follows.
\begin{lstlisting}
def divmod(x: Int, y: Int): Pair(Int, Int) = Pair(x / y, x % y)
\end{lstlisting}
}
How are elements of tuples accessed? Since tuples are case classes,
there are two possibilities. One can either access a tuple's fields
using the names of the constructor parameters \lstinline@_$i$@, as in the following example:
\begin{lstlisting}
val xy = divmod(x, y)
println("quotient: " + xy._1 + ", rest: " + xy._2)
\end{lstlisting}
Or one uses pattern matching on tuples, as in the following example:
\begin{lstlisting}
divmod(x, y) match {
  case Tuple2(n, d) =>
    println("quotient: " + n + ", rest: " + d)
}
\end{lstlisting}
Note that type parameters are never used in patterns; it would have
been illegal to write case \code{Tuple2[Int, Int](n, d)}.

Tuples are so convenient that Scala defines special syntax for
them. To form a tuple with $n$ elements $x_1 \commadots x_n$ one can
write $(x_1 \commadots x_n)$. This is equivalent to
\lstinline@Tuple$n$($x_1 \commadots x_n$)@. The $(...)$ syntax works
equivalently for types and for patterns. With that tuple syntax, the
\lstinline@divmod@ example is written as follows:
\begin{lstlisting}
def divmod(x: Int, y: Int): (Int, Int) = (x / y, x % y)
divmod(x, y) match {
  case (n, d) => println("quotient: " + n + ", rest: " + d)
}
\end{lstlisting}
\section{Functions}\label{sec:functions}

Scala is a functional language in that functions are first-class
values.  Scala is also an object-oriented language in that every value
is an object.  It follows that functions are objects in Scala.  For
instance, a function from type \code{String} to type \code{Int} is
represented as an instance of the trait \code{Function1[String, Int]}.
The \code{Function1} trait is defined as follows.
\begin{lstlisting}
package scala
trait Function1[-A, +B] {
  def apply(x: A): B
}
\end{lstlisting}
Besides \code{Function1}, there are also definitions of for functions
of all other arities (the current implementation implements this only
up to a reasonable limit).  That is, there is one definition for each
possible number of function parameters.  Scala's function type syntax
~\lstinline@$(T_1 \commadots T_n)$ => $S$@~ is simply an abbreviation
for the parameterized type ~\lstinline@Function$n$[$T_1 \commadots
T_n, S$]@~.

Scala uses the same syntax $f(x)$ for function application, no matter
whether $f$ is a method or a function object. This is made possible by
the following convention: A function application $f(x)$ where $f$ is
an object (as opposed to a method) is taken to be a shorthand for
\lstinline@$f$.apply($x$)@. Hence, the \code{apply} method of a
function type is inserted automatically where this is necessary.

That's also why we defined array subscripting in
Section~\ref{sec:first-arrays} by an \code{apply} method.  For any
array \code{a}, the subscript operation \code{a(i)} is taken to be a
shorthand for \code{a.apply(i)}.

Functions are an example where a contra-variant type parameter
declaration is useful. For example, consider the following code:
\begin{lstlisting}
val f: (AnyRef => Int)  =  x => x.hashCode()
val g: (String => Int)  =  f
g("abc")
\end{lstlisting}
It's sound to bind the value \code{g} of type \code{String => Int} to
\code{f}, which is of type \code{AnyRef => Int}. Indeed, all one can
do with function of type \code{String => Int} is pass it a string in
order to obtain an integer. Clearly, the same works for function
\code{f}: If we pass it a string (or any other object), we obtain an
integer.  This demonstrates that function subtyping is contra-variant
in its argument type whereas it is covariant in its result type.
In short, $S \Rightarrow T$ is a subtype of $S' \Rightarrow T'$, provided
$S'$ is a subtype of $S$ and $T$ is a subtype of $T'$.

\example Consider the Scala code
\begin{lstlisting}
val plus1: (Int => Int)  =  (x: Int) => x + 1
plus1(2)
\end{lstlisting}
This is expanded into the following object code.
\begin{lstlisting}
val plus1: Function1[Int, Int] = new Function1[Int, Int] {
  def apply(x: Int): Int = x + 1
}
plus1.apply(2)
\end{lstlisting}
Here, the object creation \lstinline@new Function1[Int, Int]{ ... }@
represents an instance of an {\em anonymous class}. It combines the
creation of a new \code{Function1} object with an implementation of 
the \code{apply} method (which is abstract in \code{Function1}).
Equivalently, but more verbosely, one could have used a local class:
\begin{lstlisting}
val plus1: Function1[Int, Int] = {
  class Local extends Function1[Int, Int] {
    def apply(x: Int): Int = x + 1
  }
  new Local: Function1[Int, Int]
}
plus1.apply(2)
\end{lstlisting}
 
\chapter{Lists}

Lists are an important data structure in many Scala programs.  
A list containing the elements \code{x}$_1$, \ldots, \code{x}$_n$ is written
\code{List(x}$_1$\code{, ..., x}$_n$\code{)}. Examples are:
\begin{lstlisting}
val fruit = List("apples", "oranges", "pears")
val nums  = List(1, 2, 3, 4)
val diag3 = List(List(1, 0, 0), List(0, 1, 0), List(0, 0, 1))
val empty = List()
\end{lstlisting}
Lists are similar to arrays in languages such as C or Java, but there
are also three important differences. First, lists are immutable. That
is, elements of a list cannot be changed by assignment. Second, 
lists have a recursive structure, whereas arrays are flat. Third,
lists support a much richer set of operations than arrays usually do.

\section{Using Lists}

\paragraph{The List type}
Like arrays, lists are {\em homogeneous}. That is, the elements of a
list all have the same type.  The type of a list with elements of type
\code{T} is written \code{List[T]} (compare to \code{T[]} in Java).
\begin{lstlisting}
val fruit: List[String]    = List("apples", "oranges", "pears")
val nums : List[Int]       = List(1, 2, 3, 4)
val diag3: List[List[Int]] = List(List(1, 0, 0), List(0, 1, 0), List(0, 0, 1))
val empty: List[Int]       = List()
\end{lstlisting}

\paragraph{List constructors}
All lists are built from two more fundamental constructors, \code{Nil}
and \code{::} (pronounced ``cons''). \code{Nil} represents an empty
list. The infix operator \code{::} expresses list extension. That is,
\code{x :: xs} represents a list whose first element is \code{x},
which is followed by (the elements of) list \code{xs}.  Hence, the
list values above could also have been defined as follows (in fact
their previous definition is simply syntactic sugar for the definitions below).
\begin{lstlisting}
val fruit  = "apples" :: ("oranges" :: ("pears" :: Nil))
val nums   = 1 :: (2 :: (3 :: (4 :: Nil)))
val diag3  = (1 :: (0 :: (0 :: Nil))) ::
             (0 :: (1 :: (0 :: Nil))) ::
             (0 :: (0 :: (1 :: Nil))) :: Nil
val empty  = Nil
\end{lstlisting}
The `\code{::}' operation associates to the right: \code{A :: B :: C} is
interpreted as \code{A :: (B :: C)}.  Therefore, we can drop the
parentheses in the definitions above. For instance, we can write
shorter
\begin{lstlisting}
val nums  =  1 :: 2 :: 3 :: 4 :: Nil
\end{lstlisting}

\paragraph{Basic operations on lists}
All operations on lists can be expressed in terms of the following three:

\begin{tabular}{ll}
\code{head}  &  returns the first element of a list,\\
\code{tail}  &  returns the list consisting of all elements except the\\
& first element,\\
\code{isEmpty} & returns \code{true} iff the list is empty
\end{tabular}

These operations are defined as methods of list objects. So we invoke
them by selecting from the list that's operated on. Examples:
\begin{lstlisting}
empty.isEmpty   = true
fruit.isEmpty   = false
fruit.head      = "apples"
fruit.tail.head = "oranges"
diag3.head      = List(1, 0, 0)
\end{lstlisting}
The \code{head} and \code{tail} methods are defined only for non-empty
lists.  When selected from an empty list, they throw an exception.

As an example of how lists can be processed, consider sorting the
elements of a list of numbers into ascending order. One simple way to
do so is {\em insertion sort}, which works as follows: To sort a
non-empty list with first element \code{x} and rest \code{xs}, sort
the remainder \code{xs} and insert the element \code{x} at the right
position in the result. Sorting an empty list will yield the
empty list. Expressed as Scala code:
\begin{lstlisting}
def isort(xs: List[Int]): List[Int] =
  if (xs.isEmpty) Nil
  else insert(xs.head, isort(xs.tail))
\end{lstlisting}

\begin{exercise} Provide an implementation of the missing function
\code{insert}.
\end{exercise}

\paragraph{List patterns} In fact, \code{::} is defined as a case
class in Scala's standard library. Hence, it is possible to decompose
lists by pattern matching, using patterns composed from the \code{Nil}
and \code{::} constructors. For instance, \code{isort} can be written
alternatively as follows.
\begin{lstlisting}
def isort(xs: List[Int]): List[Int] = xs match {
  case List() => List()
  case x :: xs1 => insert(x, isort(xs1))
}
\end{lstlisting}
where
\begin{lstlisting}
def insert(x: Int, xs: List[Int]): List[Int] = xs match {
  case List() => List(x)
  case y :: ys => if (x <= y) x :: xs else y :: insert(x, ys)
}
\end{lstlisting}

\section{Definition of class List I: First Order Methods}
\label{sec:list-first-order}

Lists are not built in in Scala; they are defined by an abstract class
\code{List}, which comes with two subclasses for \code{::} and \code{Nil}.
In the following we present a tour through class \code{List}.
\begin{lstlisting}
package scala
abstract class List[+A] {
\end{lstlisting}
\code{List} is an abstract class, so one cannot define elements by
calling the empty \code{List} constructor (e.g. by
\code{new List}).  The class has a type parameter \code{a}. It is
co-variant in this parameter, which means that
\code{List[S] <: List[T]} for all types \code{S} and \code{T} such that
\code{S <: T}.  The class is situated in the package
\code{scala}. This is a package containing the most important standard
classes of Scala.
 \code{List} defines a number of methods, which are
explained in the following.

\paragraph{Decomposing lists}
First, there are the three basic methods \code{isEmpty},
\code{head}, \code{tail}. Their implementation in terms of pattern
matching is straightforward:
\begin{lstlisting}
def isEmpty: Boolean = this match {
  case Nil => true
  case x :: xs => false
}
def head: A = this match {
  case Nil => error("Nil.head")
  case x :: xs => x
}
def tail: List[A] = this match {
  case Nil => error("Nil.tail")
  case x :: xs => xs
}
\end{lstlisting}

The next function computes the length of a list.
\begin{lstlisting}
def length: Int = this match {
  case Nil => 0
  case x :: xs => 1 + xs.length
}
\end{lstlisting}
\begin{exercise} Design a tail-recursive version of \code{length}.
\end{exercise}

The next two functions are the complements of \code{head} and
\code{tail}.
\begin{lstlisting}
def last: A
def init: List[A]
\end{lstlisting}
\code{xs.last} returns the last element of list \code{xs}, whereas
\code{xs.init} returns all elements of \code{xs} except the last.
Both functions have to traverse the entire list, and are thus less
efficient than their \code{head} and \code{tail} analogues.
Here is the implementation of \code{last}.
\begin{lstlisting}
def last: A = this match {
  case Nil      => error("Nil.last")
  case x :: Nil => x
  case x :: xs  => xs.last
}
\end{lstlisting}
The implementation of \code{init} is analogous.

The next three functions return a prefix of the list, or a suffix, or
both.
\begin{lstlisting}
def take(n: Int): List[A] =
  if (n == 0 || isEmpty) Nil else head :: tail.take(n-1)

def drop(n: Int): List[A] =
  if (n == 0 || isEmpty) this else tail.drop(n-1)

def split(n: Int): (List[A], List[A]) = (take(n), drop(n))
\end{lstlisting}
\code{(xs take n)} returns the first \code{n} elements of list
\code{xs}, or the whole list, if its length is smaller than \code{n}.
\code{(xs drop n)} returns all elements of \code{xs} except the
\code{n} first ones. Finally, \code{(xs split n)} returns a pair
consisting of the lists resulting from \code{xs take n} and
\code{xs drop n}.

The next function returns an element at a given index in a list.
It is thus analogous to array subscripting. Indices start at 0.
\begin{lstlisting}
def apply(n: Int): A = drop(n).head
\end{lstlisting}
The \code{apply} method has a special meaning in Scala. An object with
an \code{apply} method can be applied to arguments as if it was a
function. For instance, to pick the 3'rd element of a list \code{xs},
one can write either \code{xs.apply(3)} or \code{xs(3)} -- the latter
expression expands into the first.

With \code{take} and \code{drop}, we can extract sublists consisting
of consecutive elements of the original list.  To extract the sublist
$xs_m \commadots xs_{n-1}$ of a list \code{xs}, use:

\begin{lstlisting}
xs.drop(m).take(n - m)
\end{lstlisting}

\paragraph{Zipping lists} The next function combines two lists into a list of pairs.
Given two lists
\begin{lstlisting}
xs = List(x$_1$, ..., x$_n$)   $\mbox{\rm, and}$
ys = List(y$_1$, ..., y$_n$)   ,
\end{lstlisting}
\code{xs zip ys} constructs the list
\lstinline@List((x$_1$, y$_1$), ..., (x$_n$, y$_n$))@.
If the two lists have different lengths, the longer one of the two is
truncated. Here is the definition of \code{zip} -- note that it is a
polymorphic method.
\begin{lstlisting}
def zip[B](that: List[B]): List[(a,b)] =
  if (this.isEmpty || that.isEmpty) Nil
  else (this.head, that.head) :: (this.tail zip that.tail)
\end{lstlisting}

\paragraph{Consing lists.}
Like any infix operator, \code{::}
is also implemented as a method of an object. In this case, the object
is the list that is extended. This is possible, because operators
ending with a `\code{:}' character are treated specially in Scala.  
All such operators are treated as methods of their right operand. E.g.,
\begin{lstlisting}
    x :: y = y.::(x)       $\mbox{\rm whereas}$       x + y = x.+(y)                  
\end{lstlisting}
Note, however, that operands of a binary operation are in each case
evaluated from left to right.  So, if \code{D} and \code{E} are
expressions with possible side-effects, \code{D :: E} is translated to
\lstinline@{val x = D; E.::(x)}@ in order to maintain the left-to-right
order of operand evaluation.

Another difference between operators ending in a `\code{:}' and other
operators concerns their associativity.  Operators ending in
`\code{:}' are right-associative, whereas other operators are
left-associative.  E.g.,
\begin{lstlisting}
    x :: y :: z = x :: (y :: z)   $\mbox{\rm whereas}$    x + y + z = (x + y) + z
\end{lstlisting}
The definition of \code{::} as a method in
class \code{List} is as follows:
\begin{lstlisting}
def ::[B >: A](x: B): List[B] = new scala.::(x, this)
\end{lstlisting}
Note that \code{::} is defined for all elements \code{x} of type
\code{B} and lists of type \code{List[A]} such that the type \code{B}
of \code{x} is a supertype of the list's element type \code{A}. The result
is in this case a list of \code{B}'s. This
is expressed by the type parameter \code{B} with lower bound \code{A}
in the signature of \code{::}. 

\paragraph{Concatenating lists}
An operation similar to \code{::} is list concatenation, written
`\code{:::}'. The result of \code{(xs ::: ys)} is a list consisting of
all elements of \code{xs}, followed by all elements of \code{ys}.
Because it ends in a colon, \code{:::} is right-associative and is
considered as a method of its right-hand operand. Therefore,
\begin{lstlisting}
xs ::: ys ::: zs  =   xs ::: (ys ::: zs)
                  =   zs.:::(ys).:::(xs)
\end{lstlisting}
Here is the implementation of the \code{:::} method:
\begin{lstlisting}
  def :::[B >: A](prefix: List[B]): List[B] = prefix match {
    case Nil => this
    case p :: ps => this.:::(ps).::(p)
  }
\end{lstlisting}

\paragraph{Reversing lists} Another useful operation
is list reversal. There is a method \code{reverse} in \code{List} to
that effect. Let's try to give its implementation:
\begin{lstlisting}
def reverse[A](xs: List[A]): List[A] = xs match {
  case Nil => Nil
  case x :: xs => reverse(xs) ::: List(x)
}
\end{lstlisting}
This implementation has the advantage of being simple, but it is not
very efficient.  Indeed, one concatenation is executed for every
element in the list. List concatenation takes time proportional to the
length of its first operand. Therefore, the complexity of
\code{reverse(xs)} is
\[
n + (n - 1) + ... + 1 = n(n+1)/2
\]
where $n$ is the length of \code{xs}. Can \code{reverse} be
implemented more efficiently? We will see later that there exists
another implementation which has only linear complexity.

\section{Example: Merge sort}

The insertion sort presented earlier in this chapter is simple to
formulate, but also not very efficient. It's average complexity is
proportional to the square of the length of the input list. We now
design a program to sort the elements of a list which is more
efficient than insertion sort. A good algorithm for this is {\em merge
sort}, which works as follows.

First, if the list has zero or one elements, it is already sorted, so
one returns the list unchanged. Longer lists are split into two
sub-lists, each containing about half the elements of the original
list. Each sub-list is sorted by a recursive call to the sort
function, and the resulting two sorted lists are then combined in a
merge operation.

For a general implementation of merge sort, we still have to specify
the type of list elements to be sorted, as well as the function to be
used for the comparison of elements. We obtain a function of maximal
generality by passing these two items as parameters. This leads to the
following implementation.
\begin{lstlisting}
def msort[A](less: (A, A) => Boolean)(xs: List[A]): List[A] = {
  def merge(xs1: List[A], xs2: List[A]): List[A] =
    if (xs1.isEmpty) xs2
    else if (xs2.isEmpty) xs1
    else if (less(xs1.head, xs2.head)) xs1.head :: merge(xs1.tail, xs2)
    else xs2.head :: merge(xs1, xs2.tail)
  val n = xs.length/2
  if (n == 0) xs
  else merge(msort(less)(xs take n), msort(less)(xs drop n))
}
\end{lstlisting}
The complexity of \code{msort} is $O(N;log(N))$, where $N$ is the
length of the input list. To see why, note that splitting a list in
two and merging two sorted lists each take time proportional to the
length of the argument list(s). Each recursive call of \code{msort}
halves the number of elements in its input, so there are $O(log(N))$
consecutive recursive calls until the base case of lists of length 1
is reached.  However, for longer lists each call spawns off two
further calls. Adding everything up we obtain that at each of the
$O(log(N))$ call levels, every element of the original lists takes
part in one split operation and in one merge operation. Hence, every
call level has a total cost proportional to $O(N)$. Since there are
$O(log(N))$ call levels, we obtain an overall cost of
$O(N;log(N))$. That cost does not depend on the initial distribution
of elements in the list, so the worst case cost is the same as the
average case cost. This makes merge sort an attractive algorithm for
sorting lists.

Here is an example how \code{msort} is used.
\begin{lstlisting}
msort((x: Int, y: Int) => x < y)(List(5, 7, 1, 3))
\end{lstlisting}
The definition of \code{msort} is curried, to make it easy to specialize it with particular
comparison functions. For instance,
\begin{lstlisting}

val intSort = msort((x: Int, y: Int) => x < y)
val reverseSort = msort((x: Int, y: Int) => x > y)
\end{lstlisting}

\section{Definition of class List II: Higher-Order Methods}

The examples encountered so far show that functions over lists often
have similar structures. We can identify several patterns of
computation over lists, like:
\begin{itemize}
      \item transforming every element of a list in some way.
      \item extracting from a list all elements satisfying a criterion.
      \item combine the elements of a list using some operator.
\end{itemize}
Functional programming languages enable programmers to write general
functions which implement patterns like this by means of higher order
functions. We now discuss a set of commonly used higher-order
functions, which are implemented as methods in class \code{List}.

\paragraph{Mapping over lists}
A common operation is to transform each element of a list and then
return the lists of results.  For instance, to scale each element of a
list by a given factor.
\begin{lstlisting}
def scaleList(xs: List[Double], factor: Double): List[Double] = xs match {
  case Nil => xs
  case x :: xs1 => x * factor :: scaleList(xs1, factor)
}
\end{lstlisting}
This pattern can be generalized to the \code{map} method of class \code{List}:
\begin{lstlisting}
abstract class List[A] { ...
  def map[B](f: A => B): List[B] = this match {
    case Nil => this
    case x :: xs => f(x) :: xs.map(f)
  }
\end{lstlisting}
Using \code{map}, \code{scaleList} can be more concisely written as follows.
\begin{lstlisting}
def scaleList(xs: List[Double], factor: Double) =
  xs map (x => x * factor)
\end{lstlisting}

As another example, consider the problem of returning a given column
of a matrix which is represented as a list of rows, where each row is
again a list. This is done by the following function \code{column}.

\begin{lstlisting}
def column[A](xs: List[List[A]], index: Int): List[A] =
  xs map (row => row(index))
\end{lstlisting}

Closely related to \code{map} is the \code{foreach} method, which
applies a given function to all elements of a list, but does not
construct a list of results. The function is thus applied only for its
side effect. \code{foreach} is defined as follows.
\begin{lstlisting}
  def foreach(f: A => Unit) {
    this match {
      case Nil => ()
      case x :: xs => f(x); xs.foreach(f)
    }
  }
\end{lstlisting}
This function can be used for printing all elements of a list, for instance:
\begin{lstlisting}
  xs foreach (x => println(x))
\end{lstlisting} 

\begin{exercise} Consider a function which squares all elements of a list and
returns a list with the results. Complete the following two equivalent
definitions of \code{squareList}.

\begin{lstlisting}
def squareList(xs: List[Int]): List[Int] = xs match {
  case List() => ??
  case y :: ys => ??
}
def squareList(xs: List[Int]): List[Int] =
  xs map ??
\end{lstlisting}
\end{exercise}

\paragraph{Filtering Lists}
Another common operation selects from a list all elements fulfilling a
given criterion. For instance, to return a list of all positive
elements in some given lists of integers:
\begin{lstlisting}
def posElems(xs: List[Int]): List[Int] = xs match {
  case Nil => xs
  case x :: xs1 => if (x > 0) x :: posElems(xs1) else posElems(xs1)
}
\end{lstlisting}
This pattern is generalized to the \code{filter} method of class \code{List}:
\begin{lstlisting}
  def filter(p: A => Boolean): List[A] = this match {
    case Nil => this
    case x :: xs => if (p(x)) x :: xs.filter(p) else xs.filter(p)
  }
\end{lstlisting}
Using \code{filter}, \code{posElems} can be more concisely written as
follows.
\begin{lstlisting}
def posElems(xs: List[Int]): List[Int] =
  xs filter (x => x > 0)
\end{lstlisting}

An operation related to filtering is testing whether all elements of a
list satisfy a certain condition. Dually, one might also be interested
in the question whether there exists an element in a list that
satisfies a certain condition. These operations are embodied in the
higher-order functions \code{forall} and \code{exists} of class
\code{List}.
\begin{lstlisting}
def forall(p: A => Boolean): Boolean =
  isEmpty || (p(head) && (tail forall p))
def exists(p: A => Boolean): Boolean =
  !isEmpty && (p(head) || (tail exists p))
\end{lstlisting}
To illustrate the use of \code{forall}, consider the question of whether
a number if prime. Remember that a number $n$ is prime of it can be
divided without remainder only by one and itself. The most direct
translation of this definition would test that $n$ divided by all
numbers from 2 up to and excluding itself gives a non-zero
remainder. This list of numbers can be generated using a function
\code{List.range} which is defined in object \code{List} as follows.
\begin{lstlisting}
package scala
object List { ...
  def range(from: Int, end: Int): List[Int] =
    if (from >= end) Nil else from :: range(from + 1, end)
\end{lstlisting}
For example, \code{List.range(2, n)}
generates the list of all integers from 2 up to and excluding $n$.
The function \code{isPrime} can now simply be defined as follows.
\begin{lstlisting}
def isPrime(n: Int) =
  List.range(2, n) forall (x => n % x != 0)
\end{lstlisting}
We see that the mathematical definition of prime-ness has been
translated directly into Scala code. 

Exercise: Define \code{forall} and \code{exists} in terms of \code{filter}.


\paragraph{Folding and Reducing Lists}
Another common operation is to combine the elements of a list with
some operator.  For instance:
\begin{lstlisting}
sum(List(x$_1$, ..., x$_n$))       =  0 + x$_1$ + ... + x$_n$
product(List(x$_1$, ..., x$_n$))   =  1 * x$_1$ * ... * x$_n$
\end{lstlisting}
Of course, we can implement both functions with a
recursive scheme:
\begin{lstlisting}
def sum(xs: List[Int]): Int = xs match {
  case Nil => 0
  case y :: ys => y + sum(ys)
}
def product(xs: List[Int]): Int = xs match {
  case Nil => 1
  case y :: ys => y * product(ys)
}
\end{lstlisting}
But we can also use the generalization of this program scheme embodied
in the \code{reduceLeft} method of class \code{List}.  This method
inserts a given binary operator between adjacent elements of a given list.
E.g.\ 
\begin{lstlisting}
List(x$_1$, ..., x$_n$).reduceLeft(op) = (...(x$_1$ op x$_2$) op ... ) op x$_n$
\end{lstlisting}
Using \code{reduceLeft}, we can make the common pattern
in \code{sum} and \code{product} apparent:
\begin{lstlisting}
def sum(xs: List[Int])      =  (0 :: xs) reduceLeft {(x, y) => x + y}
def product(xs: List[Int])  =  (1 :: xs) reduceLeft {(x, y) => x * y}
\end{lstlisting}
Here is the implementation of \code{reduceLeft}.
\begin{lstlisting}
  def reduceLeft(op: (A, A) => A): A = this match {
    case Nil     => error("Nil.reduceLeft")
    case x :: xs => (xs foldLeft x)(op)
  }
  def foldLeft[B](z: B)(op: (B, A) => B): B = this match {
    case Nil => z
    case x :: xs => (xs foldLeft op(z, x))(op)
  }
}
\end{lstlisting}
We see that the \code{reduceLeft} method is defined in terms of
another generally useful method, \code{foldLeft}.  The latter takes as
additional parameter an {\em accumulator} \code{z}, which is returned
when \code{foldLeft} is applied on an empty list. That is,
\begin{lstlisting}
(List(x$_1$, ..., x$_n$) foldLeft z)(op)   =  (...(z op x$_1$) op ... ) op x$_n$
\end{lstlisting}
The \code{sum} and \code{product} methods can be defined alternatively
using \code{foldLeft}:
\begin{lstlisting}
def sum(xs: List[Int])      =  (xs foldLeft 0) {(x, y) => x + y}
def product(xs: List[Int])  =  (xs foldLeft 1) {(x, y) => x * y}
\end{lstlisting}

\paragraph{FoldRight and ReduceRight}
Applications of \code{foldLeft} and \code{reduceLeft} expand to
left-leaning trees. \todo{insert pictures}.  They have duals
\code{foldRight} and \code{reduceRight}, which produce right-leaning
trees.
\begin{lstlisting}
List(x$_1$, ..., x$_n$).reduceRight(op)     =  x$_1$ op ( ... (x$_{n-1}$ op x$_n$)...)
(List(x$_1$, ..., x$_n$) foldRight acc)(op) =  x$_1$ op ( ... (x$_n$ op acc)...)
\end{lstlisting}
These are defined as follows.
\begin{lstlisting}
  def reduceRight(op: (A, A) => A): A = this match {
    case Nil => error("Nil.reduceRight")
    case x :: Nil => x
    case x :: xs => op(x, xs.reduceRight(op))
  }
  def foldRight[B](z: B)(op: (A, B) => B): B = this match {
    case Nil => z
    case x :: xs => op(x, (xs foldRight z)(op))
  }
\end{lstlisting}

Class \code{List} defines also two symbolic abbreviations for
\code{foldLeft} and \code{foldRight}:
\begin{lstlisting}
  def /:[B](z: B)(f: (B, A) => B): B = foldLeft(z)(f)
  def :\[B](z: B)(f: (A, B) => B): B = foldRight(z)(f)
\end{lstlisting}
The method names picture the left/right leaning trees of the fold
operations by forward or backward slashes. The \code{:} points in each
case to the list argument whereas the end of the slash points to the
accumulator (or: zero) argument \code{z}. 
That is, 
\begin{lstlisting}
(z /: List(x$_1$, ..., x$_n$))(op) = (...(z op x$_1$) op ... ) op x$_n$ 
(List(x$_1$, ..., x$_n$) :\ z)(op) = x$_1$ op ( ... (x$_n$ op acc)...)
\end{lstlisting}
For associative and commutative operators, \code{/:} and
\code{:\\} are equivalent (even though there may be a difference
in efficiency).  
%But sometimes, only one of the two operators is
%appropriate or has the right type:

\begin{exercise} Consider the problem of writing a function \code{flatten},
which takes a list of element lists as arguments. The result of
\code{flatten} should be the concatenation of all element lists into a
single list. Here is the an implementation of this method in terms of 
\code{:\\}.
\begin{lstlisting}
def flatten[A](xs: List[List[A]]): List[A] =
  (xs :\ (Nil: List[A])) {(x, xs) => x ::: xs}
\end{lstlisting} 
Consider replacing the body of \lstinline@flatten@
by 
\begin{lstlisting}
  ((Nil: List[A]) /: xs) ((xs, x) => xs ::: x)
\end{lstlisting}
What would be the difference in asymptotic
complexity between the two versions of \lstinline@flatten@?

In fact \code{flatten} is predefined together with a set of other
userful function in an object called \code{List} in the standatd Scala
library. It can be accessed from user program by calling
\code{List.flatten}. Note that \code{flatten} is not a method of class
\code{List} -- it would not make sense there, since it applies only
to lists of lists, not to all lists in general.
\end{exercise}

\paragraph{List Reversal Again} We have seen in 
Section~\ref{sec:list-first-order} an implementation of method
\code{reverse} whose run-time was quadratic in the length of the list
to be reversed. We now develop a new implementation of \code{reverse},
which has linear cost.  The idea is to use a \code{foldLeft}
operation based on the following program scheme.
\begin{lstlisting}
class List[+A] { ...
  def reverse: List[A] = (z? /: this)(op?)
\end{lstlisting}
It only remains to fill in the \code{z?} and \code{op?} parts.  Let's
try to deduce them from examples.
\begin{lstlisting}
  Nil
= Nil.reverse                 // by specification
= (z /: Nil)(op)              // by the template for reverse
= (Nil foldLeft z)(op)        // by the definition of /:
= z                           // by definition of foldLeft
\end{lstlisting}
Hence, \code{z?} must be \code{Nil}. To deduce the second operand,
let's study reversal of a list of length one.
\begin{lstlisting}
  List(x)
= List(x).reverse             // by specification
= (Nil /: List(x))(op)        // by the template for reverse, with z = Nil
= (List(x) foldLeft Nil)(op)  // by the definition of /:
= op(Nil, x)                  // by definition of foldLeft
\end{lstlisting}
Hence, \code{op(Nil, x)} equals \code{List(x)}, which is the same
as \code{x :: Nil}. This suggests to take as \code{op} the
\code{::} operator with its operands exchanged.  Hence, we arrive at
the following implementation for \code{reverse}, which has linear complexity.
\begin{lstlisting}
def reverse: List[A] =
  ((Nil: List[A]) /: this) {(xs, x) => x :: xs}
\end{lstlisting}
(Remark: The type annotation of \code{Nil} is necessary 
to make the type inferencer work.)

\begin{exercise} Fill in the missing expressions to complete the following
definitions of some basic list-manipulation operations as fold
operations.
\begin{lstlisting}
def mapFun[A, B](xs: List[A], f: A => B): List[B] =
  (xs :\ List[B]()){ ?? }

def lengthFun[A](xs: List[A]): int =
  (0 /: xs){ ?? }
\end{lstlisting}
\end{exercise}

\paragraph{Nested Mappings}

We can employ higher-order list processing functions to express many
computations that are normally expressed as nested loops in imperative
languages. 

As an example, consider the following problem: Given a positive
integer $n$, find all pairs of positive integers $i$ and $j$, where 
$1 \leq j < i < n$ such that $i + j$ is prime. For instance, if $n = 7$,
the pairs are
\bda{c|lllllll}
i     & 2 & 3 & 4 & 4 & 5 & 6 & 6\\
j     & 1 & 2 & 1 & 3 & 2 & 1 & 5\\ \hline
i + j & 3 & 5 & 5 & 7 & 7 & 7 & 11
\eda

A natural way to solve this problem consists of two steps. In a first step,
one generates the sequence of all pairs $(i, j)$ of integers such that
$1 \leq j < i < n$. In a second step one then filters from this sequence
all pairs $(i, j)$ such that $i + j$ is prime.

Looking at the first step in more detail, a natural way to generate
the sequence of pairs consists of three sub-steps.  First, generate
all integers between $1$ and $n$ for $i$.  
\item
Second, for each integer $i$ between $1$ and $n$, generate the list of
pairs $(i, 1)$ up to $(i, i-1)$. This can be achieved by a
combination of \code{range} and \code{map}:
\begin{lstlisting}
  List.range(1, i) map (x => (i, x))
\end{lstlisting}
Finally, combine all sublists using \code{foldRight} with \code{:::}.
Putting everything together gives the following expression:
\begin{lstlisting}
List.range(1, n)
  .map(i => List.range(1, i).map(x => (i, x)))
  .foldRight(List[(Int, Int)]()) {(xs, ys) => xs ::: ys}
  .filter(pair => isPrime(pair._1 + pair._2))
\end{lstlisting}

\paragraph{Flattening Maps}
The combination of mapping and then concatenating sublists 
resulting from the map
is so common that we there is a special method 
for it in class \code{List}:
\begin{lstlisting}
abstract class List[+A] { ...
  def flatMap[B](f: A => List[B]): List[B] = this match {
    case Nil => Nil
    case x :: xs => f(x) ::: (xs flatMap f)
  }
}
\end{lstlisting}
With \code{flatMap}, the pairs-whose-sum-is-prime expression 
could have been written more concisely as follows.
\begin{lstlisting}
List.range(1, n)
  .flatMap(i => List.range(1, i).map(x => (i, x)))
  .filter(pair => isPrime(pair._1 + pair._2))
\end{lstlisting}



\section{Summary}

This chapter has introduced lists as a fundamental data structure in
programming. Since lists are immutable, they are a common data type in
functional programming languages. They have a role comparable to
arrays in imperative languages. However, the access patterns between
arrays and lists are quite different. Where array accessing is always
done by indexing, this is much less common for lists.  We have seen
that \code{scala.List} defines a method called \code{apply} for indexing
however this operation is much more costly than in the case of arrays
(linear as opposed to constant time). Instead of indexing, lists are
usually traversed recursively, where recursion steps are usually based
on a pattern match over the traversed list. There is also a rich set of
higher-order combinators which allow one to instantiate a set of
predefined patterns of computations over lists.

\comment{
\bsh{Reasoning About Lists}

Recall the concatenation operation for lists:

\begin{lstlisting}
class List[+A] {
  ...
  def ::: (that: List[A]): List[A] =
    if (isEmpty) that
    else head :: (tail ::: that)
}
\end{lstlisting}

We would like to verify that concatenation is associative, with the
empty list \code{List()} as left and right identity:
\bda{lcl}
   (xs ::: ys) ::: zs &=& xs ::: (ys ::: zs) \\
   xs ::: List()          &=& xs \gap =\ List() ::: xs
\eda
\emph{Q}: How can we prove statements like the one above?

\emph{A}: By \emph{structural induction} over lists.
\es
\bsh{Reminder: Natural Induction}

Recall the proof principle of \emph{natural induction}:

To show a property \mathtext{P(n)} for all numbers \mathtext{n \geq b}:
\be
\item Show that \mathtext{P(b)} holds (\emph{base case}).
\item For arbitrary \mathtext{n \geq b} show:
\begin{quote}
     if \mathtext{P(n)} holds, then \mathtext{P(n+1)} holds as well
\end{quote}
(\emph{induction step}).
\ee
%\es\bs
\emph{Example}: Given
\begin{lstlisting}
def factorial(n: Int): Int =
  if (n == 0) 1
  else n * factorial(n-1)
\end{lstlisting}
show that, for all \code{n >= 4},
\begin{lstlisting}
   factorial(n) >= 2$^n$
\end{lstlisting}
\es\bs
\Case{\code{4}}
is established by simple calculation of \code{factorial(4) = 24} and \code{2$^4$ = 16}.

\Case{\code{n+1}} 
We have for \code{n >= 4}:
\begin{lstlisting}
    \= factorial(n + 1)
 =     \> $\expl{by the second clause of factorial(*)}$
       \> (n + 1) * factorial(n)
 >=    \> $\expl{by calculation}$
       \> 2 * factorial(n)
 >=    \> $\expl{by the induction hypothesis}$
       \> 2 * 2$^n$.
\end{lstlisting}
Note that in our proof we can freely apply reduction steps such as in (*)
anywhere in a term.


This works because purely functional programs do not have side
effects; so a term is equivalent to the term it reduces to.

The principle is called {\em\emph{referential transparency}}.
\es
\bsh{Structural Induction}

The principle of structural induction is analogous to natural induction:

In the case of lists, it is as follows:

To prove a property \mathtext{P(xs)} for all lists \mathtext{xs},
\be
\item Show that \code{P(List())} holds (\emph{base case}).
\item For arbitrary lists \mathtext{xs} and elements \mathtext{x} 
      show:
\begin{quote}
     if \mathtext{P(xs)} holds, then \mathtext{P(x :: xs)} holds as well
\end{quote}
(\emph{induction step}).
\ee

\es
\bsh{Example}

We show \code{(xs ::: ys) ::: zs  =  xs ::: (ys ::: zs)} by structural induction
on \code{xs}.

\Case{\code{List()}}
For the left-hand side, we have:
\begin{lstlisting}
    \= (List() ::: ys) ::: zs
 =     \> $\expl{by first clause of \prog{:::}}$
       \> ys ::: zs
\end{lstlisting}
For the right-hand side, we have:
\begin{lstlisting}
    \= List() ::: (ys ::: zs)
 =     \> $\expl{by first clause of \prog{:::}}$
       \> ys ::: zs
\end{lstlisting}
So the case is established.

\es
\bs
\Case{\code{x :: xs}} 

For the left-hand side, we have:
\begin{lstlisting}
    \= ((x :: xs) ::: ys) ::: zs
 =     \> $\expl{by second clause of \prog{:::}}$
       \> (x :: (xs ::: ys)) ::: zs
 =     \> $\expl{by second clause of \prog{:::}}$
       \> x :: ((xs ::: ys) ::: zs)
 =     \> $\expl{by the induction hypothesis}$
       \> x :: (xs ::: (ys ::: zs))
\end{lstlisting}

For the right-hand side, we have:
\begin{lstlisting}
    \= (x :: xs) ::: (ys ::: zs)
 =     \> $\expl{by second clause of \prog{:::}}$
       \> x :: (xs ::: (ys ::: zs))
\end{lstlisting}
So the case (and with it the property) is established.

\begin{exercise}
Show by induction on \code{xs} that \code{xs ::: List()  =  xs}.
\es
\bsh{Example (2)}
\end{exercise}

As a more difficult example, consider function
\begin{lstlisting}
abstract class List[A] { ...
  def reverse: List[A] = this match {
    case List() => List()
    case x :: xs => xs.reverse ::: List(x)
  }
}
\end{lstlisting}
We would like to prove the proposition that
\begin{lstlisting}
   xs.reverse.reverse  =  xs  .
\end{lstlisting}
We proceed by induction over \code{xs}. The base case is easy to establish:
\begin{lstlisting}
    \= List().reverse.reverse
 =     \> $\expl{by first clause of \prog{reverse}}$
       \> List().reverse
 =     \> $\expl{by first clause of \prog{reverse}}$
       \> List()
\end{lstlisting}
\es\bs
For the induction step, we try:
\begin{lstlisting}
    \= (x :: xs).reverse.reverse
 =     \> $\expl{by second clause of \prog{reverse}}$
       \> (xs.reverse ::: List(x)).reverse
\end{lstlisting}
There's nothing more we can do to this expression, so we turn to the right side:
\begin{lstlisting}
    \= x :: xs
 =     \> $\expl{by induction hypothesis}$
       \> x :: xs.reverse.reverse
\end{lstlisting}
The two sides have simplified to different expressions.

So we still have to show that
\begin{lstlisting}
  (xs.reverse ::: List(x)).reverse  =  x :: xs.reverse.reverse
\end{lstlisting}
Trying to prove this directly by induction does not work.

Instead we have to {\em generalize} the equation to:
\begin{lstlisting}
  (ys ::: List(x)).reverse  =  x :: ys.reverse
\end{lstlisting}
\es\bs
This equation can be proved by a second induction argument over \code{ys}.
(See blackboard).

\begin{exercise}
Is it the case that \code{(xs drop m) at n  =  xs at (m + n)} for all 
natural numbers \code{m}, \code{n} and all lists \code{xs}?
\end{exercise}

\es
\bsh{Structural Induction on Trees}

Structural induction is not restricted to lists; it works for arbitrary
trees.

The general induction principle is as follows.

To show that property \code{P(t)} holds for all trees of a certain type,
\begin{itemize}
\item Show \code{P(l)} for all leaf trees \code{$l$}.
\item For every interior node \code{t} with subtrees \code{s$_1$, ..., s$_n$}, 
      show that \code{P(s$_1$) $\wedge$ ... $\wedge$ P(s$_n$) => P(t)}.
\end{itemize} 

\example Recall our definition of \code{IntSet} with 
operations \code{contains} and \code{incl}:

\begin{lstlisting}
abstract class IntSet {
  abstract def incl(x: Int): IntSet
  abstract def contains(x: Int): Boolean
}
\end{lstlisting}
\es\bs
\begin{lstlisting}
case class Empty extends IntSet {
  def contains(x: Int): Boolean = false
  def incl(x: Int): IntSet = NonEmpty(x, Empty, Empty)
}
case class NonEmpty(elem: Int, left: Set, right: Set) extends IntSet {
  def contains(x: Int): Boolean =
    if (x < elem) left contains x
    else if (x > elem) right contains x
    else true
  def incl(x: Int): IntSet =
    if (x < elem) NonEmpty(elem, left incl x, right)
    else if (x > elem) NonEmpty(elem, left, right incl x)
    else this
}
\end{lstlisting}
(With \code{case} added, so that we can use factory methods instead of \code{new}).

What does it mean to prove the correctness of this implementation?
\es
\bsh{Laws of IntSet}

One way to state and prove the correctness of an implementation is
to prove laws that hold for it.

In the case of \code{IntSet}, three such laws would be:

For all sets \code{s}, elements \code{x}, \code{y}:

\begin{lstlisting}
Empty contains x          \= =  false
(s incl x) contains x     \> =  true
(s incl x) contains y     \> =  s contains y         if x $\neq$ y
\end{lstlisting}

(In fact, one can show that these laws characterize the desired data
type completely).

How can we establish that these laws hold?

\emph{Proposition 1}: \code{Empty contains x =  false}.

\emph{Proof}: By the definition of \code{contains} in \code{Empty}.
\es\bs
\emph{Proposition 2}: \code{(xs incl x) contains x = true}

\emph{Proof:}

\Case{\code{Empty}}
\begin{lstlisting}
    \= (Empty incl x) contains x
 =     \> $\expl{by definition of \prog{incl} in \prog{Empty}}$
       \> NonEmpty(x, Empty, Empty) contains x
 =     \> $\expl{by definition of \prog{contains} in \prog{NonEmpty}}$
       \> true
\end{lstlisting}

\Case{\code{NonEmpty(x, l, r)}}
\begin{lstlisting}
    \= (NonEmpty(x, l, r) incl x) contains x
 =     \> $\expl{by definition of \prog{incl} in \prog{NonEmpty}}$
       \> NonEmpty(x, l, r) contains x
 =     \> $\expl{by definition of \prog{contains} in \prog{Empty}}$
       \> true
\end{lstlisting}
\es\bs
\Case{\code{NonEmpty(y, l, r)} where \code{y < x}}
\begin{lstlisting}
    \= (NonEmpty(y, l, r) incl x) contains x
 =     \> $\expl{by definition of \prog{incl} in \prog{NonEmpty}}$
       \> NonEmpty(y, l, r incl x) contains x
 =     \> $\expl{by definition of \prog{contains} in \prog{NonEmpty}}$
       \> (r incl x) contains x
 =     \> $\expl{by the induction hypothesis}$
       \> true
\end{lstlisting}

\Case{\code{NonEmpty(y, l, r)} where \code{y > x}} is analogous.

\bigskip

\emph{Proposition 3}: If \code{x $\neq$ y} then
\code{xs incl y contains x  =  xs contains x}.

\emph{Proof:} See blackboard.
\es
\bsh{Exercise}

Say we add a \code{union} function to \code{IntSet}:

\begin{lstlisting}
class IntSet { ...
  def union(other: IntSet): IntSet
}
class Expty extends IntSet { ...
  def union(other: IntSet) = other
}
class NonEmpty(x: Int, l: IntSet, r: IntSet) extends IntSet { ...
  def union(other: IntSet): IntSet = l union r union other incl x
}
\end{lstlisting}

The correctness of \code{union} can be subsumed with the following
law:

\emph{Proposition 4}: 
\code{(xs union ys) contains x  =  xs contains x || ys contains x}.
Is that true ? What hypothesis is missing ? Show a counterexample.

Show Proposition 4 using structural induction on \code{xs}.
\es
\comment{

\emph{Proof:} By induction on \code{xs}.

\Case{\code{Empty}}

\Case{\code{NonEmpty(x, l, r)}}

\Case{\code{NonEmpty(y, l, r)} where \code{y < x}}

\begin{lstlisting}
    \= (Empty union ys) contains x 
 =      \> $\expl{by definition of \prog{union} in \prog{Empty}}$
        \> ys contains x
 =      \> $\expl{Boolean algebra}$
        \> false || ys contains x
 =      \> $\expl{by definition of \prog{contains} in \prog{Empty} (reverse)}$
        \> (Empty contains x) || (ys contains x)
\end{lstlisting}

\begin{lstlisting}
    \= (NonEmpty(x, l, r) union ys) contains x
 =      \> $\expl{by definition of \prog{union} in \prog{NonEmpty}}$
        \> (l union r union ys incl x) contains x
 =      \> $\expl{by Proposition 2}$
        \> true
 =      \> $\expl{Boolean algebra}$
        \> true || (ys contains x)
 =      \> $\expl{by definition of \prog{contains} in \prog{NonEmpty} (reverse)}$
        \> (NonEmpty(x, l, r) contains x) || (ys contains x)
\end{lstlisting}

\begin{lstlisting}
    \= (NonEmpty(y, l, r) union ys) contains x
 =      \> $\expl{by definition of \prog{union} in \prog{NonEmpty}}$
        \> (l union r union ys incl y) contains x
 =      \> $\expl{by Proposition 3}$
        \> (l union r union ys) contains x
 =      \> $\expl{by the induction hypothesis}$
        \> ((l union r) contains x) || (ys contains x)
 =      \> $\expl{by Proposition 3}$
        \> ((l union r incl y) contains x) || (ys contains x)
\end{lstlisting}

\Case{\code{NonEmpty(y, l, r)} where \code{y < x}}
 ... is analogous.

\es
}}
\chapter{\label{sec:for-notation}For-Comprehensions}

The last chapter demonstrated that higher-order functions such as
\verb@map@, \verb@flatMap@, \verb@filter@ provide powerful
constructions for dealing with lists.  But sometimes the level of
abstraction required by these functions makes a program hard to
understand.

To help understandability, Scala has a special notation which
simplifies common patterns of applications of higher-order functions.
This notation builds a bridge between set-comprehensions in
mathematics and for-loops in imperative languages such as C or
Java. It also closely resembles the query notation of relational
databases.

As a first example, say we are given a list \code{persons} of persons
with \code{name} and \code{age} fields.  To print the names of all
persons in the sequence which are aged over 20, one can write:
\begin{lstlisting}
for (p <- persons if p.age > 20) yield p.name
\end{lstlisting}
This is equivalent to the following expression , which uses
higher-order functions \code{filter} and \code{map}:
\begin{lstlisting}
persons filter (p => p.age > 20) map (p => p.name)
\end{lstlisting}
The for-comprehension looks a bit like a for-loop in imperative languages,
except that it constructs a list of the results of all iterations.

Generally, a for-comprehension is of the form
\begin{lstlisting}
for ( $s$ ) yield $e$
\end{lstlisting}
Here, $s$ is a sequence of {\em generators}, {\em definitions} and
{\em filters}.  A {\em generator} is of the form \code{val x <- e},
where \code{e} is a list-valued expression. It binds \code{x} to
successive values in the list.  A {\em definition} is of the form
\code{val x = e}. It introduces \code{x} as a name for the value of
\code{e} in the rest of the comprehension. A {\em filter} is an
expression \code{f} of type \code{Boolean}.  It omits from
consideration all bindings for which \code{f} is \code{false}.  The
sequence $s$ starts in each case with a generator.  If there are
several generators in a sequence, later generators vary more rapidly
than earlier ones.

The sequence $s$ may also be enclosed in braces instead of
parentheses, in which case the semicolons between generators,
definitions and filters can be omitted.

Here are two examples that show how for-comprehensions are used.
First, let's redo an example of the previous chapter: Given a positive
integer $n$, find all pairs of positive integers $i$ and $j$, where $1
\leq j < i < n$ such that $i + j$ is prime. With a for-comprehension
this problem is solved as follows:
\begin{lstlisting}
for { i <- List.range(1, n)
      j <- List.range(1, i)
      if isPrime(i+j) } yield {i, j}
\end{lstlisting}
This is arguably much clearer than the solution using \code{map},
\code{flatMap} and \code{filter} that we have developed previously.

As a second example, consider computing the scalar product of two
vectors \code{xs} and \code{ys}. Using a for-comprehension, this can
be written as follows.
\begin{lstlisting}
  sum(for ((x, y) <- xs zip ys) yield x * y)
\end{lstlisting}

\section{The N-Queens Problem}

For-comprehensions are especially useful for solving combinatorial
puzzles. An example of such a puzzle is the 8-queens problem: Given a
standard chess-board, place 8 queens such that no queen is in check from any
other (a queen can check another piece if they are on the same
column, row, or diagonal). We will now develop a solution to this
problem, generalizing it to chess-boards of arbitrary size. Hence, the
problem is to place $n$ queens on a chess-board of size $n \times n$.

To solve this problem, note that we need to place a queen in each row.
So we could place queens in successive rows, each time checking that a
newly placed queen is not in check from any other queens that have
already been placed. In the course of this search, it might arrive
that a queen to be placed in row $k$ would be in check in all fields
of that row from queens in row $1$ to $k-1$. In that case, we need to
abort that part of the search in order to continue with a different
configuration of queens in columns $1$ to $k-1$.

This suggests a recursive algorithm.  Assume that we have already
generated all solutions of placing $k-1$ queens on a board of size $n
\times n$. We can represent each such solution by a list of length
$k-1$ of column numbers (which can range from $1$ to $n$).  We treat
these partial solution lists as stacks, where the column number of the
queen in row $k-1$ comes first in the list, followed by the column
number of the queen in row $k-2$, etc. The bottom of the stack is the
column number of the queen placed in the first row of the board.  All
solutions together are then represented as a list of lists, with one
element for each solution.

Now, to place the $k$'the queen, we generate all possible extensions
of each previous solution by one more queen. This yields another list
of solution lists, this time of length $k$. We continue the process
until we have reached solutions of the size of the chess-board $n$.
This algorithmic idea is embodied in function \code{placeQueens} below:
\begin{lstlisting}
def queens(n: Int): List[List[Int]] = {
  def placeQueens(k: Int): List[List[Int]] =
    if (k == 0) List(List())
    else for { queens <- placeQueens(k - 1)
               column <- List.range(1, n + 1)
               if isSafe(column, queens, 1) } yield column :: queens
  placeQueens(n)
}
\end{lstlisting}

\begin{exercise} Write the function
\begin{lstlisting}
  def isSafe(col: Int, queens: List[Int], delta: Int): Boolean
\end{lstlisting}
which tests whether a queen in the given column \verb@col@ is safe with 
respect to the \verb@queens@ already placed. Here, \verb@delta@ is the difference between the row of the queen to be
placed and the row of the first queen in the list.
\end{exercise}

\section{Querying with For-Comprehensions}

The for-notation is essentially equivalent to common operations of
database query languages.  For instance, say we are given a 
database \code{books}, represented as a list of books, where
\code{Book} is defined as follows.
\begin{lstlisting}
case class Book(title: String, authors: List[String])
\end{lstlisting}
Here is a small example database:
\begin{lstlisting}
val books: List[Book] = List(
  Book("Structure and Interpretation of Computer Programs",
       List("Abelson, Harold", "Sussman, Gerald J.")),
  Book("Principles of Compiler Design",
       List("Aho, Alfred", "Ullman, Jeffrey")),
  Book("Programming in Modula-2",
       List("Wirth, Niklaus")),
  Book("Introduction to Functional Programming"),
       List("Bird, Richard")),
  Book("The Java Language Specification",
       List("Gosling, James", "Joy, Bill", "Steele, Guy", "Bracha, Gilad")))
\end{lstlisting}
Then, to find the titles of all books whose author's last name is ``Ullman'':
\begin{lstlisting}
for (b <- books; a <- b.authors if a startsWith "Ullman")
yield b.title
\end{lstlisting}
(Here, \code{startsWith} is a method in \code{java.lang.String}).  Or,
to find the titles of all books that have the string ``Program'' in
their title:
\begin{lstlisting}
for (b <- books if (b.title indexOf "Program") >= 0)
yield b.title
\end{lstlisting}
Or, to find the names of all authors that have written at least two
books in the database.
\begin{lstlisting}
for (b1 <- books; b2 <- books if b1 != b2;
     a1 <- b1.authors; a2 <- b2.authors if a1 == a2)
yield a1
\end{lstlisting}
The last solution is not yet perfect, because authors will appear
several times in the list of results.  We still need to remove
duplicate authors from result lists.  This can be achieved with the
following function.
\begin{lstlisting}
def removeDuplicates[A](xs: List[A]): List[A] =
  if (xs.isEmpty) xs
  else xs.head :: removeDuplicates(xs.tail filter (x => x != xs.head))
\end{lstlisting}
Note that the last expression in method \code{removeDuplicates}
can be equivalently expressed using a for-comprehension.
\begin{lstlisting}
xs.head :: removeDuplicates(for (x <- xs.tail if x != xs.head) yield x)
\end{lstlisting}

\section{Translation of For-Comprehensions}

Every for-comprehension can be expressed in terms of the three
higher-order functions \code{map}, \code{flatMap} and \code{filter}.
Here is the translation scheme, which is also used by the Scala compiler.
\begin{itemize}
\item
A simple for-comprehension
\begin{lstlisting}
for (x <- e) yield e'
\end{lstlisting}
is translated to
\begin{lstlisting}
e.map(x => e')
\end{lstlisting}
\item
A for-comprehension
\begin{lstlisting}
for (x <- e if f; s) yield e'
\end{lstlisting}
where \code{f} is a filter and \code{s} is a (possibly empty)
sequence of generators or filters
is translated to
\begin{lstlisting}
for (x <- e.filter(x => f); s) yield e'
\end{lstlisting}
and then translation continues with the latter expression.
\item
A for-comprehension
\begin{lstlisting}
for (x <- e; y <- e'; s) yield e''
\end{lstlisting}
where \code{s} is a (possibly empty)
sequence of generators or filters
is translated to
\begin{lstlisting}
e.flatMap(x => for (y <- e'; s) yield e'')
\end{lstlisting}
and then translation continues with the latter expression.
\end{itemize}
For instance, taking our "pairs of integers whose sum is prime" example:
\begin{lstlisting}
for { i <- range(1, n)
      j <- range(1, i)
      if isPrime(i+j)
} yield {i, j}
\end{lstlisting}
Here is what we get when we translate this expression:
\begin{lstlisting}
range(1, n)
  .flatMap(i =>
    range(1, i)
      .filter(j => isPrime(i+j))
      .map(j => (i, j)))
\end{lstlisting}

Conversely, it would also be possible to express functions \code{map},
\code{flatMap}{ and \code{filter} using for-comprehensions. Here are the
three functions again, this time implemented using for-comprehensions.
\begin{lstlisting}
object Demo {
  def map[A, B](xs: List[A], f: A => B): List[B] =
    for (x <- xs) yield f(x)

  def flatMap[A, B](xs: List[A], f: A => List[B]): List[B] =
    for (x <- xs; y <- f(x)) yield y

  def filter[A](xs: List[A], p: A => Boolean): List[A] =
    for (x <- xs if p(x)) yield x
}
\end{lstlisting}
Not surprisingly, the translation of the for-comprehension in the body of
\code{Demo.map} will produce a call to \code{map} in class \code{List}.
Similarly, \code{Demo.flatMap} and \code{Demo.filter} translate to
\code{flatMap} and \code{filter} in class \code{List}.

\begin{exercise}
Define the following function in terms of \code{for}.
\begin{lstlisting}
def flatten[A](xss: List[List[A]]): List[A] =
  (xss :\ (Nil: List[A])) ((xs, ys) => xs ::: ys)
\end{lstlisting}
\end{exercise}

\begin{exercise}
Translate
\begin{lstlisting}
for (b <- books; a <- b.authors if a startsWith "Bird") yield b.title
for (b <- books if (b.title indexOf "Program") >= 0) yield b.title
\end{lstlisting}
to higher-order functions.
\end{exercise}

\section{For-Loops}\label{sec:for-loops}

For-comprehensions resemble for-loops in imperative languages, except
that they produce a list of results. Sometimes, a list of results is
not needed but we would still like the flexibility of generators and
filters in iterations over lists. This is made possible by a variant
of the for-comprehension syntax, which expresses for-loops:
\begin{lstlisting}
for ( $s$ ) $e$
\end{lstlisting}
This construct is the same as the standard for-comprehension syntax
except that the keyword \code{yield} is missing. The for-loop is
executed by executing the expression $e$ for each element generated
from the sequence of generators and filters $s$.

As an example, the following expression prints out all elements of a
matrix represented as a list of lists:
 \begin{lstlisting}
for (xs <- xss) {
  for (x <- xs) print(x + "\t")
  println()
}
\end{lstlisting}
The translation of for-loops to higher-order methods of class
\code{List} is similar to the translation of for-comprehensions, but
is simpler. Where for-comprehensions translate to \code{map} and
\code{flatMap}, for-loops translate in each case to \code{foreach}.

\section{Generalizing For}

We have seen that the translation of for-comprehensions only relies on
the presence of methods \code{map}, \code{flatMap}, and
\code{filter}. Therefore it is possible to apply the same notation to
generators that produce objects other than lists; these objects only
have to support the three key functions \code{map}, \code{flatMap},
and \code{filter}.

The standard Scala library has several other abstractions that support
these three methods and with them support for-comprehensions. We will
encounter some of them in the following chapters. As a programmer you
can also use this principle to enable for-comprehensions for types you
define -- these types just need to support methods \code{map},
\code{flatMap}, and \code{filter}.

There are many examples where this is useful: Examples are database
interfaces, XML trees, or optional values. 

One caveat: It is not assured automatically that the result
translating a for-comprehension is well-typed. To ensure this, the
types of \code{map}, \code{flatMap} and \code{filter} have to be
essentially similar to the types of these methods in class \code{List}.

To make this precise, assume you have a parameterized class
 \code{C[A]} for which you want to enable for-comprehensions. Then
 \code{C} should define \code{map}, \code{flatMap} and \code{filter}
 with the following types:
\begin{lstlisting}
def map[B](f: A => B): C[B]
def flatMap[B](f: A => C[B]): C[B]
def filter(p: A => Boolean): C[A]
\end{lstlisting}
It would be attractive to enforce these types statically in the Scala
compiler, for instance by requiring that any type supporting
for-comprehensions implements a standard trait with these methods
\footnote{In the programming language Haskell, which has similar
constructs, this abstraction is called a ``monad with zero''}.  The
problem is that such a standard trait would have to abstract over the
identity of the class \code{C}, for instance by taking \code{C} as a
type parameter.  Note that this parameter would be a type constructor,
which gets applied to {\em several different} types in the signatures of
methods \code{map} and \code{flatMap}. Unfortunately, the Scala type
system is too weak to express this construct, since it can handle only
type parameters which are fully applied types.

\chapter{Mutable State}

Most programs we have presented so far did not have side-effects
\footnote{We ignore here the fact that some of our program printed to
standard output, which technically is a side effect.}.  Therefore, the
notion of {\em time} did not matter.  For a program that terminates,
any sequence of actions would have led to the same result!  This is
also reflected by the substitution model of computation, where a
rewrite step can be applied anywhere in a term, and all rewritings
that terminate lead to the same solution.  In fact, this {\em
confluence} property is a deep result in $\lambda$-calculus, the
theory underlying functional programming. 

In this chapter, we introduce functions with side effects and study
their behavior. We will see that as a consequence we have to
fundamentally modify up the substitution model of computation which we
employed so far.

\section{Stateful Objects}

We normally view the world as a set of objects, some of which have
state that {\em changes} over time.  Normally, state is associated
with a set of variables that can be changed in the course of a
computation.  There is also a more abstract notion of state, which
does not refer to particular constructs of a programming language: An
object {\em has state} (or: {\em is stateful}) if its behavior is
influenced by its history.

For instance, a bank account object has state, because the question
``can I withdraw 100 CHF?''
might have different answers during the lifetime of the account.

In Scala, all mutable state is ultimately built from variables.  A
variable definition is written like a value definition, but starts
with \verb@var@ instead of \verb@val@. For instance, the following two
definitions introduce and initialize two variables \code{x} and
\code{count}.
\begin{lstlisting}
var x: String = "abc"
var count = 111
\end{lstlisting}
Like a value definition, a variable definition associates a name with
a value. But in the case of a variable definition, this association
may be changed later by an assignment.  Such assignments are written
as in C or Java. Examples:
\begin{lstlisting}
x = "hello"
count = count + 1
\end{lstlisting}
In Scala, every defined variable has to be initialized at the point of
its definition. For instance, the statement ~\code{var x: Int;}~ is
{\em not} regarded as a variable definition, because the initializer
is missing\footnote{If a statement like this appears in a class, it is
instead regarded as a variable declaration, which introduces
abstract access methods for the variable, but does not associate these
methods with a piece of state.}. If one does not know, or does not
care about, the appropriate initializer, one can use a wildcard
instead. I.e.
\begin{lstlisting}
val x: T = _
\end{lstlisting}
will initialize \code{x} to some default value (\code{null} for
reference types, \code{false} for booleans, and the appropriate
version of \code{0} for numeric value types).

Real-world objects with state are represented in Scala by objects that
have variables as members. For instance, here is a class that
represents bank accounts.
\begin{lstlisting}
class BankAccount {
  private var balance = 0
  def deposit(amount: Int) {
    if (amount > 0) balance += amount
  }

  def withdraw(amount: Int): Int =
    if (0 < amount && amount <= balance) {
      balance -= amount
      balance
    } else error("insufficient funds")
}
\end{lstlisting}
The class defines a variable \code{balance} which contains the current
balance of an account. Methods \code{deposit} and \code{withdraw}
change the value of this variable through assignments.  Note that
\code{balance} is \code{private} in class \code{BankAccount} -- hence
it can not be accessed directly outside the class.

To create bank-accounts, we use the usual object creation notation:
\begin{lstlisting}
val myAccount = new BankAccount
\end{lstlisting}

\example Here is a \code{scalaint} session that deals with bank
accounts.

\begin{lstlisting}
scala> :l bankaccount.scala
Loading bankaccount.scala...
defined class BankAccount
scala> val account = new BankAccount
account: BankAccount = BankAccount$\Dollar$class@1797795
scala> account deposit 50
unnamed0: Unit = ()
scala> account withdraw 20
unnamed1: Int = 30
scala> account withdraw 20
unnamed2: Int = 10
scala> account withdraw 15
java.lang.Error: insufficient funds
        at scala.Predef$\Dollar$error(Predef.scala:74)
        at BankAccount$\Dollar$class.withdraw(<console>:14)
        at <init>(<console>:5)
scala> 
\end{lstlisting}
The example shows that applying the same operation (\code{withdraw
20}) twice to an account yields different results. So, clearly,
accounts are stateful objects.  

\paragraph{Sameness and Change}
Assignments pose new problems in deciding when two expressions are
``the same''.
If assignments are excluded, and one writes
\begin{lstlisting}
val x = E; val y = E
\end{lstlisting}
where \code{E} is some arbitrary expression,
then \code{x} and \code{y} can reasonably be assumed to be the same.
I.e. one could have equivalently written
\begin{lstlisting}
val x = E; val y = x
\end{lstlisting}
(This property is usually called {\em referential transparency}). But
once we admit assignments, the two definition sequences are different.
Consider:
\begin{lstlisting}
val x = new BankAccount; val y = new BankAccount
\end{lstlisting}
To answer the question whether \code{x} and \code{y} are the same, we
need to be more precise what ``sameness'' means. This meaning is
captured in the notion of {\em operational equivalence}, which,
somewhat informally, is stated as follows.

Suppose we have two definitions of \code{x} and \code{y}.
To test whether \code{x} and \code{y} define the same value, proceed
as follows.
\begin{itemize}
\item
Execute the definitions followed by an
arbitrary sequence \code{S} of operations that involve \code{x} and
\code{y}. Observe the results (if any).
\item
Then, execute the definitions with another sequence \code{S'} which
results from \code{S} by renaming all occurrences of \code{y} in
\code{S} to \code{x}.
\item
If the results of running \code{S'} are different, then surely
\code{x} and \code{y} are different.
\item
On the other hand, if all possible pairs of sequences \lstinline@{S, S'}@
yield the same results, then \code{x} and \code{y} are the same.
\end{itemize}
In other words, operational equivalence regards two definitions
\code{x} and \code{y} as defining the same value, if no possible
experiment can distinguish between \code{x} and \code{y}. An
experiment in this context are two version of an arbitrary program which use either
\code{x} or \code{y}.
 
Given this definition, let's test whether
\begin{lstlisting}
val x = new BankAccount; val y = new BankAccount
\end{lstlisting}
defines values \code{x} and \code{y} which are the same.
Here are the definitions again, followed by a test sequence:

\begin{lstlisting}
> val x = new BankAccount
> val y = new BankAccount
> x deposit 30
30
> y withdraw 20
java.lang.RuntimeException: insufficient funds
\end{lstlisting}

Now, rename all occurrences of \code{y} in that sequence to
\code{x}. We get:
\begin{lstlisting}
> val x = new BankAccount
> val y = new BankAccount
> x deposit 30
30
> x withdraw 20
10
\end{lstlisting}
Since the final results are different, we have established that
\code{x} and \code{y} are not the same.
On the other hand, if we define
\begin{lstlisting}
val x = new BankAccount; val y = x
\end{lstlisting}
then no sequence of operations can distinguish between \code{x} and
\code{y}, so \code{x} and \code{y} are the same in this case.

\paragraph{Assignment and the Substitution Model}
These examples show that our previous substitution model of
computation cannot be used anymore.  After all, under this
model we could always replace a value name by its
defining expression.
For instance in
\begin{lstlisting}
val x = new BankAccount; val y = x
\end{lstlisting}
the \code{x} in the definition of \code{y} could
be replaced by \code{new BankAccount}.
But we have seen that this change leads to a different program.
So the substitution model must be invalid, once we add assignments. 

\section{Imperative Control Structures}

Scala has the \code{while} and \code{do-while} loop constructs known
from the C and Java languages. There is also a single branch \code{if}
which leaves out the else-part as well as a \code{return} statement which
aborts a function prematurely. This makes it possible to program in a
conventional imperative style. For instance, the following function,
which computes the \code{n}'th power of a given parameter \code{x}, is
implemented using \code{while} and single-branch \code{if}.
\begin{lstlisting}
def power(x: Double, n: Int): Double = {
  var r = 1.0
  var i = n
  var j = 0
  while (j < 32) {
    r = r * r
    if (i < 0)
      r *= x
    i = i << 1
    j += 1
  }
  r
}
\end{lstlisting}
These imperative control constructs are in the language for
convenience. They could have been left out, as the same constructs can
be implemented using just functions. As an example, let's develop a
functional implementation of the while loop. \code{whileLoop} should
be a function that takes two parameters: a condition, of type
\code{Boolean}, and a command, of type \code{Unit}. Both condition and
command need to be passed by-name, so that they are evaluated
repeatedly for each loop iteration.  This leads to the following
definition of \code{whileLoop}.
\begin{lstlisting}
def whileLoop(condition: => Boolean)(command: => Unit) {
  if (condition) {
    command; whileLoop(condition)(command)
  } else ()
}
\end{lstlisting}
Note that \code{whileLoop} is tail recursive, so it operates in
constant stack space.

\begin{exercise} Write a function \code{repeatLoop}, which should be 
applied as follows:
\begin{lstlisting}
repeatLoop { command } ( condition )
\end{lstlisting}
Is there also a way to obtain a loop syntax like the following?
\begin{lstlisting}
repeatLoop { command } until ( condition )
\end{lstlisting}
\end{exercise}

Some other control constructs known from C and Java are missing in
Scala: There are no \code{break} and \code{continue} jumps for loops.
There are also no for-loops in the Java sense -- these have been
replaced by the more general for-loop construct discussed in
Section~\ref{sec:for-loops}.

\section{Extended Example: Discrete Event Simulation}

We now discuss an example that demonstrates how assignments and
higher-order functions can be combined in interesting ways.  
We will build a simulator for digital circuits.

The example is taken from Abelson and Sussman's book
\cite{abelson-sussman:structure}. We augment their basic (Scheme-)
code by an object-oriented structure which allows code-reuse through
inheritance. The example also shows how discrete event simulation programs
in general are structured and built.

We start with a little language to describe digital circuits.
A digital circuit is built from {\em wires} and {\em function boxes}.
Wires carry signals which are transformed by function boxes.
We will represent signals by the booleans \code{true} and
\code{false}.

Basic function boxes (or: {\em gates}) are:
\begin{itemize}
\item An \emph{inverter}, which negates its signal
\item An \emph{and-gate}, which sets its output to the conjunction of its input.
\item An \emph{or-gate}, which sets its output to the disjunction of its
input.
\end{itemize}
Other function boxes can be built by combining basic ones.

Gates have {\em delays}, so an output of a gate will change only some
time after its inputs change.

\paragraph{A Language for Digital Circuits}

We describe the elements of a digital circuit by the following set of
Scala classes and functions.

First, there is a class \code{Wire} for wires.
We can construct wires as follows.
\begin{lstlisting}
val a = new Wire
val b = new Wire
val c = new Wire
\end{lstlisting}
Second, there are procedures
\begin{lstlisting}
def inverter(input: Wire, output: Wire)
def andGate(a1: Wire, a2: Wire, output: Wire)
def orGate(o1: Wire, o2: Wire, output: Wire)
\end{lstlisting}
which ``make'' the basic gates we need (as side-effects).
More complicated function boxes can now be built from these.
For instance, to construct a half-adder, we can define:
\begin{lstlisting}
  def halfAdder(a: Wire, b: Wire, s: Wire, c: Wire) {
    val d = new Wire
    val e = new Wire
    orGate(a, b, d)
    andGate(a, b, c)
    inverter(c, e)
    andGate(d, e, s)
  }
\end{lstlisting}
This abstraction can itself be used, for instance in defining a full
adder:
\begin{lstlisting}
  def fullAdder(a: Wire, b: Wire, cin: Wire, sum: Wire, cout: Wire) {
    val s = new Wire
    val c1 = new Wire
    val c2 = new Wire
    halfAdder(a, cin, s, c1)
    halfAdder(b, s, sum, c2)
    orGate(c1, c2, cout)
  }
\end{lstlisting}
Class \code{Wire} and functions \code{inverter}, \code{andGate}, and
\code{orGate} represent thus a little language in which users can
define digital circuits.  We now give implementations of this class
and these functions, which allow one to simulate circuits.
These implementations are based on a simple and general API for
discrete event simulation.

\paragraph{The Simulation API}

Discrete event simulation performs user-defined \emph{actions} at
specified \emph{times}.  
An {\em action} is represented as a function which takes no parameters and
returns a \code{Unit} result:
\begin{lstlisting}
type Action = () => Unit
\end{lstlisting}
The \emph{time} is simulated; it is not the actual ``wall-clock'' time.

A concrete simulation will be done inside an object which inherits
from the abstract \code{Simulation} class. This class has the following
signature:

\begin{lstlisting}
abstract class Simulation {
  def currentTime: Int
  def afterDelay(delay: Int, action: => Action)
  def run()
}
\end{lstlisting}
Here,
\code{currentTime} returns the current simulated time as an integer
number,
\code{afterDelay} schedules an action to be performed at a specified
delay after \code{currentTime}, and
\code{run} runs the simulation until there are no further actions to be 
performed.

\paragraph{The Wire Class}
A wire needs to support three basic actions.
\begin{itemize}
\item[]
\code{getSignal: Boolean}~~ returns the current signal on the wire.
\item[]
\code{setSignal(sig: Boolean)}~~ sets the wire's signal to \code{sig}.
\item[]
\code{addAction(p: Action)}~~ attaches the specified procedure
\code{p} to the {\em actions} of the wire. All attached action
procedures will be executed every time the signal of a wire changes.
\end{itemize}
Here is an implementation of the \code{Wire} class:
\begin{lstlisting}
class Wire {
  private var sigVal = false
  private var actions: List[Action] = List()
  def getSignal = sigVal
  def setSignal(s: Boolean) =
    if (s != sigVal) {
      sigVal = s
      actions.foreach(action => action())
    }
  def addAction(a: Action) {
    actions = a :: actions; a()
  }
}
\end{lstlisting}
Two private variables make up the state of a wire.  The variable
\code{sigVal} represents the current signal, and the variable
\code{actions} represents the action procedures currently attached to
the wire.

\paragraph{The Inverter Class}
We implement an inverter by installing an action on its input wire,
namely the action which puts the negated input signal onto the output
signal.  The action needs to take effect at \code{InverterDelay}
simulated time units after the input changes. This suggests the 
following implementation:
\begin{lstlisting}
def inverter(input: Wire, output: Wire) {
  def invertAction() {
    val inputSig = input.getSignal
    afterDelay(InverterDelay) { output setSignal !inputSig }
  }
  input addAction invertAction
}
\end{lstlisting}

\paragraph{The And-Gate Class}
And-gates are implemented analogously to inverters.  The action of an
\code{andGate} is to output the conjunction of its input signals.
This should happen at \code{AndGateDelay} simulated time units after
any one of its two inputs changes. Hence, the following implementation:
\begin{lstlisting}
def andGate(a1: Wire, a2: Wire, output: Wire) {
  def andAction() {
    val a1Sig = a1.getSignal
    val a2Sig = a2.getSignal
    afterDelay(AndGateDelay) { output setSignal (a1Sig & a2Sig) }
  }
  a1 addAction andAction
  a2 addAction andAction
}
\end{lstlisting}

\begin{exercise} Write the implementation of \code{orGate}.
\end{exercise}

\begin{exercise} Another way is to define an or-gate by a combination of
inverters and and gates. Define a function \code{orGate} in terms of
\code{andGate} and \code{inverter}. What is the delay time of this function?
\end{exercise}

\paragraph{The Simulation Class}

Now, we just need to implement class \code{Simulation}, and we are
done.  The idea is that we maintain inside a \code{Simulation} object
an \emph{agenda} of actions to perform.  The agenda is represented as
a list of pairs of actions and the times they need to be run.  The
agenda list is sorted, so that earlier actions come before later ones.
\begin{lstlisting} 
abstract class Simulation {
  case class WorkItem(time: Int, action: Action)
  private type Agenda = List[WorkItem]
  private var agenda: Agenda = List()
\end{lstlisting}
There is also a private variable \code{curtime} to keep track of the
current simulated time.
\begin{lstlisting}
  private var curtime = 0
\end{lstlisting}
An application of the method \code{afterDelay(delay, block)} 
inserts the element \lstinline@WorkItem(currentTime + delay, () => block)@
into the \code{agenda} list at the appropriate place.
\begin{lstlisting}
private def insert(ag: Agenda, item: WorkItem): Agenda =
  if (ag.isEmpty || item.time < ag.head.time) item :: ag
  else ag.head :: insert(ag.tail, item)

def afterDelay(delay: Int)(block: => Unit) {
  val item = WorkItem(currentTime + delay, () => block)
  agenda = insert(agenda, item)
}
\end{lstlisting}
An application of the \code{run} method removes successive elements
from the \code{agenda} and performs their actions.
It continues until the agenda is empty:
\begin{lstlisting}
private def next() {
  agenda match {
    case WorkItem(time, action) :: rest =>
      agenda = rest; curtime = time; action()
    case List() =>
  }
}

def run() {
  afterDelay(0) { println("*** simulation started ***") }
  while (!agenda.isEmpty) next()
}
\end{lstlisting}

\paragraph{Running the Simulator}
To run the simulator, we still need a way to inspect changes of
signals on wires. To this purpose, we write a function \code{probe}.
\begin{lstlisting}
def probe(name: String, wire: Wire) {
  wire addAction {
    println(name + " " + currentTime + " new_value = " + wire.getSignal)
  }
}
\end{lstlisting}
Now, to see the simulator in action, let's define four wires, and place
probes on two of them: 
\begin{lstlisting}
scala> val input1, input2, sum, carry = new Wire

scala> probe("sum", sum)
sum 0 new_value = false

scala> probe("carry", carry)
carry 0 new_value = false
\end{lstlisting}
Now let's define a half-adder connecting the wires:
\begin{lstlisting}
scala> halfAdder(input1, input2, sum, carry)
\end{lstlisting}
Finally, set one after another the signals on the two input wires to
\code{true} and run the simulation.
\begin{lstlisting}
scala> input1 setSignal true; run
*** simulation started ***
sum 8 new_value = true

scala> input2 setSignal true; run
carry 11 new_value = true
sum 15 new_value = false
\end{lstlisting}

\section{Summary}

We have seen in this chapter the constructs that let us model state in
Scala -- these are variables, assignments, and imperative control
structures.  State and Assignment complicate our mental model of
computation.  In particular, referential transparency is lost.  On the
other hand, assignment gives us new ways to formulate programs
elegantly. As always, it depends on the situation whether purely
functional programming or programming with assignments works best.

\chapter{Computing with Streams}

The previous chapters have introduced variables, assignment and
stateful objects.  We have seen how real-world objects that change
with time can be modeled by changing the state of variables in a
computation.  Time changes in the real world thus are modeled by time
changes in program execution. Of course, such time changes are usually
stretched out or compressed, but their relative order is the same.
This seems quite natural, but there is a also price to pay: Our simple
and powerful substitution model for functional computation is no
longer applicable once we introduce variables and assignment.

Is there another way? Can we model state change in the real world
using only immutable functions? Taking mathematics as a guide, the
answer is clearly yes: A time-changing quantity is simply modeled by
a function \code{f(t)} with a time parameter \code{t}. The same can be
done in computation. Instead of overwriting a variable with successive
values, we represent all these values as successive elements in a
list. So, a mutable variable \code{var x: T} gets replaced by an
immutable value \code{val x: List[T]}. In a sense, we trade space for
time -- the different values of the variable now all exist concurrently
as different elements of the list.  One advantage of the list-based
view is that we can ``time-travel'', i.e. view several successive
values of the variable at the same time. Another advantage is that we
can make use of the powerful library of list processing functions,
which often simplifies computation. For instance, consider the
imperative way to compute the sum of all prime numbers in an interval:
\begin{lstlisting}
def sumPrimes(start: Int, end: Int): Int = {
  var i = start
  var acc = 0
  while (i < end) {
    if (isPrime(i)) acc += i
    i += 1
  }
  acc
}
\end{lstlisting}
Note that the variable \code{i} ``steps through'' all values of the interval
\code{[start .. end-1]}.

A more functional way is to represent the list of values of variable \code{i} directly as \code{range(start, end)}. Then the function can be rewritten as follows.
\begin{lstlisting}
def sumPrimes(start: Int, end: Int) =
  sum(range(start, end) filter isPrime)
\end{lstlisting}

No contest which program is shorter and clearer!  However, the
functional program is also considerably less efficient since it
constructs a list of all numbers in the interval, and then another one
for the prime numbers. Even worse from an efficiency point of view is
the following example:

To find the second prime number between \code{1000} and \code{10000}:
\begin{lstlisting}
  range(1000, 10000) filter isPrime at 1
\end{lstlisting}
Here, the list of all numbers between \code{1000} and \code{10000} is
constructed.  But most of that list is never inspected!

However, we can obtain efficient execution for examples like these by
a trick:
\begin{quote}
%\red
 Avoid computing the tail of a sequence unless that tail is actually
     necessary for the computation.
\end{quote}
We define a new class for such sequences, which is called \code{Stream}.

Streams are created using the constant \code{empty} and the constructor \code{cons},
which are both defined in module \code{scala.Stream}. For instance, the following
expression constructs a stream with elements \code{1} and \code{2}:
\begin{lstlisting}
Stream.cons(1, Stream.cons(2, Stream.empty))
\end{lstlisting}
As another example, here is the analogue of \code{List.range},
but returning a stream instead of a list:
\begin{lstlisting}
def range(start: Int, end: Int): Stream[Int] =
  if (start >= end) Stream.empty
  else Stream.cons(start, range(start + 1, end))
\end{lstlisting}
(This function is also defined as given above in module
\code{Stream}).  Even though \code{Stream.range} and \code{List.range}
look similar, their execution behavior is completely different: 

\code{Stream.range} immediately returns with a \code{Stream} object
whose first element is \code{start}.  All other elements are computed
only when they are \emph{demanded} by calling the \code{tail} method
(which might be never at all).  

Streams are accessed just as lists. Similarly to lists, the basic access
methods are \code{isEmpty}, \code{head} and \code{tail}. For instance,
we can print all elements of a stream as follows.
\begin{lstlisting}
def print(xs: Stream[A]) {
  if (!xs.isEmpty) { Console.println(xs.head); print(xs.tail) }
}
\end{lstlisting}
Streams also support almost all other methods defined on lists (see
below for where their methods sets differ). For instance, we can find
the second prime number between \code{1000} and \code{10000} by applying methods
\code{filter} and \code{apply} on an interval stream:
\begin{lstlisting}
  Stream.range(1000, 10000) filter isPrime at 1
\end{lstlisting}
The difference to the previous list-based implementation is that now
we do not needlessly construct and test for primality any numbers
beyond 3.

\paragraph{Consing and appending streams} Two methods in class \code{List}
which are not supported by class \code{Stream} are \code{::} and
\code{:::}.  The reason is that these methods are dispatched on their
right-hand side argument, which means that this argument needs to be
evaluated before the method is called. For instance, in the case of
\code{x :: xs} on lists, the tail \code{xs} needs to be evaluated
before \code{::} can be called and the new list can be constructed.
This does not work for streams, where we require that the tail of a
stream should not be evaluated until it is demanded by a \code{tail} operation.
The argument why list-append \code{:::} cannot be adapted to streams is analogous.

Instead of \code{x :: xs}, one uses \code{Stream.cons(x, xs)} for
constructing a stream with first element \code{x} and (unevaluated)
rest \code{xs}.  Instead of \code{xs ::: ys}, one uses the operation
\code{xs append ys}.  

\chapter{Iterators}

Iterators are the imperative version of streams. Like streams,
iterators describe potentially infinite lists. However, there is no
data-structure which contains the elements of an iterator. Instead, 
iterators allow one to step through the sequence, using two abstract methods \code{next} and \code{hasNext}.
\begin{lstlisting}
trait Iterator[+A] {
  def hasNext: Boolean
  def next: A
\end{lstlisting}
Method \code{next} returns successive elements.  Method \code{hasNext}
indicates whether there are still more elements to be returned by
\code{next}. Iterators also support some other methods, which are
explained later.

As an example, here is an application which prints the squares of all
numbers from 1 to 100.
\begin{lstlisting}
val it: Iterator[Int] = Iterator.range(1, 100)
while (it.hasNext) {
  val x = it.next
  println(x * x)
}
\end{lstlisting}

\section{Iterator Methods}

Iterators support a rich set of methods besides \code{next} and
\code{hasNext}, which is described in the following. Many of these
methods mimic a corresponding functionality in lists.

\paragraph{Append}
Method \code{append} constructs an iterator which resumes with the
given iterator ~\code{it}~ after the current iterator has finished.
\begin{lstlisting}
  def append[B >: A](that: Iterator[B]): Iterator[B] = new Iterator[B] {
    def hasNext = Iterator.this.hasNext || that.hasNext
    def next = if (Iterator.this.hasNext) Iterator.this.next else that.next
  }    
\end{lstlisting}
The terms \code{Iterator.this.next} and \code{Iterator.this.hasNext}
in the definition of \code{append} call the corresponding methods as
they are defined in the enclosing \code{Iterator} class.  If the
\code{Iterator} prefix to \code{this} would have been missing,
\code{hasNext} and \code{next} would have called recursively the
methods being defined in the result of \code{append}, which is not
what we want.

\paragraph{Map, FlatMap, Foreach} Method \code{map} 
constructs an iterator which returns all elements of the original
iterator transformed by a given function \code{f}.
\begin{lstlisting}
  def map[B](f: A => B): Iterator[B] = new Iterator[B] {
    def hasNext = Iterator.this.hasNext
    def next = f(Iterator.this.next)
  }
\end{lstlisting}
Method \code{flatMap} is like method \code{map}, except that the
transformation function \code{f} now returns an iterator.
The result of \code{flatMap} is the iterator resulting from appending
together all iterators returned from successive calls of \code{f}.
\begin{lstlisting}
  def flatMap[B](f: A => Iterator[B]): Iterator[B] = new Iterator[B] {
    private var cur: Iterator[B] = Iterator.empty
    def hasNext: Boolean =
      if (cur.hasNext) true
      else if (Iterator.this.hasNext) { cur = f(Iterator.this.next); hasNext }
      else false
    def next: B =
      if (cur.hasNext) cur.next
      else if (Iterator.this.hasNext) { cur = f(Iterator.this.next); next }
      else error("next on empty iterator")
  }
\end{lstlisting}
Closely related to \code{map} is the \code{foreach} method, which
applies a given function to all elements of an iterator, but does not
construct a list of results
\begin{lstlisting}
  def foreach(f: A => Unit): Unit =
    while (hasNext) { f(next) }
\end{lstlisting}

\paragraph{Filter} Method \code{filter} constructs an iterator which
returns all elements of the original iterator that satisfy a criterion
\code{p}.
\begin{lstlisting}
  def filter(p: A => Boolean) = new BufferedIterator[a] {
    private val source =
      Iterator.this.buffered
    private def skip
      { while (source.hasNext && !p(source.head)) { source.next } }
    def hasNext: Boolean =
      { skip; source.hasNext }
    def next: A =
      { skip; source.next }
    def head: A =
      { skip; source.head }
  }
\end{lstlisting}
In fact, \code{filter} returns instances of a subclass of iterators
which are ``buffered''.  A \code{BufferedIterator} object is an
iterator which has in addition a method \code{head}. This method
returns the element which would otherwise have been returned by
\code{head}, but does not advance beyond that element. Hence, the
element returned by \code{head} is returned again by the next call to
\code{head} or \code{next}. Here is the definition of the
\code{BufferedIterator} trait.
\begin{lstlisting}
trait BufferedIterator[+A] extends Iterator[A] {
  def head: A
}
\end{lstlisting}
Since \code{map}, \code{flatMap}, \code{filter}, and \code{foreach}
exist for iterators, it follows that for-comprehensions and for-loops
can also be used on iterators. For instance, the application which prints the squares of numbers between 1 and 100 could have equivalently been expressed as follows.
\begin{lstlisting}
for (i <- Iterator.range(1, 100))
  println(i * i)
\end{lstlisting}

\paragraph{Zip} Method \code{zip} takes another iterator and
returns an iterator consisting of pairs of corresponding elements
returned by the two iterators.
\begin{lstlisting}
  def zip[B](that: Iterator[B]) = new Iterator[(a, b)] {
    def hasNext = Iterator.this.hasNext && that.hasNext
    def next = {Iterator.this.next, that.next}
  }
}
\end{lstlisting}

\section{Constructing Iterators}

Concrete iterators need to provide implementations for the two
abstract methods \code{next} and \code{hasNext} in class
\code{Iterator}. The simplest iterator is \code{Iterator.empty} which
always returns an empty sequence:
\begin{lstlisting}
object Iterator {
  object empty extends Iterator[Nothing] {
    def hasNext = false
    def next = error("next on empty iterator")
  }
\end{lstlisting}
A more interesting iterator enumerates all elements of an array. This
iterator is constructed by the \code{fromArray} method, which is also defined in the object \code{Iterator}
\begin{lstlisting}
  def fromArray[A](xs: Array[A]) = new Iterator[A] {
    private var i = 0
    def hasNext: Boolean =
      i < xs.length
    def next: A =
      if (i < xs.length) { val x = xs(i); i += 1; x }
      else error("next on empty iterator")
  }
\end{lstlisting}
Another iterator enumerates an integer interval.  The
\code{Iterator.range} function returns an iterator which traverses a
given interval of integer values. It is defined as follows.
\begin{lstlisting}
object Iterator {
  def range(start: Int, end: Int) = new Iterator[Int] {
    private var current = start
    def hasNext = current < end
    def next = {
      val r = current
      if (current < end) current += 1
      else error("end of iterator")
      r
    }
  }
}
\end{lstlisting}
All iterators seen so far terminate eventually. It is also possible to
define iterators that go on forever. For instance, the following
iterator returns successive integers from some start
value\footnote{Due to the finite representation of type \prog{int},
numbers will wrap around at $2^{31}$.}.
\begin{lstlisting}
def from(start: Int) = new Iterator[Int] {
  private var last = start - 1
  def hasNext = true
  def next = { last += 1; last }
}
\end{lstlisting}

\section{Using Iterators}

Here are two more examples how iterators are used. First, to print all
elements of an array \code{xs: Array[Int]}, one can write:
\begin{lstlisting}
  Iterator.fromArray(xs) foreach (x => println(x))
\end{lstlisting}
Or, using a for-comprehension:
\begin{lstlisting}
  for (x <- Iterator.fromArray(xs))
    println(x)
\end{lstlisting}
As a second example, consider the problem of finding the indices of
all the elements in an array of \code{double}s greater than some
\code{limit}. The indices should be returned as an iterator.
This is achieved by the following expression.
\begin{lstlisting}
import Iterator._
fromArray(xs)
.zip(from(0))
.filter(case (x, i) => x > limit)
.map(case (x, i) => i)
\end{lstlisting}
Or, using a for-comprehension:
\begin{lstlisting}
import Iterator._
for ((x, i) <- fromArray(xs) zip from(0); x > limit)
yield i
\end{lstlisting}

\chapter{Lazy Values}

Lazy values provide a way to delay initialization of a value until the
first time it is accessed. This may be useful when dealing with
values that might not be needed during execution, and whose
computational cost is signifficant. As a first example, let's consider
a database of employees, containing for each employee its manager and
its team.
\begin{lstlisting}
case class Employee(id: Int, 
                    name: String, 
                    managerId: Int) {
  val manager: Employee = Db.get(managerId)
  val team: List[Employee] = Db.team(id)
}
\end{lstlisting}

The \lstinline@Employee@ class given above will eagerly initialize
all its fields, loading the whole employee table in memory. This is
certainly not optimal, and it can be easily  improved my making the
fields lazy. This way we delay the database access until it is really
needed, if it is ever needed.
\begin{lstlisting}
case class Employee(id: Int, 
                    name: String, 
                    managerId: Int) {
  lazy val manager: Employee = Db.get(managerId)
  lazy val team: List[Employee] = Db.team(id)
}
\end{lstlisting}

To see what is really happening, we can use this mockup
database which shows when records are fetched:
\begin{lstlisting}
object Db {
  val table = Map(1 -> (1, "Haruki Murakami", -1),
                  2 -> (2, "Milan Kundera", 1),
                  3 -> (3, "Jeffrey Eugenides", 1),
                  4 -> (4, "Mario Vargas Llosa", 1),
                  5 -> (5, "Julian Barnes", 2))

  def team(id: Int) = {
    for (rec <- table.values.toList; if rec._3 == id)
      yield recToEmployee(rec)
  }

  def get(id: Int) = recToEmployee(table(id))

  private def recToEmployee(rec: (Int, String, Int)) = {
    println("[db] fetching " + rec._1)
    Employee(rec._1, rec._2, rec._3)
  }
}
\end{lstlisting}
The output when running a program that retrieves one employee 
confirms that the database is only accessed when referring  the lazy
values.

Another use of lazy values is to resolve the
initialization order of applications composed of several
modules. Before lazy values were introduced, the same effect was
achieved by using \lstinline@object@ definitions. As a second example,
we consider a compiler composed of several modules. We look first at a
simple symbol table that defines a class for symbols and two
predefined functions.
\begin{lstlisting}
class Symbols(val compiler: Compiler) {
  import compiler.types._

  val Add = new Symbol("+", FunType(List(IntType, IntType), IntType))
  val Sub = new Symbol("-", FunType(List(IntType, IntType), IntType))

  class Symbol(name: String, tpe: Type) {
    override def toString = name + ": " + tpe
  }
}
\end{lstlisting}
The \lstinline@symbols@ module is parameterized with a \lstinline@Compiler@
instance, which provides access to other services, such as the types
module. In our example there are only two predefined functions,
addition and subtraction, and their definitions depend on the \lstinline@types@
module.
\begin{lstlisting}
class Types(val compiler: Compiler) {
  import compiler.symtab._

  abstract class Type
  case class FunType(args: List[Type], res: Type) extends Type
  case class NamedType(sym: Symbol) extends Type
  case object IntType extends Type
}
\end{lstlisting}
In order to hook the two components together a compiler object is
created and passed as an argument to the two components. 
\begin{lstlisting}
class Compiler {
  val symtab = new Symbols(this)
  val types  = new Types(this)
}
\end{lstlisting}
Unfortunately, the straight-forward approach fails at runtime because
the \lstinline@symtab@ module needs the \lstinline@types@
module. In general, the dependency between modules can be complicated
and getting the right initialization order is difficult, or even
impossible when there are cycles. The easy fix is to make such fields
\lstinline@lazy@ and let the compiler figure out the right order.
\begin{lstlisting}
class Compiler {
  lazy val symtab = new Symbols(this)
  lazy val types  = new Types(this)
}
\end{lstlisting}
Now the two modules are initialized on first access, and the compiler
may run as expected. 

\subsection*{Syntax}
The \lstinline@lazy@ modifier is allowed only on concrete value
definitions. All typing rules for value definitions apply for
\lstinline@lazy@ values as well, with one restriction removed:
recursive local values are allowed.

\chapter{Implicit Parameters and Conversions}\label{sec:implicits}

Implicit parameters and conversions are powerful tools for
custimizing existing libraries and for creating high-level
abstractions. As an example, let's start with an abstract class of
semi-groups that support an unspecified \lstinline@add@ operation.
\begin{lstlisting}
abstract class SemiGroup[A] {
  def add(x: A, y: A): A
}
\end{lstlisting}
Here's a subclass \lstinline@Monoid@ of \lstinline@SemiGroup@ which adds a
\lstinline@unit@ element.
\begin{lstlisting}
abstract class Monoid[A] extends SemiGroup[A] {
  def unit: A
}
\end{lstlisting}
Here are two implementations of monoids:
\begin{lstlisting}
object stringMonoid extends Monoid[String] {
  def add(x: String, y: String): String = x.concat(y)
  def unit: String = ""
}

object intMonoid extends Monoid[Int] {
  def add(x: Int, y: Int): Int = x + y
  def unit: Int = 0
}
\end{lstlisting}
A \lstinline@sum@ method, which works over arbitrary
monoids, can be written in plain Scala as follows.
\begin{lstlisting}
def sum[A](xs: List[A])(m: Monoid[A]): A =
  if (xs.isEmpty) m.unit
  else m.add(xs.head, sum(m)(xs.tail)
\end{lstlisting}
This \lstinline@sum@ method can be called as follows:
\begin{lstlisting}
sum(List("a", "bc", "def"))(stringMonoid)
sum(List(1, 2, 3))(intMonoid)
\end{lstlisting}
All this works, but it is not very nice. The problem is that the
monoid implementations have to be passed into all code that uses them.
We would sometimes wish that the system could figure out the correct
arguments automatically, similar to what is done when type arguments
are inferred. This is what implicit parameters provide.

\subsection*{Implicit Parameters: The Basics}

In Scala 2 there is a new \lstinline@implicit@ keyword that can be
used at the beginning of a parameter list. Syntax:
\begin{lstlisting}
ParamClauses ::= {`(' [Param {`,' Param}] ')'} 
                 [`(' implicit Param {`,' Param} `)']
\end{lstlisting}
If the keyword is present, it makes all parameters in the list
implicit.  
For instance, the following version of \lstinline@sum@ has
\lstinline@m@ as an implicit parameter.
\begin{lstlisting}
def sum[A](xs: List[A])(implicit m: Monoid[A]): A =
  if (xs.isEmpty) m.unit
  else m.add(xs.head, sum(xs.tail))
\end{lstlisting}
As can be seen from the example, it is possible to combine normal and
implicit parameters. However, there may only be one implicit parameter
list for a method or constructor, and it must come last.

\lstinline@implicit@ can also be used as a modifier for definitions
and declarations. Examples:

\begin{lstlisting}
implicit object stringMonoid extends Monoid[String] {
  def add(x: String, y: String): String = x.concat(y)
  def unit: String = ""
}
implicit object intMonoid extends Monoid[Int] {
  def add(x: Int, y: Int): Int = x + y
  def unit: Int = 0
}
\end{lstlisting}

The principal idea behind implicit parameters is that arguments for
them can be left out from a method call. If the arguments
corresponding to an implicit parameter section are missing, they are
inferred by the Scala compiler.

The actual arguments that are eligible to be passed to an implicit
parameter are all identifiers X that can be accessed at the point
of the method call without a prefix and that denote an implicit
definition or parameter.

If there are several eligible arguments which match the implicit
parameter's type, the Scala compiler will chose a most specific one,
using the standard rules of static overloading resolution.
For instance, assume the call
\begin{lstlisting}
  sum(List(1, 2, 3))
\end{lstlisting}
in a context where \lstinline@stringMonoid@ and \lstinline@intMonoid@
are visible.  We know that the formal type parameter \lstinline@a@ of
\lstinline@sum@ needs to be instantiated to \lstinline@int@. The only
eligible value which matches the implicit formal parameter type
\lstinline@Monoid[Int]@ is \lstinline@intMonoid@ so this object will %
be passed as implicit parameter.

This discussion also shows that implicit parameters are inferred after
any type arguments are inferred. 

\subsection*{Implicit Conversions}

Say you have an expression $E$ of type $T$ which is expected to type
$S$. $T$ does not conform to $S$ and is not convertible to $S$ by
some other predefined conversion. Then the Scala compiler will try to
apply as last resort an implicit conversion $I(E)$. Here, $I$ is an
identifier denoting an implicit definition or parameter that is
accessible without a prefix at the point of the conversion, that can
be applied to arguments of type $T$ and whose result type conforms to the
expected type $S$.

Implicit conversions can also be applied in member
selections. Given a selection $E.x$ where $x$ is not a member of the
type $E$, the Scala compiler will try to insert an implicit conversion
$I(E).x$, so that $x$ is a member of $I(E)$.

Here is an example of an implicit conversion function that converts
integers into instances of class \lstinline@scala.Ordered@:
\begin{lstlisting}
implicit def int2ordered(x: Int): Ordered[Int] = new Ordered[Int] {
  def compare(y: Int): Int =
    if (x < y1) -1
    else if (x > y1) 1
    else 0
}
\end{lstlisting}

\subsection*{View Bounds}

View bounds are convenient syntactic sugar for implicit
parameters. Consider for instance a generic sort method:
\begin{lstlisting}
def sort[A <$\mbox{\%}$ Ordered[A]](xs: List[A]): List[A] =
  if (xs.isEmpty || xs.tail.isEmpty) xs
  else {
    val {ys, zs} = xs.splitAt(xs.length / 2)
    merge(ys, zs)
  }
\end{lstlisting}
The view bounded type parameter \lstinline@[a <$\mbox{\%}$ Ordered[a]]@
expresses that \lstinline@sort@ is applicable to lists of type
\lstinline@a@ such that there exists an implicit conversion from
\lstinline@a@ to \lstinline@Ordered[a]@. The definition is treated as
a shorthand for the following method signature with an implicit
parameter:
\begin{lstlisting}
def sort[A](xs: List[A])(implicit $c$: A => Ordered[A]): List[A] = ...
\end{lstlisting}
(Here, the parameter name $c$ is chosen arbitrarily in a way
that does not collide with other names in the program.)

As a more detailed example, consider the \lstinline@merge@ method that
comes with the \lstinline@sort@ method above:
\begin{lstlisting}
def merge[A <$\mbox{\%}$ Ordered[A]](xs: List[A], ys: List[A]): List[A] =
  if (xs.isEmpty) ys
  else if (ys.isEmpty) xs
  else if (xs.head < ys.head) xs.head :: merge(xs.tail, ys)
  else if ys.head :: merge(xs, ys.tail)
\end{lstlisting}
After expanding view bounds and inserting implicit conversions, this
method implementation becomes:
\begin{lstlisting}
def merge[A](xs: List[A], ys: List[A])
            (implicit $c$: A => Ordered[A]): List[A] =
  if (xs.isEmpty) ys
  else if (ys.isEmpty) xs
  else if (c(xs.head) < ys.head) xs.head :: merge(xs.tail, ys)
  else if ys.head :: merge(xs, ys.tail)(c)
\end{lstlisting}
The last two lines of this method definition illustrate two different
uses of the implicit parameter $c$. It is applied in a conversion in
the condition of the second to last line, and it is passed as implicit
argument in the recursive call to \lstinline@merge@ on the last line.

\comment{
\chapter{Combinator Parsing}\label{sec:combinator-parsing}

In this chapter we describe how to write combinator parsers in
Scala. Such parsers are constructed from predefined higher-order
functions, so called {\em parser combinators}, that closely model the
constructions of an EBNF grammar \cite{wirth:ebnf}.

As running example, we consider parsers for possibly nested
lists of identifiers and numbers, which
are described by the following context-free grammar.
\bda{p{3cm}cp{10cm}}
letter &::=& /* all letters */ \\
digit  &::=& /* all digits */ \\[0.5em]
ident  &::=& letter \{letter $|$ digit \}\\
number &::=& digit \{digit\}\\[0.5em]
list   &::=& `(' [listElems] `)' \\
listElems &::=& expr [`,' listElems] \\
expr   &::=& ident | number | list

\eda

\section{Simple Combinator Parsing}

In this section we will only be concerned with the task of recognizing
input strings, not with processing them. So we can describe parsers
by the sets of input strings they accept.  There are two
fundamental operators over parsers:
\code{&} expresses the sequential composition of a parser with
another, while \code{|} expresses an alternative. These operations
will both be defined as methods of a \code{Parser} class.  We will
also define constructors for the following primitive parsers:

\begin{tabular}{ll}
\code{empty}    & The parser that accepts the empty string
\\
\code{failure(msg: String)}  & The parser that accepts no string (\verb@msg@ 
                               stands for an error message)

\\
\code{chr(c: char)}
                & The parser that accepts the single-character string ``$c$''.
\\
\code{chrSuchThat(p: Char => Boolean)}
                & The parser that accepts single-character strings
                  ``$c$'' \\
                & for which $p(c)$ is true.
\end{tabular}

There are also the two higher-order parser combinators \code{opt},
expressing optionality and \code{rep}, expressing repetition.
For any parser $p$, \code{opt(}$p$\code{)} yields a parser that
accepts the strings accepted by $p$ or else the empty string, while
\code{rep(}$p$\code{)} accepts arbitrary sequences of the strings accepted by
$p$. In EBNF, \code{opt(}$p$\code{)} corresponds to $[p]$ and
\code{rep(}$p$\code{)} corresponds to $\{p\}$.

The central idea of parser combinators is that parsers can be produced
by a straightforward rewrite of the grammar, replacing \code{::=} with
\code{=}, sequencing with
\code{&}, repetition \code{\{...\}} with
\code{rep(...)} and optional occurrence \code{[...]} with \code{opt(...)}.
Applying this process to the grammar of lists
yields the following trait.
\begin{lstlisting}
trait ListParsers extends Parsers {
  def chrSuchThat(p: Char => Boolean): Parser
  def chr(c: Char): Parser = chrSuchThat(d ==)

  def letter    : Parser = chr(Character.isLetter)
  def digit     : Parser = chr(Character.isDigit)

  def ident     : Parser = letter &&& rep(letter ||| digit)
  def number    : Parser = digit &&& rep(digit)
  def list      : Parser = chr('(') &&& opt(listElems) &&& chr(')')
  def listElems : Parser = expr &&& (chr(',') &&& listElems ||| empty)
  def expr      : Parser = ident ||| number ||| list
}
\end{lstlisting}
This class isolates the grammar from other aspects of parsing. It
abstracts over the type of input 
and over the method used to parse a single character
(represented by the abstract method \code{chr(p: char =>
boolean))}. The missing bits of information need to be supplied by code
applying the parser class.

It remains to explain how to implement a library with the combinators
described above. We will pack combinators and their underlying
implementation in a base class \code{Parsers}, which is inherited by
\code{ListParsers}.  The first question to decide is which underlying
representation type to use for a parser. We treat parsers here
essentially as functions that take a datum of the input type
\code{InType} and that yield a parse result of type
\code{Option[InType]}.  The \code{Option} type is predefined as
follows.
\begin{lstlisting}
abstract class Option[+a]
case object None extends Option[Nothing]
case class Some[a](x: a) extends Option[a]
\end{lstlisting}
A parser applied to some input either succeeds or fails. If it fails,
it returns the constant \code{None}. If it succeeds, it returns a
value of the form \code{Some(in1)} where \code{in1} represents the
input that remains to be parsed.
\begin{lstlisting}
trait Parsers {
  type InType
  abstract class Parser {
    type Result = Option[InType]
    def apply(in: InType): Result
\end{lstlisting}
A parser also implements the combinators
for sequence and alternative:
\begin{lstlisting}
  /*** p &&& q applies first p, and if that succeeds, then q
   */
  def &&& (q: => Parser) = new Parser {
    def apply(in: InType): Result = Parser.this.apply(in) match {
      case None => None
      case Some(in1)  => q(in1)
    }
  }

  /*** p ||| q applies first p, and, if that fails, then q.
   */
  def ||| (q: => Parser) = new Parser {
    def apply(in: InType): Result = Parser.this.apply(in) match {
      case None => q(in)
      case s => s
    }
  }
\end{lstlisting}
The implementations of the primitive parsers \code{empty} and \code{fail}
are trivial:
\begin{lstlisting}
  val empty = new Parser { def apply(in: InType): Result = Some(in) }
  val fail  = new Parser { def apply(in: InType): Result = None }
\end{lstlisting}
The higher-order parser combinators \code{opt} and \code{rep} can be
defined in terms of the combinators for sequence and alternative:
\begin{lstlisting}
  def opt(p: Parser): Parser = p ||| empty;    // p? = (p | <empty>)
  def rep(p: Parser): Parser = opt(rep1(p));   // p* = [p+]
  def rep1(p: Parser): Parser = p &&& rep(p);  // p+ = p p*
} // end Parser
\end{lstlisting}
To run combinator parsers, we still need to decide on a way to handle
parser input. Several possibilities exist: The input could be
represented as a list, as an array, or as a random access file.  Note
that the presented combinator parsers use backtracking to change from
one alternative to another.  Therefore, it must be possible to reset
input to a point that was previously parsed. If one restricted the
focus to LL(1) grammars, a non-backtracking implementation of the
parser combinators in class \code{Parsers} would also be possible. In
that case sequential input methods based on (say) iterators or
sequential files would also be possible.

In our example, we represent the input by a pair of a string, which
contains the input phrase as a whole, and an index, which represents
the portion of the input which has not yet been parsed. Since the
input string does not change, just the index needs to be passed around
as a result of individual parse steps.  This leads to the following
class of parsers that read strings:
\begin{lstlisting}
class ParseString(s: String) extends Parsers {
  type InType = Int
  def chrSuchThat(p: Char => Boolean) = new Parser {
    def apply(in: Int): Parser#Result =
      if (in < s.length() && p(s charAt in)) Some(in + 1)
      else None
  }
  val input = 0
}
\end{lstlisting}
This class implements a method \code{chr(p: Char => Boolean)} and a
value \code{input}. The \code{chr} method builds a parser that either
reads a single character satisfying the given predicate \code{p} or
fails.  All other parsers over strings are ultimately implemented in
terms of that method. The \code{input} value represents the input as a
whole. In out case, it is simply value \code{0}, the start index of
the string to be read.

Note \code{apply}'s result type, \code{Parser#Result}. This syntax
selects the type element \code{Result} of the type \code{Parser}. It
thus corresponds roughly to selecting a static inner class from some
outer class in Java. Note that we could {\em not} have written
\code{Parser.Result}, as the latter would express selection of the
\code{Result} element from a {\em value} named \code{Parser}.

We have now extended the root class \code{Parsers} in two different
directions: Class \code{ListParsers} defines a grammar of phrases to
be parsed, whereas class \code{ParseString} defines a method by which
such phrases are input. To write a concrete parsing application, we
need to define both grammar and input method. We do this by combining
two extensions of \code{Parsers} using a {\em mixin composition}.
Here is the start of a sample application:
\begin{lstlisting}
object Test {
  def main(args: Array[String]) {
    val ps = new ParseString(args(0)) with ListParsers
  }
\end{lstlisting}
The last line above creates a new family of parsers by composing class
\code{ListParsers} with class \code{ParseString}. The two classes
share the common superclass \code{Parsers}. The abstract method
\code{chr} in \code{ListParsers} is implemented by class \code{ParseString}.

To run the parser, we apply the start symbol of the grammar
\code{expr} the argument code{input} and observe the result:
\begin{lstlisting}
    ps.expr(ps.input) match {
      case Some(n) =>
        println("parsed: " + args(0).substring(0, n))
      case None =>
        println("nothing parsed")
    }
  }
}// end Test
\end{lstlisting}
Note the syntax ~\code{ps.expr(input)}, which treats the \code{expr}
parser as if it was a function. In Scala, objects with \code{apply}
methods can be applied directly to arguments as if they were functions.

Here is an example run of the program above:
\begin{lstlisting}
> java examples.Test "(x,1,(y,z))"
parsed: (x,1,(y,z))
> java examples.Test "(x,,1,(y,z))"
nothing parsed
\end{lstlisting}

\section{\label{sec:parsers-results}Parsers that Produce Results}

The combinator library of the previous section does not support the
generation of output from parsing. But usually one does not just want
to check whether a given string belongs to the defined language, one
also wants to convert the input string into some internal
representation such as an abstract syntax tree.

In this section, we modify our parser library to build parsers that
produce results. We will make use of the for-comprehensions introduced
in Chapter~\ref{sec:for-notation}.  The basic combinator of sequential
composition, formerly ~\code{p &&& q}, now becomes
\begin{lstlisting}
for (x <- p; y <- q) yield e .
\end{lstlisting}
Here, the names \code{x} and \code{y} are bound to the results of
executing the parsers \code{p} and \code{q}. \code{e} is an expression
that uses these results to build the tree returned by the composed
parser.

Before describing the implementation of the new parser combinators, we
explain how the new building blocks are used. Say we want to modify
our list parser so that it returns an abstract syntax tree of the
parsed expression. Syntax trees are given by the following class hierarchy:
\begin{lstlisting}
abstract class Tree
case class Id (s: String)         extends Tree
case class Num(n: Int)            extends Tree
case class Lst(elems: List[Tree]) extends Tree
\end{lstlisting}
That is, a syntax tree is an identifier, an integer number, or a
\code{Lst} node with a list of trees as descendants.

As a first step towards parsers that produce results we define three
little parsers that return a single read character as result.
\begin{lstlisting}
trait CharParsers extends Parsers {
  def any: Parser[Char]
  def chr(ch: Char): Parser[Char] =
    for (c <- any if c == ch) yield c
  def chrSuchThat(p: Char => Boolean): Parser[Char] =
    for (c <- any if p(c)) yield c
}
\end{lstlisting}
The \code{any} parser succeeds with the first character of remaining
input as long as input is nonempty. It is abstract in class
\code{ListParsers} since we want to abstract in this class from the
concrete input method used.  The two \code{chr} parsers return as before
the first input character if it equals a given character or matches a
given predicate. They are now implemented in terms of \code{any}.

The next level is represented by parsers reading identifiers, numbers
and lists. Here is a parser for identifiers.
\begin{lstlisting}
trait ListParsers extends CharParsers {
  def ident: Parser[Tree] = 
    for {
      c: Char <- chrSuchThat(Character.isLetter)
      cs: List[Char] <- rep(chrSuchThat(Character.isLetterOrDigit))
    } yield Id((c :: cs).mkString("", "", ""))
\end{lstlisting}
Remark: Because \code{chrSuchThat(...)} returns a single character, its
repetition \code{rep(chrSuchThat(...))} returns a list of characters. The
\code{yield} part of the for-comprehension converts all intermediate
results into an \code{Id} node with a string as element.  To convert
the read characters into a string, it conses them into a single list,
and invokes the \code{mkString} method on the result.

Here is a parser for numbers:
\begin{lstlisting}
  def number: Parser[Tree] =
    for {
      d: Char <- chrSuchThat(Character.isDigit)
      ds: List[Char] <- rep(chrSuchThat(Character.isDigit))
    } yield Num(((d - '0') /: ds) ((x, digit) => x * 10 + digit - '0'))
\end{lstlisting}
Intermediate results are in this case the leading digit of
the read number, followed by a list of remaining digits.  The
\code{yield} part of the for-comprehension reduces these to a number
by a fold-left operation.

Here is a parser for lists:
\begin{lstlisting}
  def list: Parser[Tree] = 
    for {
      _ <- chr('(')
      es <- listElems ||| succeed(List())
      _ <- chr(')')
    } yield Lst(es)

  def listElems: Parser[List[Tree]] = 
    for {
      x <- expr
      xs <- chr(',') &&& listElems ||| succeed(List())
    } yield x :: xs
\end{lstlisting}
The \code{list} parser returns a \code{Lst} node with a list of trees
as elements.  That list is either the result of \code{listElems}, or,
if that fails, the empty list (expressed here as: the result of a
parser which always succeeds with the empty list as result).

The highest level of our grammar is represented by function
\code{expr}:
\begin{lstlisting}
  def expr: Parser[Tree] = 
    ident ||| number ||| list
}// end ListParsers.
\end{lstlisting}
We now present the parser combinators that support the new
scheme. Parsers that succeed now return a parse result besides the
un-consumed input.
\begin{lstlisting}
trait Parsers {
  type InType
  abstract class Parser[A] {
    type Result = Option[(A, InType)]
    def apply(in: InType): Result
\end{lstlisting}
Parsers are parameterized with the type of their result. The class
\code{Parser[a]} now defines new methods \code{map}, \code{flatMap}
and \code{filter}. The \code{for} expressions are mapped by the
compiler to calls of these functions using the scheme described in
Chapter~\ref{sec:for-notation}. For parsers, these methods are
implemented as follows.
\begin{lstlisting}
    def filter(pred: A => Boolean) = new Parser[A] {
      def apply(in: InType): Result = Parser.this.apply(in) match {
        case None => None
        case Some(x, in1) => if (pred(x)) Some(x, in1) else None
      }
    }
    def map[B](f: A => B) = new Parser[B] {
      def apply(in: InType): Result = Parser.this.apply(in) match {
        case None => None
        case Some(x, in1) => Some(f(x), in1)
      }
    }
    def flatMap[b](f: A => Parser[B]) = new Parser[B] {
      def apply(in: InType): Result = Parser.this.apply(in) match {
        case None => None
        case Some(x, in1) => f(x).apply(in1)
      }
    }
\end{lstlisting}
The \code{filter} method takes as parameter a predicate $p$ which it
applies to the results of the current parser. If the predicate is
false, the parser fails by returning \code{None}; otherwise it returns
the result of the current parser.  The \code{map} method takes as
parameter a function $f$ which it applies to the results of the
current parser. The \code{flatMap} takes as parameter a function
\code{f} which returns a parser.  It applies \code{f} to the result of
the current parser and then continues with the resulting parser.  The
\code{|||} method is essentially defined as before.  The
\code{&&&} method can now be defined in terms of \code{for}.
\begin{lstlisting}
    def ||| (p: => Parser[A]) = new Parser[A] {
      def apply(in: InType): Result = Parser.this.apply(in) match {
        case None => p(in)
        case s => s
      }
    }

    def &&& [B](p: => Parser[B]): Parser[B] =
      for (_ <- this; x <- p) yield x
  }// end Parser
\end{lstlisting}

The primitive parser \code{succeed} replaces \code{empty}. It consumes
no input and returns its parameter as result.
\begin{lstlisting}
  def succeed[A](x: A) = new Parser[A] {
    def apply(in: InType) = Some(x, in)
  }
\end{lstlisting}

The parser combinators \code{rep} and \code{opt} now also return
results. \code{rep} returns a list which contains as elements the
results of each iteration of its sub-parser. \code{opt} returns a list
which is either empty or returns as single element the result of the
optional parser.
\begin{lstlisting}
  def rep[A](p: Parser[A]): Parser[List[A]] =
    rep1(p) ||| succeed(List())

  def rep1[A](p: Parser[A]): Parser[List[A]] =
    for (x <- p; xs <- rep(p)) yield x :: xs

  def opt[A](p: Parser[A]): Parser[List[A]] =
    (for (x <- p) yield List(x)) ||| succeed(List())
} // end Parsers
\end{lstlisting}
The root class \code{Parsers} abstracts over which kind of
input is parsed.  As before, we determine the input method by a separate class.
Here is \code{ParseString}, this time adapted to parsers that return results.
It defines now the method \code{any}, which returns the first input character.
\begin{lstlisting}
class ParseString(s: String) extends Parsers {
  type InType = Int
  val input = 0
  def any = new Parser[Char] {
    def apply(in: Int): Parser[Char]#Result =
      if (in < s.length()) Some(s charAt in, in + 1) else None
  }
}
\end{lstlisting}
The rest of the application is as before. Here is a test program which
constructs a list parser over strings and prints out the result of
applying it to the command line argument.
\begin{lstlisting}
object Test {
  def main(args: Array[String]) {
    val ps = new ParseString(args(0)) with ListParsers
    ps.expr(ps.input) match {
      case Some(list, _) => println("parsed: " + list)
      case None => println("nothing parsed")
    }
  }
}
\end{lstlisting}

\begin{exercise}\label{exercise:end-marker} The parsers we have defined so
far can succeed even if there is some input beyond the parsed text. To
prevent this, one needs a parser which recognizes the end of input.
Redesign the parser library so that such a parser can be introduced.
Which classes need to be modified?
\end{exercise}
}

\chapter{\label{sec:hm}Hindley/Milner Type Inference}

This chapter demonstrates Scala's data types and pattern matching by
developing a type inference system in the Hindley/Milner style
\cite{milner:polymorphism}. The source language for the type inferencer is
lambda calculus with a let construct called Mini-ML. Abstract syntax
trees for the Mini-ML are represented by the following data type of
\code{Terms}.
\begin{lstlisting}
abstract class Term {}
case class Var(x: String) extends Term {
  override def toString = x
}
case class Lam(x: String, e: Term) extends Term {
  override def toString = "(\\" + x + "." + e + ")"
}
case class App(f: Term, e: Term) extends Term {
  override def toString = "(" + f + " " + e + ")"
}
case class Let(x: String, e: Term, f: Term) extends Term {
  override def toString = "let " + x + " = " + e + " in " + f
}
\end{lstlisting}
There are four tree constructors: \code{Var} for variables, \code{Lam}
for function abstractions, \code{App} for function applications, and
\code{Let} for let expressions. Each case class overrides the
\code{toString} method of class \code{Any}, so that terms can be
printed in legible form.

We next define the types that are
computed by the inference system.
\begin{lstlisting}
sealed abstract class Type {}
case class Tyvar(a: String) extends Type {
  override def toString = a
}
case class Arrow(t1: Type, t2: Type) extends Type {
  override def toString = "(" + t1 + "->" + t2 + ")"
}
case class Tycon(k: String, ts: List[Type]) extends Type {
  override def toString = 
    k + (if (ts.isEmpty) "" else ts.mkString("[", ",", "]"))
}
\end{lstlisting}
There are three type constructors: \code{Tyvar} for type variables,
\code{Arrow} for function types and \code{Tycon} for type constructors
such as \code{Boolean} or \code{List}. Type constructors have as
component a list of their type parameters. This list is empty for type
constants such as \code{Boolean}. Again, the type constructors
implement the \code{toString} method in order to display types legibly.

Note that \code{Type} is a \code{sealed} class. This means that no
subclasses or data constructors that extend \code{Type} can be formed
outside the sequence of definitions in which \code{Type} is defined.
This makes \code{Type} a {\em closed} algebraic data type with exactly
three alternatives. By contrast, type \code{Term} is an {\em open}
algebraic type for which further alternatives can be defined.

The main parts of the type inferencer are contained in object
\code{typeInfer}. We start with a utility function which creates
fresh type variables:
\begin{lstlisting}
object typeInfer {
  private var n: Int = 0
  def newTyvar(): Type = { n += 1; Tyvar("a" + n) }
\end{lstlisting}
We next define a class for substitutions. A substitution is an
idempotent function from type variables to types. It maps a finite
number of type variables to some types, and leaves all other type
variables unchanged. The meaning of a substitution is extended
point-wise to a mapping from types to types.
\begin{lstlisting}
  abstract class Subst extends Function1[Type,Type] {

    def lookup(x: Tyvar): Type

    def apply(t: Type): Type = t match {
      case tv @ Tyvar(a) => val u = lookup(tv); if (t == u) t else apply(u)
      case Arrow(t1, t2) => Arrow(apply(t1), apply(t2))
      case Tycon(k, ts) => Tycon(k, ts map apply)
    }

    def extend(x: Tyvar, t: Type) = new Subst {
      def lookup(y: Tyvar): Type = if (x == y) t else Subst.this.lookup(y)
    }
  }
  val emptySubst = new Subst { def lookup(t: Tyvar): Type = t }
\end{lstlisting}
We represent substitutions as functions, of type \code{Type =>
Type}. This is achieved by making class \code{Subst} inherit from the
unary function type \code{Function1[Type, Type]}\footnote{
The class inherits the function type as a mixin rather than as a direct 
superclass. This is because in the current Scala implementation, the
\code{Function1} type is a Java interface, which cannot be used as a direct
superclass of some other class.}.
To be an instance
of this type, a substitution \code{s} has to implement an \code{apply}
method that takes a \code{Type} as argument and yields another
\code{Type} as result. A function application \code{s(t)} is then
interpreted as \code{s.apply(t)}.

The \code{lookup} method is abstract in class \code{Subst}.  There are
two concrete forms of substitutions which differ in how they
implement this method.  One form is defined by the \code{emptySubst} value,
the other is defined by the \code{extend} method in class
\code{Subst}.

The next data type describes type schemes, which consist of a type and
a list of names of type variables which appear universally quantified
in the type scheme. 
For instance, the type scheme $\forall a\forall b.a \!\arrow\! b$ would be represented in the type checker as:
\begin{lstlisting}
TypeScheme(List(Tyvar("a"), Tyvar("b")), Arrow(Tyvar("a"), Tyvar("b"))) .
\end{lstlisting}
The class definition of type schemes does not carry an extends
clause; this means that type schemes extend directly class
\code{AnyRef}.  Even though there is only one possible way to
construct a type scheme, a case class representation was chosen
since it offers convenient ways to decompose an instance of this type into its
parts.
\begin{lstlisting}
  case class TypeScheme(tyvars: List[Tyvar], tpe: Type) {
    def newInstance: Type = {
      (emptySubst /: tyvars) ((s, tv) => s.extend(tv, newTyvar())) (tpe)
    }
  }
\end{lstlisting}
Type scheme objects come with a method \code{newInstance}, which
returns the type contained in the scheme after all universally type
variables have been renamed to fresh variables. The implementation of
this method folds (with \code{/:}) the type scheme's type variables
with an operation which extends a given substitution \code{s} by
renaming a given type variable \code{tv} to a fresh type
variable. The resulting substitution renames all type variables of the
scheme to fresh ones. This substitution is then applied to the type
part of the type scheme.

The last type we need in the type inferencer is
\code{Env}, a type for environments, which associate variable names
with type schemes. They are represented by a type alias \code{Env} in
module \code{typeInfer}:
\begin{lstlisting}
  type Env = List[(String, TypeScheme)]
\end{lstlisting}
There are two operations on environments. The \code{lookup} function
returns the type scheme associated with a given name, or \code{null}
if the name is not recorded in the environment.
\begin{lstlisting}
  def lookup(env: Env, x: String): TypeScheme = env match {
    case List() => null
    case (y, t) :: env1 => if (x == y) t else lookup(env1, x)
  }
\end{lstlisting}
The \code{gen} function turns a given type into a type scheme,
quantifying over all type variables that are free in the type, but
not in the environment.
\begin{lstlisting}
  def gen(env: Env, t: Type): TypeScheme = 
    TypeScheme(tyvars(t) diff tyvars(env), t)
\end{lstlisting}
The set of free type variables of a type is simply the set of all type
variables which occur in the type. It is represented here as a list of
type variables, which is constructed as follows.
\begin{lstlisting}
  def tyvars(t: Type): List[Tyvar] = t match {
    case tv @ Tyvar(a) => 
      List(tv)
    case Arrow(t1, t2) => 
      tyvars(t1) union tyvars(t2)
    case Tycon(k, ts) => 
      (List[Tyvar]() /: ts) ((tvs, t) => tvs union tyvars(t))
  }
\end{lstlisting}
Note that the syntax \code{tv @ ...} in the first pattern introduces a variable
which is bound to the pattern that follows. Note also that the explicit type parameter \code{[Tyvar]} in the expression of the third
clause is needed to make local type inference work.

The set of free type variables of a type scheme is the set of free
type variables of its type component, excluding any quantified type variables:
\begin{lstlisting}
  def tyvars(ts: TypeScheme): List[Tyvar] = 
    tyvars(ts.tpe) diff ts.tyvars
\end{lstlisting}
Finally, the set of free type variables of an environment is the union
of the free type variables of all type schemes recorded in it.
\begin{lstlisting}
  def tyvars(env: Env): List[Tyvar] =
    (List[Tyvar]() /: env) ((tvs, nt) => tvs union tyvars(nt._2))
\end{lstlisting}
A central operation of Hindley/Milner type checking is unification,
which computes a substitution to make two given types equal (such a
substitution is called a {\em unifier}).  Function \code{mgu} computes
the most general unifier of two given types $t$ and $u$ under a
pre-existing substitution $s$.  That is, it returns the most general
substitution $s'$ which extends $s$, and which makes $s'(t)$ and
$s'(u)$ equal types. 
\begin{lstlisting}
  def mgu(t: Type, u: Type, s: Subst): Subst = (s(t), s(u)) match {
    case (Tyvar(a), Tyvar(b)) if (a == b) =>
      s
    case (Tyvar(a), _) if !(tyvars(u) contains a) =>
      s.extend(Tyvar(a), u)
    case (_, Tyvar(a)) =>
      mgu(u, t, s)
    case (Arrow(t1, t2), Arrow(u1, u2)) =>
      mgu(t1, u1, mgu(t2, u2, s))
    case (Tycon(k1, ts), Tycon(k2, us)) if (k1 == k2) =>
      (s /: (ts zip us)) ((s, tu) => mgu(tu._1, tu._2, s))
    case _ => 
      throw new TypeError("cannot unify " + s(t) + " with " + s(u))
  }
\end{lstlisting}
The \code{mgu} function throws a \code{TypeError} exception if no
unifier substitution exists. This can happen because the two types
have different type constructors at corresponding places, or because a
type variable is unified with a type that contains the type variable
itself. Such exceptions are modeled here as instances of case classes
that inherit from the predefined \code{Exception} class.
\begin{lstlisting}
  case class TypeError(s: String) extends Exception(s) {}
\end{lstlisting}
The main task of the type checker is implemented by function
\code{tp}. This function takes as parameters an environment $env$, a
term $e$, a proto-type $t$, and a
pre-existing substitution $s$.  The function yields a substitution
$s'$ that extends $s$ and that
turns $s'(env) \ts e: s'(t)$ into a derivable type judgment according
to the derivation rules of the Hindley/Milner type system \cite{milner:polymorphism}.  A
\code{TypeError} exception is thrown if no such substitution exists.
\begin{lstlisting}
  def tp(env: Env, e: Term, t: Type, s: Subst): Subst = {
    current = e
    e match {
      case Var(x) =>
        val u = lookup(env, x)
        if (u == null) throw new TypeError("undefined: " + x)
        else mgu(u.newInstance, t, s)

      case Lam(x, e1) =>
        val a, b = newTyvar()
        val s1 = mgu(t, Arrow(a, b), s)
        val env1 = {x, TypeScheme(List(), a)} :: env
        tp(env1, e1, b, s1)

      case App(e1, e2) =>
        val a = newTyvar()
        val s1 = tp(env, e1, Arrow(a, t), s)
        tp(env, e2, a, s1)

      case Let(x, e1, e2) =>
        val a = newTyvar()
        val s1 = tp(env, e1, a, s)
        tp({x, gen(env, s1(a))} :: env, e2, t, s1)
    }
  } 
  var current: Term = null
\end{lstlisting}
To aid error diagnostics, the \code{tp} function stores the currently
analyzed sub-term in variable \code{current}. Thus, if type checking
is aborted with a \code{TypeError} exception, this variable will
contain the subterm that caused the problem.

The last function of the type inference module, \code{typeOf}, is a
simplified facade for \code{tp}. It computes the type of a given term
$e$ in a given environment $env$. It does so by creating a fresh type
variable $a$, computing a typing substitution that makes $env \ts e: a$
into a derivable type judgment, and returning
the result of applying the substitution to $a$.
\begin{lstlisting}
  def typeOf(env: Env, e: Term): Type = {
    val a = newTyvar()
    tp(env, e, a, emptySubst)(a)
  }
}// end typeInfer
\end{lstlisting}
To apply the type inferencer, it is convenient to have a predefined
environment that contains bindings for commonly used constants. The
module \code{predefined} defines an environment \code{env} that
contains bindings for the types of booleans, numbers and lists
together with some primitive operations over them. It also
defines a fixed point operator \code{fix}, which can be used to
represent recursion.
\begin{lstlisting}
object predefined {
  val booleanType = Tycon("Boolean", List())
  val intType = Tycon("Int", List())
  def listType(t: Type) = Tycon("List", List(t))

  private def gen(t: Type): typeInfer.TypeScheme = typeInfer.gen(List(), t)
  private val a = typeInfer.newTyvar()
  val env = List(
    {"true", gen(booleanType)},
    {"false", gen(booleanType)},
    {"if", gen(Arrow(booleanType, Arrow(a, Arrow(a, a))))},
    {"zero", gen(intType)},
    {"succ", gen(Arrow(intType, intType))},
    {"nil", gen(listType(a))},
    {"cons", gen(Arrow(a, Arrow(listType(a), listType(a))))},
    {"isEmpty", gen(Arrow(listType(a), booleanType))},
    {"head", gen(Arrow(listType(a), a))},
    {"tail", gen(Arrow(listType(a), listType(a)))},
    {"fix", gen(Arrow(Arrow(a, a), a))}
  )
}
\end{lstlisting}
Here's an example how the type inferencer can be used.
Let's define a function \code{showType} which returns the type of
a given term computed in the predefined environment
\code{Predefined.env}:
\begin{lstlisting}
object testInfer {
  def showType(e: Term): String =
    try {
      typeInfer.typeOf(predefined.env, e).toString
    } catch {
      case typeInfer.TypeError(msg) => 
        "\n cannot type: " + typeInfer.current +
        "\n reason: " + msg
    }
\end{lstlisting}
Then the application
\begin{lstlisting}
> testInfer.showType(Lam("x", App(App(Var("cons"), Var("x")), Var("nil"))))
\end{lstlisting}
would give the response
\begin{lstlisting}
> (a6->List[a6])
\end{lstlisting}
\comment{
To make the type inferencer more useful, we complete it with a
parser. 
Function \code{main} of module \code{testInfer}
parses and typechecks a Mini-ML expression which is given as the first
command line argument.
\begin{lstlisting}
  def main(args: Array[String]) {
    val ps = new ParseString(args(0)) with MiniMLParsers
    ps.all(ps.input) match {
      case Some(term, _) =>
        println("" + term + ": " + showType(term))
      case None =>
        println("syntax error")
    }
  }
%}// testInfer
\end{lstlisting}
To do the parsing, method \code{main} uses the combinator parser
scheme of Chapter~\ref{sec:combinator-parsing}. It creates a parser
family \code{ps} as a mixin composition of parsers
that understand MiniML (but do not know where input comes from) and
parsers that read input from a given string.  The \code{MiniMLParsers}
object implements parsers for the following grammar.
\begin{lstlisting}
term  ::= "\" ident "." term
       |  term1 {term1}
       |  "let" ident "=" term "in" term
term1 ::= ident
       |  "(" term ")"
all   ::= term ";"
\end{lstlisting}
Input as a whole is described by the production \code{all}; it
consists of a term followed by a semicolon. We allow ``whitespace''
consisting of one or more space, tabulator or newline characters
between any two lexemes (this is not reflected in the grammar
above). Identifiers are defined as in
Chapter~\ref{sec:combinator-parsing} except that an identifier cannot
be one of the two reserved words "let" and "in".
\begin{lstlisting}
trait MiniMLParsers extends CharParsers {

  /** whitespace */
  def whitespace = rep{chr(' ') ||| chr('\t') ||| chr('\n')}

  /** A given character, possible preceded by whitespace */
  def wschr(ch: Char) = whitespace &&& chr(ch)

  /** identifiers or keywords */
  def id: Parser[String] = 
    for {
      c: Char <- whitespace &&& chrSuchThat(Character.isLetter)
      cs: List[Char] <- rep(chrSuchThat(Character.isLetterOrDigit))
    } yield (c :: cs).mkString("", "", "")

  /** Non-keyword identifiers */
  def ident: Parser[String] =
    for { s <- id if s != "let" && s != "in" } yield s

  /** term = '\' ident '.' term | term1 {term1} | let ident "=" term in term */
  def term: Parser[Term] = (
    ( for {
        _ <- wschr('\\')
        x <- ident
        _ <- wschr('.')
        t <- term
      } yield Lam(x, t): Term )
    |||
    ( for {
        letid <- id if letid == "let"
        x <- ident
        _ <- wschr('=')
        t <- term; 
        inid <- id; inid == "in"
        c <- term
      } yield Let(x, t, c) )
    |||
    ( for {
        t <- term1
        ts <- rep(term1)
      } yield (t /: ts)((f, arg) => App(f, arg)) )
  )     

  /** term1 = ident | '(' term ')' */
  def term1: Parser[Term] = (
    ( for { s <- ident }
      yield Var(s): Term )
    |||
    ( for {
        _ <- wschr('(')
        t <- term
        _ <- wschr(')')
      } yield t )
  )

  /** all = term ';' */
  def all: Parser[Term] = 
    for {
      t <- term
      _ <- wschr(';')
    } yield t
}
\end{lstlisting}
Here are some sample MiniML programs and the output the type inferencer gives for each of them:
\begin{lstlisting}
> java testInfer
| "\x.\f.f(f x);"
(\x.(\f.(f (f x)))): (a8->((a8->a8)->a8))

> java testInfer 
| "let id = \x.x  
|  in if (id true) (id nil) (id (cons zero nil));"
let id = (\x.x) in (((if (id true)) (id nil)) (id ((cons zero) nil))): List[Int]

> java testInfer
| "let id = \x.x 
|  in if (id true) (id nil);"
let id = (\x.x) in ((if (id true)) (id nil)): (List[a13]->List[a13])

> java testInfer
| "let length = fix (\len.\xs.
|    if (isEmpty xs) 
|      zero 
|      (succ (len (tail xs))))
|  in (length nil);"
let length = (fix (\len.(\xs.(((if (isEmpty xs)) zero) 
(succ (len (tail xs))))))) in (length nil): Int

> java testInfer 
| "let id = \x.x 
|  in if (id true) (id nil) zero;"
let id = (\x.x) in (((if (id true)) (id nil)) zero): 
 cannot type: zero
 reason: cannot unify Int with List[a14]
\end{lstlisting}

\begin{exercise}\label{exercise:hm-parse} Using the parser library constructed in
Exercise~\ref{exercise:end-marker}, modify the MiniML parser library
so that no marker ``;'' is necessary for indicating the end of input.
\end{exercise}
}

\begin{exercise}\label{execcise:hm-extend} Extend the Mini-ML \comment{parser and} type
inferencer with a \code{letrec} construct which allows the definition of
recursive functions. Syntax:
\begin{lstlisting}
letrec ident "=" term in term .
\end{lstlisting}
The typing of \code{letrec} is as for \code{let}, 
except that the defined identifier is visible in the defining expression. Using \code{letrec}, the \code{length} function for lists can now be defined as follows.
\begin{lstlisting}
letrec length = \xs.
  if (isEmpty xs)
    zero
    (succ (length (tail xs)))
in ...
\end{lstlisting}
\end{exercise}

\chapter{Abstractions for Concurrency}\label{sec:ex-concurrency}

This section reviews common concurrent programming patterns and shows
how they can be implemented in Scala.

\section{Signals and Monitors}

\example
The {\em monitor} provides the basic means for mutual exclusion
of processes in Scala. Every instance of class \code{AnyRef} can be
used as a monitor by calling one or more of the methods below.
\begin{lstlisting}
  def synchronized[A] (e: => A): A
  def wait()
  def wait(msec: Long)
  def notify()
  def notifyAll()
\end{lstlisting}
The \code{synchronized} method executes its argument computation
\code{e} in mutual exclusive mode -- at any one time, only one thread
can execute a \code{synchronized} argument of a given monitor.

Threads can suspend inside a monitor by waiting on a signal.  Threads
that call the \code{wait} method wait until a \code{notify} method of
the same object is called subsequently by some other thread. Calls to
\code{notify} with no threads waiting for the signal are ignored.

There is also a timed form of \code{wait}, which blocks only as long
as no signal was received or the specified amount of time (given in
milliseconds) has elapsed. Furthermore, there is a \code{notifyAll}
method which unblocks all threads which wait for the signal.  These
methods, as well as class \code{Monitor} are primitive in Scala; they
are implemented in terms of the underlying runtime system.

Typically, a thread waits for some condition to be established. If the
condition does not hold at the time of the wait call, the thread
blocks until some other thread has established the condition. It is
the responsibility of this other thread to wake up waiting processes
by issuing a \code{notify} or \code{notifyAll}. Note however, that
there is no guarantee that a waiting process gets to run immediately
after the call to notify is issued. It could be that other processes
get to run first which invalidate the condition again. Therefore, the
correct form of waiting for a condition $C$ uses a while loop:
\begin{lstlisting}
while (!$C$) wait()
\end{lstlisting}

As an example of how monitors are used, here is is an implementation
of a bounded buffer class.
\begin{lstlisting}
class BoundedBuffer[A](N: Int) {
  var in = 0, out = 0, n = 0
  val elems = new Array[A](N)

  def put(x: A) = synchronized {
    while (n >= N) wait()
    elems(in) = x ; in = (in + 1) % N ; n = n + 1
    if (n == 1) notifyAll()
  }

  def get: A = synchronized {
    while (n == 0) wait()
    val x = elems(out) ; out = (out + 1) % N ; n = n - 1
    if (n == N - 1) notifyAll()
    x
  }
}
\end{lstlisting}
And here is a program using a bounded buffer to communicate between a
producer and a consumer process.
\begin{lstlisting}
import scala.concurrent.ops._
...
val buf = new BoundedBuffer[String](10)
spawn { while (true) { val s = produceString ; buf.put(s) } }
spawn { while (true) { val s = buf.get ; consumeString(s) } }
}
\end{lstlisting}
The \code{spawn} method spawns a new thread which executes the
expression given in the parameter. It is defined in object \code{concurrent.ops}
as follows.
\begin{lstlisting}
def spawn(p: => Unit) {
  val t = new Thread() { override def run() = p }
  t.start()
}
\end{lstlisting}

\comment{
\section{Logic Variable}

A logic variable (or lvar for short) offers operations \code{:=}
and \code{value} to define the variable and to retrieve its value.
Variables can be \code{define}d only once. A call to \code{value}
blocks until the variable has been defined.

Logic variables can be implemented as follows.

\begin{lstlisting}
class LVar[A] {
  private val defined = new Signal
  private var isDefined: Boolean = false
  private var v: A
  def value = synchronized {
    if (!isDefined) defined.wait
    v
  }
  def :=(x: A) = synchronized {
    v = x; isDefined = true; defined.send
  }
}
\end{lstlisting}
}

\section{SyncVars}

A synchronized variable (or syncvar for short) offers \code{get} and
\code{put} operations to read and set the variable. \code{get} operations
block until the variable has been defined. An \code{unset} operation
resets the variable to undefined state.

Here's the standard implementation of synchronized variables.
\begin{lstlisting}
package scala.concurrent
class SyncVar[A] {
  private var isDefined: Boolean = false
  private var value: A = _
  def get = synchronized {
    while (!isDefined) wait()
    value
  }
  def set(x: A) = synchronized {
    value = x; isDefined = true; notifyAll()
  }
  def isSet: Boolean = synchronized {
    isDefined
  }
  def unset = synchronized {
    isDefined = false
  }
}
\end{lstlisting}

\section{Futures}
\label{sec:futures}

A {\em future} is a value which is computed in parallel to some other
client thread, to be used by the client thread at some future time.
Futures are used in order to make good use of parallel processing
resources.  A typical usage is:

\begin{lstlisting}
import scala.concurrent.ops._
...
val x = future(someLengthyComputation)
anotherLengthyComputation
val y = f(x()) + g(x())
\end{lstlisting}

The \code{future} method is defined in object
\code{scala.concurrent.ops} as follows.
\begin{lstlisting}
def future[A](p: => A): Unit => A = {
  val result = new SyncVar[A]
  fork { result.set(p) }
  (() => result.get)
}
\end{lstlisting}

The \code{future} method gets as parameter a computation \code{p} to
be performed. The type of the computation is arbitrary; it is
represented by \code{future}'s type parameter \code{a}.  The
\code{future} method defines a guard \code{result}, which takes a
parameter representing the result of the computation. It then forks
off a new thread that computes the result and invokes the
\code{result} guard when it is finished. In parallel to this thread,
the function returns an anonymous function of type \code{a}.
When called, this functions waits on the result guard to be
invoked, and, once this happens returns the result argument.
At the same time, the function reinvokes the \code{result} guard with
the same argument, so that future invocations of the function can
return the result immediately.

\section{Parallel Computations}

The next example presents a function \code{par} which takes a pair of
computations as parameters and which returns the results of the computations
in another pair. The two computations are performed in parallel.

The function is defined in object
\code{scala.concurrent.ops} as follows.
\begin{lstlisting}
  def par[A, B](xp: => A, yp: => B): (A, B) = {
    val y = new SyncVar[B]
    spawn { y set yp }
    (xp, y.get)
  }
\end{lstlisting}
Defined in the same place is a function \code{replicate} which performs a
number of replicates of a computation in parallel. Each
replication instance is passed an integer number which identifies it.
\begin{lstlisting}
  def replicate(start: Int, end: Int)(p: Int => Unit) {
    if (start == end)
      ()
    else if (start + 1 == end)
      p(start)
    else {
      val mid = (start + end) / 2
      spawn { replicate(start, mid)(p) }
      replicate(mid, end)(p)
    }
  }
\end{lstlisting}

The next function uses \code{replicate} to perform parallel
computations on all elements of an array.

\begin{lstlisting}
def parMap[A,B](f: A => B, xs: Array[A]): Array[B] = {
  val results = new Array[B](xs.length)
  replicate(0, xs.length) { i => results(i) = f(xs(i)) }
  results
}
\end{lstlisting}

\section{Semaphores}

A common mechanism for process synchronization is a {\em lock} (or:
{\em semaphore}). A lock offers two atomic actions: \prog{acquire} and
\prog{release}. Here's the implementation of a lock in Scala:

\begin{lstlisting}
package scala.concurrent

class Lock {
  var available = true
  def acquire = synchronized {
    while (!available) wait()
    available = false
  }
  def release = synchronized {
    available = true
    notify()
  }
}
\end{lstlisting}

\section{Readers/Writers}

A more complex form of synchronization distinguishes between {\em
readers} which access a common resource without modifying it and {\em
writers} which can both access and modify it. To synchronize readers
and writers we need to implement operations \prog{startRead}, \prog{startWrite},
\prog{endRead}, \prog{endWrite}, such that:
\begin{itemize}
\item there can be multiple concurrent readers,
\item there can only be one writer at one time,
\item pending write requests have priority over pending read requests,
but don't preempt ongoing read operations.
\end{itemize}
The following implementation of a readers/writers lock is based on the
{\em mailbox} concept (see Section~\ref{sec:mailbox}).

\begin{lstlisting}
import scala.concurrent._

class ReadersWriters {
  val m = new MailBox
  private case class Writers(n: Int), Readers(n: Int) { m send this }
  Writers(0); Readers(0)
  def startRead = m receive {
    case Writers(n) if n == 0 => m receive {
      case Readers(n) => Writers(0); Readers(n+1)
    }
  }
  def startWrite = m receive {
    case Writers(n) =>
      Writers(n+1)
      m receive { case Readers(n) if n == 0 => }
  }
  def endRead = m receive {
    case Readers(n) => Readers(n-1)
  }
  def endWrite = m receive {
    case Writers(n) => Writers(n-1); if (n == 0) Readers(0)
  }
}
\end{lstlisting}

\section{Asynchronous Channels}

A fundamental way of interprocess communication is the asynchronous
channel. Its implementation makes use the following simple class for linked
lists:
\begin{lstlisting}
class LinkedList[A] {
  var elem: A = _
  var next: LinkedList[A] = null
}
\end{lstlisting}
To facilitate insertion and deletion of elements into linked lists,
every reference into a linked list points to the node which precedes
the node which conceptually forms the top of the list.
Empty linked lists start with a dummy node, whose successor is \code{null}.

The channel class uses a linked list to store data that has been sent
but not read yet. At the opposite end, threads that
wish to read from an empty channel, register their presence by
incrementing the \code{nreaders} field and waiting to be notified.
\begin{lstlisting}
package scala.concurrent

class Channel[A] {
  class LinkedList[A] {
    var elem: A = _
    var next: LinkedList[A] = null
  }
  private var written = new LinkedList[A]
  private var lastWritten = written
  private var nreaders = 0

  def write(x: A) = synchronized {
    lastWritten.elem = x
    lastWritten.next = new LinkedList[A]
    lastWritten = lastWritten.next
    if (nreaders > 0) notify()
  }

  def read: A = synchronized {
    if (written.next == null) {
      nreaders = nreaders + 1; wait(); nreaders = nreaders - 1
    }
    val x = written.elem
    written = written.next
    x
  }
}
\end{lstlisting}

\section{Synchronous Channels}

Here's an implementation of synchronous channels, where the sender of
a message blocks until that message has been received. Synchronous
channels only need a single variable to store messages in transit, but
three signals are used to coordinate reader and writer processes.
\begin{lstlisting}
package scala.concurrent

class SyncChannel[A] {
  private var data: A = _
  private var reading = false
  private var writing = false

  def write(x: A) = synchronized {
    while (writing) wait()
    data = x
    writing = true
    if (reading) notifyAll()
    else while (!reading) wait()
  }

  def read: A = synchronized {
    while (reading) wait()
    reading = true
    while (!writing) wait()
    val x = data
    writing = false
    reading = false
    notifyAll()
    x
  }
}
\end{lstlisting}

\section{Workers}

Here's an implementation of a {\em compute server} in Scala. The
server implements a \code{future} method which evaluates a given
expression in parallel with its caller. Unlike the implementation in
Section~\ref{sec:futures} the server computes futures only with a
predefined number of threads. A possible implementation of the server
could run each thread on a separate processor, and could hence avoid
the overhead inherent in context-switching several threads on a single
processor.

\begin{lstlisting}
import scala.concurrent._, scala.concurrent.ops._

class ComputeServer(n: Int) {

  private abstract class Job {
    type T
    def task: T
    def ret(x: T)
  }

  private val openJobs = new Channel[Job]()

  private def processor(i: Int) {
    while (true) {
      val job = openJobs.read
      job.ret(job.task)
    }
  }

  def future[A](p: => A): () => A = {
    val reply = new SyncVar[A]()
    openJobs.write{
      new Job {
        type T = A
        def task = p
        def ret(x: A) = reply.set(x)
      }
    }
    () => reply.get
  }

  spawn(replicate(0, n) { processor })
}
\end{lstlisting}
Expressions to be computed (i.e. arguments
to calls of \code{future}) are written to the \code{openJobs}
channel. A {\em job} is an object with
\begin{itemize}
\item
An abstract type \code{t} which describes the result of the compute
job.
\item
A parameterless \code{task} method of type \code{t} which denotes
the expression to be computed.
\item
A \code{return} method which consumes the result once it is
computed.
\end{itemize}
The compute server creates $n$ \code{processor} processes as part of
its initialization.  Every such process repeatedly consumes an open
job, evaluates the job's \code{task} method and passes the result on
to the job's
\code{return} method. The polymorphic \code{future} method creates
a new job where the \code{return} method is implemented by a guard
named \code{reply} and inserts this job into the set of open jobs by
calling the \code{isOpen} guard. It then waits until the corresponding
\code{reply} guard is called.

The example demonstrates the use of abstract types. The abstract type
\code{t} keeps track of the result type of a job, which can vary
between different jobs. Without abstract types it would be impossible
to implement the same class to the user in a statically type-safe
way, without relying on dynamic type tests and type casts.


Here is some code which uses the compute server to evaluate 
the expression \code{41 + 1}.
\begin{lstlisting}
object Test with Executable {
  val server = new ComputeServer(1)
  val f = server.future(41 + 1)
  println(f())
}
\end{lstlisting}

\section{Mailboxes}
\label{sec:mailbox}

Mailboxes are high-level, flexible constructs for process
synchronization and communication. They allow sending and receiving of
messages. A {\em message} in this context is an arbitrary object.
There is a special message \code{TIMEOUT} which is used to signal a
time-out.
\begin{lstlisting}
case object TIMEOUT
\end{lstlisting}
Mailboxes implement the following signature.
\begin{lstlisting}
class MailBox {
  def send(msg: Any)
  def receive[A](f: PartialFunction[Any, A]): A
  def receiveWithin[A](msec: Long)(f: PartialFunction[Any, A]): A
}
\end{lstlisting}
The state of a mailbox consists of a multi-set of messages.
Messages are added to the mailbox with the \code{send} method. Messages
are removed using the \code{receive} method, which is passed a message
processor \code{f} as argument, which is a partial function from
messages to some arbitrary result type. Typically, this function is
implemented as a pattern matching expression. The \code{receive}
method blocks until there is a message in the mailbox for which its
message processor is defined.  The matching message is then removed
from the mailbox and the blocked thread is restarted by applying the
message processor to the message. Both sent messages and receivers are
ordered in time. A receiver $r$ is applied to a matching message $m$
only if there is no other $\{$message, receiver$\}$ pair which precedes ${m,
r}$ in the partial ordering on pairs that orders each component in
time.

As a simple example of how mailboxes are used, consider a
one-place buffer:
\begin{lstlisting}
class OnePlaceBuffer {
  private val m = new MailBox             // An internal mailbox
  private case class Empty, Full(x: Int)  // Types of messages we deal with
  m send Empty                            // Initialization
  def write(x: Int)
    { m receive { case Empty => m send Full(x) } }
  def read: Int =
    m receive { case Full(x) => m send Empty; x }
}
\end{lstlisting}
Here's how the mailbox class can be implemented:
\begin{lstlisting}
class MailBox {
  private abstract class Receiver extends Signal {
    def isDefined(msg: Any): Boolean
    var msg = null
  }
\end{lstlisting}
We define an internal class for receivers with a test method
\code{isDefined}, which indicates whether the receiver is
defined for a given message.  The receiver inherits from class
\code{Signal} a \code{notify} method which is used to wake up a
receiver thread. When the receiver thread is woken up, the message it
needs to be applied to is stored in the \code{msg} variable of
\code{Receiver}.
\begin{lstlisting}
  private val sent = new LinkedList[Any]
  private var lastSent = sent
  private val receivers = new LinkedList[Receiver]
  private var lastReceiver = receivers
\end{lstlisting}
The mailbox class maintains two linked lists,
one for sent but unconsumed messages, the other for waiting receivers.
\begin{lstlisting}
  def send(msg: Any) = synchronized {
    var r = receivers, r1 = r.next
    while (r1 != null && !r1.elem.isDefined(msg)) {
      r = r1; r1 = r1.next
    }
    if (r1 != null) {
      r.next = r1.next; r1.elem.msg = msg; r1.elem.notify
    } else {
      lastSent = insert(lastSent, msg)
    }
  }
\end{lstlisting}
The \code{send} method first checks whether a waiting receiver is
applicable to the sent message. If yes, the receiver is notified.
Otherwise, the message is appended to the linked list of sent messages.
\begin{lstlisting}
  def receive[A](f: PartialFunction[Any, A]): A = {
    val msg: Any = synchronized {
      var s = sent, s1 = s.next
      while (s1 != null && !f.isDefinedAt(s1.elem)) {
        s = s1; s1 = s1.next
      }
      if (s1 != null) {
        s.next = s1.next; s1.elem
      } else {
        val r = insert(lastReceiver, new Receiver {
          def isDefined(msg: Any) = f.isDefinedAt(msg)
        })
        lastReceiver = r
        r.elem.wait()
        r.elem.msg
      }
    }
    f(msg)
  }
\end{lstlisting}
The \code{receive} method first checks whether the message processor function
\code{f} can be applied to a message that has already been sent but that
was not yet consumed. If yes, the thread continues immediately by
applying \code{f} to the message. Otherwise, a new receiver is created
and linked into the \code{receivers} list, and the thread waits for a
notification on this receiver. Once the thread is woken up again, it
continues by applying \code{f} to the message that was stored in the
receiver. The insert method on linked lists is defined as follows.
\begin{lstlisting}
  def insert(l: LinkedList[A], x: A): LinkedList[A] = {
    l.next = new LinkedList[A]
    l.next.elem = x
    l.next.next = l.next
    l
  }
\end{lstlisting}
The mailbox class also offers a method \code{receiveWithin}
which blocks for only a specified maximal amount of time.  If no
message is received within the specified time interval (given in
milliseconds), the message processor argument $f$ will be unblocked
with the special \code{TIMEOUT} message.  The implementation of
\code{receiveWithin} is quite similar to \code{receive}:
\begin{lstlisting}
  def receiveWithin[A](msec: Long)(f: PartialFunction[Any, A]): A = {
    val msg: Any = synchronized {
      var s = sent, s1 = s.next
      while (s1 != null && !f.isDefinedAt(s1.elem)) {
        s = s1; s1 = s1.next 
      }
      if (s1 != null) {
        s.next = s1.next; s1.elem
      } else {
        val r = insert(lastReceiver, new Receiver {
            def isDefined(msg: Any) = f.isDefinedAt(msg)
        })
        lastReceiver = r
        r.elem.wait(msec)
        if (r.elem.msg == null) r.elem.msg = TIMEOUT
        r.elem.msg
      }
    }
    f(msg)
  }
} // end MailBox
\end{lstlisting}
The only differences are the timed call to \code{wait}, and the
statement following it.

\section{Actors}
\label{sec:actors}

Chapter~\ref{chap:example-auction} sketched as a program example the
implementation of an electronic auction service. This service was
based on high-level actor processes that work by inspecting messages
in their mailbox using pattern matching. A refined and optimized
implementation of actors is found in the \lstinline@scala.actors@
package.  We now give a sketch of a simplified version of the actors library.

The code below is different from the implementation in the \lstinline@scala.actors@
package, so it should be seen as an example how a simple version of
actors could be implemented.  It is not a description how actors are actually defined and
implemented in the standard Scala library. For the latter, please
refer to the Scala API documentation.

A simplified actor is just a thread whose communication primitives are those
of a mailbox.  Such an actor can be defined as a mixin composition
extension of Java's standard
\code{Thread} class with the \code{MailBox} class. We also override
the \code{run} method of the \code{Thread} class, so that it executes the
behavior of the actor that is defined by its \code{act} method.
The \code{!} method simply calls the \code{send} method of the
\code{MailBox} class:
\begin{lstlisting}
abstract class Actor extends Thread with MailBox {
  def act(): Unit
  override def run(): Unit = act()
  def !(msg: Any) = send(msg)
}
\end{lstlisting}

\comment{
As an extended example of an application that uses actors, we come
back to the auction server example of Section~\ref{sec:ex-auction}.
The following code implements:

\begin{figure}[thb]
\begin{lstlisting}
class AuctionMessage
case class
  Offer(bid: Int, client: Process),                  // make a bid
  Inquire(client: Process) extends AuctionMessage    // inquire status

class AuctionReply
case class
  Status(asked: Int, expiration: Date),           // asked sum, expiration date
  BestOffer,                                         // yours is the best offer
  BeatenOffer(maxBid: Int),                          // offer beaten by maxBid
  AuctionConcluded(seller: Process, client: Process),// auction concluded
  AuctionFailed                                      // failed with no bids
  AuctionOver extends AuctionReply                   // bidding is closed
\end{lstlisting}
\end{figure}

\begin{lstlisting}
class Auction(seller: Process, minBid: Int, closing: Date)
 extends Process {

  val timeToShutdown = 36000000 // msec
  val delta = 10                // bid increment
\end{lstlisting}
\begin{lstlisting}
  override def run = {
    var askedBid = minBid
    var maxBidder: Process = null
    while (true) {
      receiveWithin ((closing - Date.currentDate).msec) {
        case Offer(bid, client) => {
          if (bid >= askedBid) {
            if (maxBidder != null && maxBidder != client) {
              maxBidder send BeatenOffer(bid)
            }
            maxBidder = client
            askedBid = bid + delta
            client send BestOffer
          } else client send BeatenOffer(maxBid)
        }
\end{lstlisting}
\begin{lstlisting}
        case Inquire(client) => {
          client send Status(askedBid, closing)
        }
\end{lstlisting}
\begin{lstlisting}
        case TIMEOUT => {
          if (maxBidder != null) {
            val reply = AuctionConcluded(seller, maxBidder)
            maxBidder send reply
            seller send reply
          } else seller send AuctionFailed
          receiveWithin (timeToShutdown) {
            case Offer(_, client) => client send AuctionOver; discardAndContinue
            case _ => discardAndContinue
            case TIMEOUT => stop
          }
        }
\end{lstlisting}
\begin{lstlisting}
        case _ => discardAndContinue
      }
    }
  }
\end{lstlisting}
\begin{lstlisting}
  def houseKeeping: Int = {
    val Limit = 100
    var nWaiting: Int = 0
    receiveWithin(0) {
      case _ =>
        nWaiting = nWaiting + 1
        if (nWaiting > Limit) {
          receiveWithin(0) {
            case Offer(_, _) => continue
            case TIMEOUT =>
            case _ => discardAndContinue
          }
        } else continue
      case TIMEOUT =>
    }
  }
}
\end{lstlisting}
\begin{lstlisting}
class Bidder (auction: Process, minBid: Int, maxBid: Int)
 extends Process {
  val MaxTries = 3
  val Unknown = -1

  var nextBid = Unknown
\end{lstlisting}
\begin{lstlisting}
  def getAuctionStatus = {
    var nTries = 0
    while (nextBid == Unknown && nTries < MaxTries) {
      auction send Inquiry(this)
      nTries = nTries + 1
      receiveWithin(waitTime) {
        case Status(bid, _) => bid match {
          case None => nextBid = minBid
          case Some(curBid) => nextBid = curBid + Delta
        }
        case TIMEOUT =>
        case _ => continue
      }
    }
    status
  }
\end{lstlisting}
\begin{lstlisting}
  def bid {
    if (nextBid < maxBid) {
      auction send Offer(nextBid, this)
      receive {
        case BestOffer =>
          receive {
            case BeatenOffer(bestBid) =>
              nextBid = bestBid + Delta
              bid
            case AuctionConcluded(seller, client) =>
                   transferPayment(seller, nextBid)
            case _ => continue
          }

        case BeatenOffer(bestBid) =>
          nextBid = nextBid + Delta
          bid

        case AuctionOver =>

        case _ => continue
      }
    }
  }
\end{lstlisting}
\begin{lstlisting}
  override def run = {
    getAuctionStatus
    if (nextBid != Unknown) bid
  }

  def transferPayment(seller: Process, amount: Int)
}
\end{lstlisting}
}
