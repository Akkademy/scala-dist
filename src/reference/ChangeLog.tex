% $Id$

\chapter{Change Log}

\section*{Changes in Version 2.5.0}

\subsection*{Type constructor polymorphism\footnote{Implemented by Adriaan Moors}}
Type parameters (\sref{sec:type-params}) and abstract type members (\sref{sec:typedcl}) can now also abstract over type constructors (\sref{sec:higherkinded-types}).

This allows a more precise $Iterable$ interface:
\begin{lstlisting}
trait Iterable[+t] {
  type MyType[+t] <: Iterable[t] // MyType is a type constructor
                
  def filter(p: t => Boolean): MyType[t] = ...
  def map[s](f: t => s): MyType[s] = ...
}

abstract class List[+t] extends Iterable[t] {
  type MyType[+t] = List[t]
}
\end{lstlisting}

This definition of $Iterable$ makes explicit that mapping a function over a certain structure (e.g., a $List$) will yield the same structure (containing different elements).

\subsection*{Early object initialization}

It is now possible to initialize some fields of an object before any
parent constructors are called (\sref{sec:early-defs}). This is particularly useful for
traits, which do not have normal constructor parameters. Example:
\begin{lstlisting}
trait Greeting {
  val name: String
  val msg = "How are you, "+name
}
class C extends {
  val name = "Bob"
} with Greeting {
  println(msg)
}
\end{lstlisting}
In the code above, the field \code{name} is initialized before the
constructor of \code{Greeting} is called. Therefore, field \lstinline@msg@ in
class \code{Greeting} is properly initialized to \code{"How are you, Bob"}.

\subsection*{For-comprehensions, revised}

The syntax of for-comprehensions has been changed (\sref{sec:for-comprehensions}). For example:

\begin{lstlisting}
for (val x <- List(1, 2, 3); x % 2 == 0) println(x)
\end{lstlisting}

is now written

\begin{lstlisting}
for (x <- List(1, 2, 3) if x % 2 == 0) println(x)
\end{lstlisting}

The old syntax is still available but will be deprecated in the
future. 

\subsection*{Implicit anonymous functions}

It is now possible to define anonymous functions 
using underscores in parameter position (\sref{sec:impl-anon-fun}).
For instance, 
the expressions in the left column are each function values which 
expand to the anonymous functions on their right.
\begin{lstlisting}
_ + 1                  x => x + 1
_ * _                  (x1, x2) => x1 + x2
(_: int) * 2           (x: int) => (x: int) * 2
if (_) x else y        z => if (z) x else y
_.map(f)               x => x.map(f)
_.map(_ + 1)           x => x.map(y => y + 1)
\end{lstlisting}
As a special case (\sref{sec:meth-vals}), a partially unapplied method is now designated
~\lstinline@m _@~~ instead of the previous notation ~\lstinline@&m@.

The new notation will displace the special syntax forms
\lstinline@.m()@ for abstracting over method receivers and
\lstinline@&m@ for treating an unapplied method as a function value.
For the time being, the old syntax forms are still available, 
but they will be deprecated in the future.

\subsection*{Pattern matching anonymous functions, refined}

It is now possible to use case clauses to define a function value
directly for functions of arities greater than one
(\sref{sec:pattern-closures}).  Previously, only unary functions could
be defined that way. Example:
\begin{lstlisting}
def scalarProduct(xs: Array[Double], ys: Array[Double]) = 
  (0.0 /: (xs zip ys)) {
    case (a, (b, c)) => a + b * c
  }
\end{lstlisting}

\section*{Changes in Version 2.4.0}

\subsection*{Object-local private and protected}

The \lstinline@private@ and \lstinline@protected@ modifiers now accept
a \lstinline@[this]@ qualifier (\sref{sec:modifiers}). A definition $M$ which is labelled
\lstinline@private[this]@ is private, and in addition can be accessed
only from within the current object. That is, the only legal prefixes
for $M$ are \lstinline@this@ or \lstinline@$C$.this@.  Analogously, a
definition $M$ which is labelled \lstinline@protected[this]@ is
protected, and in addition can be accessed only from within the
current object.

\subsection*{Tuples, revised}

The syntax for tuples has been changed from $\{\ldots\}$ to $(\ldots)$ (\sref{sec:tuples}). For any
sequence of types $T_1 \commadots T_n$,

\begin{tabular}{lll}
$(T_1 \commadots T_n)$ &is a shorthand for&
\lstinline@Tuple$n$[$T_1 \commadots T_n$]@.  
\end{tabular}

Analogously, for any sequence of expressions or patterns $x_1
\commadots x_n$,

\begin{tabular}{lll}
$(x_1 \commadots x_n)$ &is a shorthand for&
\lstinline@Tuple$n$($x_1 \commadots x_n$)@.
\end{tabular}

\subsection*{Access modifiers for primary constructors}

The primary constructor of a class can now be marked \code{private} or
\code{protected} (\sref{sec:class-defs}). If such an access modifier is given, it comes
between the name of the class and its value parameters. Example:
\begin{lstlisting}
class C[T] private (x: T) { ... }
\end{lstlisting}

\subsection*{Annotations}

The support for attributes has been extended and its syntax changed
(\sref{sec:annotations}).  Attributes are now called {\em
  annotations}. The syntax has been changed to follow Java's
conventions, e.g. \lstinline^@attribute^ instead of
\lstinline@[attribute]@. The old syntax is still available but will be
deprecated in the future.

Annotations are now serialized so that they can be read by
compile-time or run-time tools. Class \code{scala.Annotation} has two
sub-traits which are used to indicate how annotations are
retained. Instances of an annotation class inheriting from trait
\code{scala.ClassfileAnnotation} will be stored in the generated class
files. Instances of an annotation class inheriting from trait
\code{scala.StaticAnnotation} will be visible to the Scala type-checker
in every compilation unit where the annotated symbol is accessed. 

\subsection*{Decidable subtyping}

The implementation of subtyping has been changed to prevent infinite
recursions. Termination of subtyping is now ensured by a new
restriction of class graphs to be finitary
(\sref{sec:inheritance-closure}).

\subsection*{Case classes cannot be abstract}

It is now explicitly ruled out that case classes can be abstract
(\sref{sec:modifiers}). The specification was silent on this point
before, but did not explain how abstract case classes were
treated. The Scala compiler allowed the idiom.

\subsection*{New syntax for self aliases and self types}

It is now possible to give an explicit alias name and/or type for the
self reference \code{this} (\sref{sec:templates}). For instance, in
\begin{lstlisting}
class C { self: D => 
  ...
}
\end{lstlisting}
the name \code{self} is introduced as an alias for \code{this} within
\code{C} and the self type (\sref{sec:class-defs}) of \code{C} is
assumed to be \code{D}. This construct is introduced now in order to
replace eventually both the qualified this construct \code{C.this} and
the \code{requires} clause in Scala.

\subsection*{Assignment Operators}

It is now possible to combine operators with assignments (\sref{sec:assops}). Example:
\begin{lstlisting}
var x: int = 0
x += 1
\end{lstlisting}

\section*{Changes in Version 2.3.2 (23-Jan-2007)}

\subsection*{Extractors}

It is now possible to define patterns independently of case classes,
using \code{unapply} methods in extractor objects
(\sref{sec:extractor-patterns}). 
Here is an example:
\begin{lstlisting}
object Twice {                              
  def apply(x:Int): int = x*2
  def unapply(z:Int): Option[int] = if (z%2==0) Some(z/2) else None
}
val x = Twice(21) 
x match { case Twice(n) => Console.println(n) } // prints 21
\end{lstlisting}
In the example, \lstinline@Twice@ is an extractor object with two methods:
\begin{itemize}
\item
The \code{apply} method is used to build even numbers.
\item
The \code{unapply} method is used to decompose an even number; it is
in a sense the reverse of \code{apply}. \lstinline@unapply@ methods return option types: 
\code{Some(...)} for a match that suceeds, \code{None} for a match that fails.
Pattern variables are returned as the elements of \code{Some}. If there are several
variables, they are grouped in a tuple.
\end{itemize}
In the second-to-last line, \code{Twice}'s \code{apply} method is used
to construct a number \code{x}. In the last line, \code{x} is tested
against the pattern
\code{Twice(n)}. This pattern succeeds for even numbers and assigns to the variable
\code{n} one half of the number that was tested. The pattern match makes use of
the \code{unapply} method of object \code{Twice}. More details on extractors can be found
in the paper ``Matching Objects with Patterns'' by Emir, Odersky and Williams.
 
\subsection*{Tuples}

A new lightweight syntax for tuples has been introduced
(\sref{sec:tuples}). For any sequence of types $T_1 \commadots T_n$,

\begin{tabular}{lll}
$\{T_1 \commadots T_n \}$ &is a shorthand for&
\lstinline@Tuple$n$[$T_1 \commadots T_n$]@.  
\end{tabular}

Analogously, for any sequence of expressions or patterns $x_1
\commadots x_n$,

\begin{tabular}{lll}
$\{x_1 \commadots x_n \}$ &is a shorthand for&
\lstinline@Tuple$n$($x_1 \commadots x_n$)@.
\end{tabular}

\subsection*{Infix operators of greater arities}

It is now possible to use methods which have more than one parameter
as infix operators (\sref{sec:infix-operations}). In this case, all
method arguments are written as a normal parameter list in parentheses. Example:
\begin{lstlisting}
class C {
  def +(x: int, y: String) = ...
}
val c = new C
c + (1, "abc")
\end{lstlisting}

\subsection*{Deprecated attribute}

A new standard attribute \lstinline@deprecated@ is available (\sref{sec:annotations}). If a
member definition is marked with this attribute, any reference to the
member will cause a ``deprecated'' warning message to be emitted.

\section*{Changes in Version 2.3.0 (23-Nov-2006)}

\subsection*{Procedures} A simplified syntax for functions returning
\lstinline@unit@ has been introduced (\sref{sec:procedures}). 
Scala now allows the following shorthands:

\begin{tabbing}
\lstinline@def f(params)@     \tab\tab\tab \=$\mbox{for}$  \tab
\lstinline@def f(params): unit@ \\
\lstinline@def f(params) { ... }@  \>$\mbox{for}$    \tab
\lstinline@def f(params): unit = { ... }@
\end{tabbing}

\subsection*{Type Patterns} The syntax of types in patterns has been 
refined (\sref{sec:type-patterns}). Scala now distinguishes between
type variables (starting with a lower case letter) and types as type
arguments in patterns.  Type variables are bound in the pattern. Other
type arguments are, as in previous versions, erased. The Scala
compiler will now issue an ``unchecked'' warning at places where type
erasure might compromise type-safety.

\subsection*{Standard Types} 

The recommended names for the two bottom classes in Scala's type
hierarchy have changed as follows:
\begin{lstlisting}
All      ==>     Nothing
AllRef   ==>     Null
\end{lstlisting}
The old names are still available as type aliases.

\section*{Changes in Version 2.1.8 (23-Aug-2006)}

\subsection*{Visibility Qualifier for protected}

Protected members can now have a visibility qualifier (\sref{sec:modifiers}),
e.g. \lstinline@protected[<qualifier>]@. In particular, one can now
simulate package protected access as in Java writing
\begin{lstlisting}
  protected[P] def X ...
\end{lstlisting}
where \code{P} would name the package containing \code{X}.

\subsection*{Relaxation of Private Acess}

Private members of a class can now be referenced from the companion
module of the class and vice versa (\sref{sec:modifiers})

\subsection*{Implicit Lookup}

The lookup method for implicit definitions has been generalized
(\sref{sec:impl-params}). 
When searching for an implicit definition matching a type $T$, now are
considered
\begin{enumerate}
\item all identifiers accessible without prefix, and
\item all members of companion modules of classes associated with $T$.
\end{enumerate}
(The second clause is more general than before). 
Here, a class is {\em associated} with a type $T$ if it is referenced
by some part of $T$, 
or if it is a base class of some part of $T$. 
For instance, to find implicit members corresponding to the type
\begin{lstlisting}
  HashSet[List[Int], String]
\end{lstlisting}
one would now look in the companion modules (aka static parts) of 
\code{HashSet}, \code{List}, \code{Int}, and \code{String}. 
Before, it was just the static part of \code{HashSet}.

\subsection*{Tightened Pattern Match}

A typed pattern match with a singleton type \code{p.type}
now tests whether the selector value is reference-equal to p
(\sref{sec:patterns}). 
Example:
\begin{lstlisting}
  val p = List(1, 2, 3)
  val q = List(1, 2)
  val r = q
  r match {
    case _: p.type => Console.println("p")
    case _: q.type => Console.println("q")
  }
\end{lstlisting}
This will match the second case and hence will print "\code{q}". 
Before, the singleton types were erased to \code{List}, and therefore
the first case 
would have matched, which is non-sensical.

\section*{Changes in Version 2.1.7 (19-Jul-2006)}

\subsection*{Multi-Line string literals} It is now possible to write
multi-line string-literals enclosed in triple quotes
(\sref{sec:string-literals}). 
Example:
\begin{lstlisting}
"""this is a
   multi-line
   string literal"""
\end{lstlisting}
No escape substitutions except for unicode escapes are performed in
such string literals.

\subsection*{Closure Syntax}

The syntax of closures has been slightly restricted (\sref{sec:closures}). The form
\begin{lstlisting}
  x: T => E
\end{lstlisting}
is valid only when enclosed in braces, i.e. ~\lstinline@{ x: T => E }@. 
The following is illegal, because it might be read as the value x typed with the type T => E:
\begin{lstlisting}
  val f = x: T => E
\end{lstlisting}
Legal alternatives are:
\begin{lstlisting}
  val f = { x: T => E }
  val f = (x: T) => E
\end{lstlisting}

\section*{Changes in Version 2.1.5 (24-May-2006)}

\subsection*{Class Literals} There is a new syntax for class literals
(\sref{sec:literal-exprs}): For any class type $C$,
\lstinline@classOf[$C$]@ designates the run-time representation of
$C$.

\section*{Changes in Version 2.0 (12-Mar-2006)}

Scala in its second version is different in some details from the
first version of the language. There have been several additions and
some old idioms are no longer supported. This appendix summarizes
the main changes.

\subsection*{New Keywords}

The following three words are now reserved; they cannot be used as
identifiers (\sref{sec:idents})
\begin{lstlisting}
implicit    match     requires
\end{lstlisting}

\subsection*{Newlines as Statement Separators}

Newlines can now be used as statement separators in place of
semicolons (\sref{sec:newlines})

\subsection*{Syntax Restrictions}

There are some other situations where old constructs no longer work:

\paragraph{\em Pattern matching expressions} The \lstinline@match@
keyword now appears only as infix operator between a selector
expression and a number of cases, as in:
\begin{lstlisting}
  expr match {
    case Some(x) => ...
    case None => ...
  }
\end{lstlisting}
Variants such as \lstinline@ expr.match {...} @ 
or just
\lstinline@ match {...} @
are no longer supported.

\paragraph{\em ``With'' in extends clauses}. The idiom
\begin{lstlisting}
class C with M { ... }
\end{lstlisting}
is no longer supported. A \lstinline@with@ connective is only allowed
following an \lstinline@extends@ clause. For instance, the line
above would have to be written
\begin{lstlisting}
class C extends AnyRef with M { ... } .
\end{lstlisting}
However, assuming \lstinline@M@ is a trait (see
\ref{sec:traits}), it is also legal to write
\begin{lstlisting}
class C extends M { ... }
\end{lstlisting}
The latter expression is treated as equivalent to
\begin{lstlisting}
class C extends S with M { ... }
\end{lstlisting}
where \lstinline@S@ is the superclass of \lstinline@M@.

\paragraph{\em Regular Expression Patterns} The only form of regular
expression pattern that is currently supported is a sequence pattern,
which might end in a sequence wildcard \code{_*}. Example:
\begin{lstlisting}
case List(1, 2, _*) => ... // will match all lists starting with \code{1,2}.
\end{lstlisting}
It is at current not clear whether this is a permanent restriction. We
are evaluating the possibility of re-introducing full regular
expression patterns in Scala.

\subsection*{Selftype Annotations}

The recommended syntax of selftype annotations has changed. 
\begin{lstlisting}
class C: T extends B { ... }
\end{lstlisting}
becomes
\begin{lstlisting}
class C requires T extends B { ... }
\end{lstlisting}
That is, selftypes are now indicated by the new \lstinline@requires@
keyword. The old syntax is still available but is considered deprecated. 
Conversions}

\subsection*{For-comprehensions}

For-comprehensions (\sref{sec:for-comprehensions}) now admit value and
pattern definitions. Example:
\begin{lstlisting}
for {
  val x <- List.range(1, 100)
  val y <- List.range(1, x)
  val z = x + y
  isPrime(z)
} yield Pair(x, y)
\end{lstlisting}
Note the definition ~\lstinline@val z = x + y@ as the third item in
the for-comprehension. 

\subsection*{Conversions}

The rules for implicit conversions of methods to functions
(\sref{sec:impl-conv}) have been tightened. Previously, a
parameterized method used as a value was always implicitly converted
to a function. This could lead to unexpected results when method
arguments where forgotten. Consider for instance the statement below:
\begin{lstlisting}
show(x.toString)
\end{lstlisting}
where \lstinline@show@ is defined as follows:
\begin{lstlisting}
def show(x: String) = Console.println(x) .
\end{lstlisting}
Most likely, the programmer forgot to supply an empty argument list
\lstinline@()@ to \lstinline@toString@. The previous Scala version would
treat this code as a partially applied method, and expand it to:
\begin{lstlisting}
show(() => x.toString())
\end{lstlisting}
As a result, the address of a closure would be printed instead of the
value of \lstinline@s@.

Scala version 2.0 will apply a conversion from partially applied
method to function value only if the expected type of the expression
is indeed a function type. For instance, the conversion would not be
applied in the code above because the expected type of
\lstinline@show@'s parameter is \lstinline@String@, not a function
type. 

The new convention disallows some previously legal code. Example:
\begin{lstlisting}
def sum(f: int => double)(a: int, b: int): double =
  if (a > b) 0 else f(a) + sum(f)(a + 1, b)

val sumInts  =  sum(x => x)  // error: missing arguments
\end{lstlisting}
The partial application of \lstinline@sum@ in the last line of
the code above will not be converted to a function type. Instead, the
compiler will produce an error message which states that arguments for method
\lstinline@sum@ are missing. The problem can be fixed by providing an
expected type for the partial application, for instance by annotating
the definition of \lstinline@sumInts@ with its type:
\begin{lstlisting}
val sumInts: (int, int) => double  =  sum(x => x)  // OK
\end{lstlisting}

On the other hand, Scala version 2.0 now automatically applies methods
with empty parameter lists to \lstinline@()@ argument lists when
necessary. For instance, the \lstinline@show@ expression above will
now be expanded to
\begin{lstlisting}
show(x.toString()) .
\end{lstlisting}

Scala version 2.0 also relaxes the rules of overriding with respect to
empty parameter lists. The revised definition of {\em matching
members} (\sref{sec:members}) makes it now possible to override a
method with an explicit, but empty parameter list \lstinline@()@ with
a parameterless method, and {\em vice versa}. For instance, 
the following class definition is now legal:
\begin{lstlisting}
class C {
  override def toString: String = ...
}
\end{lstlisting}
Previously this definition would have been rejected, because the
\lstinline@toString@ method as inherited from
\lstinline@java.lang.Object@ takes an empty parameter list.  

\subsection*{Class Parameters}

A class parameter may now be prefixed by \lstinline@val@ or
\lstinline@var@ (\sref{sec:class-defs}). 

\subsection*{Private Qualifiers}

Previously, Scala had three levels of visibility: {\em private},
{\em protected} and {\em public}. There was no way to
restrict accesses to members of the current package, as in Java. Scala
2 now defines access qualifiers that let one express this level of
visibility, among others. In the definition
\begin{lstlisting}
private[C] def f(...)
\end{lstlisting}
access to \lstinline@f@ is restricted to all code within the class or
package \lstinline@C@ (which must contain the definition of
\lstinline@f@) (\sref{sec:modifiers})

\subsection*{Changes in the Mixin Model}\label{sec:mixin-classes}

The model which details mixin composition of classes has changed
significantly. The main differences are:
\begin{enumerate}
\item
We now distinguish between {\em traits} that are used as mixin classes
and normal classes. The syntax of traits has been generalized from
version 1.0, in that traits are now allowed to have mutable
fields. However, as in version 1.0, traits may still do not have
constructor parameters.
\item
Member resolution and super accesses are now both defined in terms of
a {\em class linearization}. 
\item
Scala's notion of method overloading has been generalized; in
 particular, it is now possible to have overloaded variants of the
 same method in a subclass and in a superclass, or in several different
 mixins. This makes method overloading in Scala conceptually the
 same as in Java.
\end{enumerate}
The new mixin model is explained in more detail in
\sref{sec:globaldefs}.

\subsection*{Implicit Parameters}

Views in Scala 1.0 have been replaced by the more general concept of
implicit parameters (\sref{sec:implicits})

\subsection*{Flexible Typing of Pattern Matching}

The new version of Scala implements more flexible typing rules when it
comes to pattern matching over heterogeneous class hierarchies
(\sref{sec:pattern-match}). A {\em heterogeneous class hierarchy} is
one where subclasses inherit a common superclass with different
parameter types.  With the new rules in Scala version 2.0 one can
perform pattern matches over such hierarchies with more precise
typings that keep track of the information gained by comparing the
types of a selector and a matching pattern (\sref{ex:eval}).
This gives Scala capabilities analogous to guarded algebraic data types.

\end{document}

