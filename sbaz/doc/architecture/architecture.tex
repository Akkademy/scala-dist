\documentclass{article}
\usepackage{url}
\usepackage{times}

\newcommand{\ie}{\mbox{\emph{i.e.}}}	% \mbox keeps the last period from
\newcommand{\Ie}{\mbox{\emph{I.e.}}}	% looking like an end-of-sentence.
\newcommand{\eg}{\mbox{\emph{e.g.}}}
\newcommand{\Eg}{\mbox{\emph{E.g.}}}
\newcommand{\etal}{\emph{et al.}}

\newcommand{\hole}[1]{\emph{#1}}

\input{version}

\begin{document}
\bibliographystyle{plain}
\title{Package Universes Architecture\\
       {\svndate} \\
       (Revision: \svnversion)}
\author{Alexander Spoon (lex@lexspoon.org)}
\date{}
\maketitle

\section{Preamble}
This is a living document describing the package universes
architecture.  
The canonical web location for this document is currently the Scala
Bazaars web site:
\begin{quote}
  \url{http://www.lexspoon.org/sbaz/}
\end{quote}
Scala Bazaars is a running prototype used as infrastructure for the
Scala \cite{scala:spec, scala:web} open-source community.

Feedback should be directed either to the author or to the Scala
mailing list.


\section{Introduction}
Many long-term collaborations can be described in terms of developing
a shared library of content that is stored in discrete packages.  
%YYY it would be nice to find collaboration theory here
Examples include the Debian's Advanced Package Tool \cite{apt:howto} and
Squeak's SqueakMap for managing software packages, SuperSwikis for
managing Etoys content \cite{steinmetz02:learning}, and Wikis for
managing written articles \cite{ward01:wikiway}.  Frequently the
individual packages evolve over time: software gains new features and
bug fixes, media content is updated, and articles are edited.
Additionally, this content frequently includes dependencies.  A
software package for email reading might depend on an HTML
parser, and an active essay on fractals might require that
a 2D plotting library is loaded before the essay.

Processes and access control for such collaborations vary widely from
community to community.  Many communities want wide-open freedom to edit
and contribute.  Others, such as software development teams and
individual classrooms, allow complete freedom to members of the
community but no access to outsiders.  Still others have more
sophisticated policies where there are different groups allowed to
submit, commit, and access content in the shared space.  The precise
rules depend on the community.


This paper describes \emph{package universes}, an architecture for
supporting these package-distribution collaborations.  The
architecture is unusual for focussing on communal sets of packages
instead of on individual packages.  Packages are designed to be
compatible with a particular set of packages.  Users are assumed to
work within one single package universe, which sets a context or scope
that simplifies several aspects of packages and the distribution
system.


\section{Philosophy and strategy}

\subsection{Package universes exist}
Philosophically, package universes exist whether or not they are
treated explicitly by the architecture.  There are at least two
fundamental reasons.  First, packages are only \emph{convenient} for a
user when they have been \emph{configured} consistently with other
other packages the user wishes to use.  Packages that install files
into different or even conflicting locations in a file system cannot be
conveniently installed, but instead must fundamentally require
configuration after installation.  Even if someone could define a
convention for every conceivable option (which is not possibly anyway,
at least for software package systems), such a set of conventions
would constitute a distribution.  Other systems would doubtless use
different conventions, thus reopening the original problem.

The second reason is that packages are only \emph{reliable} for a user
when they are \emph{tested} with respect to other packages the user is
interested in.  Some software errors only appear in response to
combinations of packages.  One combination might work fine, while
another exhibits new errors.

For both reasons, packages are only convenient and reliable with
respect to other packages.  Therefore, the set of all possible
packages naturally partitions into smaller sets of mutually compatible
packages.


\subsection{Explicit management of package universes}
The package universes architecture goes further and explicitly models
package universes.  Users must designate which universe their packages
will be drawn from.  This assumption adds a restriction to how users
operate, though hopefully not an onerous one.  It is as if each
community of users operates within a limited bubble universe, only
able to see packages that are within their own universe.  If a user
wishes to use a package from a foreign universe, then
someone---possibly the user---first needs to modify the package to be
consistent with the local universe and add the modified version to the
local universe.

This assumption yields a number of simplifications in other parts of
the architecture.  For example, package names can be short.  They do
not need to include DNS domain names or otherwise attempt to be
globally unique.
Names can be short because they are all used within the
context of one universe, and each universe has a presumably
cooperative set of users (communities whose users attack each other
have larger problems than their infrastructure!).  Debian
\cite{monga04:debian,debian:web}
shows that short human-readable names, names that do not, for example,
include embedded DNS host names, are practical for communal libraries
at least as large as 10,000 packages.

As an aside, the single universe-wide scope is intended to support a
person's workspace.  It is not a suggestion or endorsement that the
packages themselves should refer to each other via the universe-wide
names.  Instead, it is intended to allow a human user a terse name for
each package in their local library, thereby supporting creative
development.


Another simplification is that versions can be totally ordered.
Branching versions are not required.  Users within the universe are
assumed to cooperate and to want the same update policy for their
packages.  Thus, all users of the universe can update their package in
tandem, instead of some users operating on different branches of
development from others.

The dependency system, in turn, can be as simple as ``depends on B.''
Versioned dependencies, such as ``depends on B newer than version
1.4,'' are not needed, because users tend to update to the newest
version of all packages anyway.  A more refined dependency system is
described below, but this approach appears sufficient for practical
usage.  The sufficiency of such a simple dependency system follows
from the fundamental design decision that each universe includes a
limited set of carefully chosen packages.  Because of this design,
software can simply select the newest version of any needed package
without needing to select among a large number of versions.



It is true that this simple versioning approach does not directly
support version changes that intentionally break compatibility.
However, because the universe has a limited user base, members of the
community can collude to avoid the problem.  Users can use multiple
package names to represent the incompatible components.  For example,
if ``libc version 5.100'' and ``libc version 6.101'' break
compatibility, then they could be considered separate packages by
naming them ``libc5 version 100'' and ``libc6 version 101.''


In general, any dependency problems that arise can be worked around by
modifying the contents of the package universe.  That is, instead of
requiring the dependency system to handle an arbitrary set of
packages, the package universes approach is to let humans engineer the
set of packages so that they have simply dependencies.  Instead of
solving harder problems, community members can work together to make
the problems easier.


The explicit universes assumption also provides leverage to enforce a
variety of update policies.  Gatekeepers for a package universe can
refuse to accept any updates that do not conform to the particular
universe's policies.  A number of examples are given below.

Finally, the design avoids centralized control.  There is no central
registry of universes.  Instead, universes are developed bottom-up by
individual communities that want to share code with each other.
Additionally, even large widely shared universes do not have complete
control over those who participate.  Local communities can build a
universe based on a larger universe which incorporates tweaks that the
local community desires.  Examples of such combinations are also
described below.



\subsection{Evolution and updating}
As community members create new content and edit old content, they
contribute new packages to the universe.  Thus, package universes
evolve over time.  Inter-package conventions and interfaces can drift
as the universe evolves.  For the universe's packages to remain
compatible with each other, the speed of drift must not be too fast.
The drift must be slow enough that users who contribute packages can
update their packages to accommodate the drifting conventions.

It is assumed that users update regularly to the newest version of
each package they use that is available within their universe.  Users
who wish to use older software are still accommodated.  Such users must
subscribe to a universe whose newest versions are essentially the same
as the old versions.  Note that even in this case, users rarely wish
to use precisely the old version.  Instead, they wish to use the old
version along with any simple bug fixes that have been found over
time.  Thus, even stable universes containing ``old'' software tend to
have a trickle of updates over time.

It is a conjecture behind the system that most communities only need a
small number of universes to accommodate the update policies the users
desire.  For any community where the conjecture holds, the
simplifications of the package universes architecture hold in force.
For any community where the conjecture fails, and where users want to
use a large number of combinations of old and new software, a more
sophisticated approach is required.  Further experience is required to
determine whether this approach of multiple package universes is
effective for most communities.  For that matter, further experience
is required to determine the most common ways, if any, that actual
communities need something more.


\section{Overall architecture}
The overall architecture is that a package universe implements a
\emph{communal library} of \emph{packages} on the network, while
individual users have a \emph{workspace} that includes a \emph{local
library}.  The local library consists of packages selectively
downloaded from the communal library.

%XXX change ``universe server'' to ``index server''
The software components are a \emph{universe server} maintaining the
communal library and allowing it to be manipulated remotely, and a
\emph{universe client} that interfaces with human users, local
libraries, and universe servers.

A package is the unit of interchange of content.  It might contain any
form of content, {\eg} software, media components, or written essays.
A package can be loaded and unloaded from the user's local library.
Each package has a name, used to refer to the package from the
workspace and (possibly) from the content of other packages, and it
has a version number, used to distinguish different revisions of the
same content.

There are several kinds of universes that are commonly discussed.  An
\emph{index-server} universe is one whose packages are those stored
explicitly on an \emph{index server}.  The contents of the index
server change over time as authorized developers add, update, and
remove packages.  An index server corresponds closely to a repository
in many Linux distributions.

Universes can be combined and wrapped into other universes in several
ways.  A \emph{name-based union} combines the packages from an ordered
list of universes.  Packages in later universes in the list override
all same-named packages from earlier in the list, including packages
with the same name but a different version.  A \emph{pure union} does
not override same-named, different-version packages, but such pure
unions appear less useful in practice than name-based unions.

A \emph{filter} universe includes a subset of the packages in another
universe.  The subset is defined by a regular expression which is
matched against the package names.  Such universes are useful for
selecting a few packages---possibly just one package---from an index
server.

The \emph{empty} universe has no packages.  It is a special case of the
union of zero universes.  The empty universe is a useful boundary
case in theory and in implementation of package universes.

A \emph{literal} universe includes an explicit list of packages in the
universe description.  Literal universes are practical in two
scenarios: test suites for package-universe implementations, and small
universes for personal use.

Finally, an \emph{indirect} universe holds a URL that references
another universe's description.  Indirect universes are useful for
communities whose package universes have a complex internal structure.
Instead of community members installing complex universe descriptions
on their machines, they can refer to a description located on a
central community web site.



\section{Package dependencies}
This section describes a dependency system that appears sufficient
while remaining simple.  It is only partially implemented.  It is
described here for a few reasons.  It demonstrates how simple it
appears a dependency system for a package universe can be, it
documents a canonical dependency system for package universes when no
other is specified, and it documents the planned dependency system
for Scala Bazaars in particular.

Package names are considered unique within a particular universe.
Whenever two packages have the same name but a different version
number, they are treated as holding different revisions of the same
content.  The package with the larger version number is preferred by
the tools by default.

Packages may have the following relations to other packages:
\begin{itemize}
\item A depends on B.  Package A absolutely requires that some
      version of package B be installed for it to function.
\item A suggests B.  Package A is usually used in conjunction with
      package B, but it is not a strict requirement.
\item A provides B.  When package A is installed, it provides the same
      functionality as package B.  Packages that depend on B are
      satisfied if package A is installed instead.  Package ``B'' can
      even be a virtual package, a package that is not concretely present
      in the repository but instead only appears in ``provides'' dependencies.
\end{itemize}
To date, only the direct depends-on relation has been implemented in
Scala Bazaars.  ``Provides'' and ``suggests'' are left for future
work.

The source package in a dependency is versioned, so that Foo version
1.1 can depend on different packages from Foo version 1.2.  However,
the target package in a dependency is not versioned.  If Foo 1.0
depends on Bar, then it does not get to choose which version of Bar it
depends on.  It is the responsibility of the community members to
ensure that the newest version of Bar---along with every version of
Bar if possible---is compatible with every package that depends on it.

A package's list of dependencies may be conjoined with both AND and OR
operations.  AND combinations allow a package to depend on multiple
other packages.  OR combinations, which have not yet been implemented
in Scala Bazaars, give each user a choice in which packages are
installed to satisfy another packages dependencies.


\section{Access control}

Update policies are enforced by providing an interface to primitive
universes based on \emph{capabilities} \cite{miller00:caps}.  Universe
servers on the network only allow operations when the requestor
presents a sufficient capability to use that request.  The servers
themselves do not use usernames and passwords nor any other from of
access-control lists.

Capabilities can be implemented as unguessably large random numbers.
Holding a capability is then equivalent to knowing the large number
associated with the capability.  A capability can be transmitted to a
user (or to a software object) by sending him or her the number.  If a
capability is revoked on a server, then all holders of that capability
lose the authority associated with that capability.


Primitive universe servers respond to capabilities designating the
following operations:
\begin{enumerate}
\item Download the list of package descriptions.
\item Edit package descriptions where the package
      name matches a fixed regular expression.
\item Create and revoke arbitrary capabilities.
\end{enumerate}
The first capability is often granted to all community members, but
might be denied to some users if there are multiple universe servers
for the community and some of them share privileged information.

The second capability allows access to packages based on their names.
There is no mechanism to grant access to specific versions of a
package, nor to separate editing operations on packages, because these
are all expected to be given out in groups.  Further, a regular
expression is allowed for matching the package name, thus allowing
granting access to multiple packages at once.  For example, a typical
regular expression is \texttt{scala-.*}, which grants access to all
packages whose names begin with \texttt{scala-}.

The third capability has no subdivisions.  It is not expected to be
frequently useful for communities to grant some but not all capability
manipulation commands.

The capabilities approach allows (hopefully) convenient implementation
of a wide variety of access-control policies for the community.
Simple communities do not need any notion of user, but instead can
have officers create, grant, and revoke capabilities manually for the
users who should have them.  Simple email is sufficient for
transmitting keys in simple communities.

More sophisticated communities can implement a capability-handout
server that uses existing community infrastructure such as LDAP
databases.  Since no user model or authentication approach is
presumed, it should be straightforward to implement such servers.

The most awkward case is for communities where the defined set of
capabilities cannot be combined into the kinds of access allowed for
users in the community.  For example, if a community wants to give
users access to packages based on versions in addition to names, the
above set of capabilities is not sufficient.  Even for such
communities, though, it is possible to implement a proxy server that
implements the more awkward operations under the correct policy.
However, with such an approach, the proxied operations cannot be
performed via standard tools.


\section{Sample configurations}
The package universes architecture allows convenient implementation of
a variety of community organizations and software-engineering
processes.  Here are a few examples.
\begin{itemize}
\item No restrictions.  Wikis have proven that effective and useful
      communities can be built even with no security restrictions at
      all.  This policy can be implemented by posting all of the
      universe's capabilities on a public web page.

\item Full access, but only to community members.  Many open-source
      projects have an organization of this form, including Debian's
      ``Debian Developers'' and FreeBSD's ``committers.''  This policy
      can be implemented by giving the capability-creation and
      -revocation capabilities to the membership gatekeepers.
      The last step of admitting a new member to the community is to
      give them their own capabilities to manipulate the community
      universes.

\item Single provider, multiple users.  An individual developer wants
      to provide a suite of packages to be used by a larger
      community---possibly the world, or possibly a limited base of
      subscribers.  To implement this policy, the developer can keep a
      universe's package-adding capabilities for personal use but
      publish the universe-retrieval capability to the desired
      user base.

\item Moderation queues.  Sometimes it is desirable to have a large
      community contribute suggested packages, but a smaller group to
      decide on what is actually included.  A moderation queue can be
      implemented as a separate universe where the package-addition
      capability is made available to whichever community is allowed
      to contribute suggestions (perhaps the entire world).
      Moderators would have both a reading capability for the
      moderation queue and a package-adding capability for the main
      universe, and could copy packages from the queue to the main
      universe whenever they are deemed appropriate.


\item Private local development.  Individual groups can form their own
      universe for private development without needing to coordinate
      with any central organization.  They simply create the universe
      and share keys with members of the group according to their
      own local security policy.

\item Public libraries plus local development.  The above organization
      can be refined by allowing developers to use publicly
      available packages even as they develop packages for private
      use.  This policy can be implemented by taking the union of the
      local universe with one or more public universes.


\item Localized versions of packages.  Local groups may want to use
      something similar to a widely-scoped universe, but override
      specific packages with localized versions.  A particular example
      is language-specific versions of packages.  Users want to use
      the localized version of a package whenever one is available,
      but the global version otherwise.  This policy is implemented by
      creating a universe holding the localized packages, and then
      having users operate in the name-based union of this universe
      with the public one.

\item Stable versus unstable streams.  Projects frequently distinguish
      between stable and unstable streams of development.  The stable
      streams include packages that are heavily tested and deemed to
      be reliable, while the unstable streams include packages that
      are newer and more powerful but are not as reliable.  A
      project can implement this policy by having separate universes
      for each development stream.  

\item Freezing new stable distributions.  A common process for
      generating stable distributions of code is to take an unstable
      stream, start testing it, and disallow any patches except for
      bug fixes.
%CITE can i find anything on this?
      After some
      point, the distribution is \emph{frozen} and considered a stable
      release.  Such processes can be implemented by manipulating the
      outstanding capabilities as time passes.  The testing phase can
      be implemented by revoking all outstanding add-package
      capabilities and switching to a moderation queue process.  The
      actual freeze can be implemented by destroying the moderation
      queue and revoking the remaining add-package capabilities.  The
      frozen universe can then be duplicated, with one fork hosting
      a new development stream while the other becomes is designated
      a stable release.
\end{itemize}



\section{Related work}
There has been much work on packages and package distribution.  This
section is currently overly brief.  Feedback is welcome, so that the
section can be enriched with comparisons to other systems.


\subsection{Packages themselves}
The present work focuses on package distribution, not much on packages
themselves.  The underlying conjecture is that it is more effective
for a community to cooperatively tweak packages within a designated
set of packages, than it is for individual packages to be so robust
and flexible that they can be installed on any machine and with any
other packages.


Nevertheless, better package formats reduce strain on the package
repositories.  For example, a more flexible package format could
potentially allow two communities to operate within the same package
universe, even when with a simpler format they would each require
their own universe.  Further, better package formats can reduce
accidents that occur as packages are developed semi-independently,
much the same way that memory-safe programming languages prevent
damage from stray pointers.

Two examples of work on package formats are OSGi packages
\cite{osgi03:book, osgi:web} and the scsh package format
\cite{schinz05:scsh}.


\subsection{Package distribution}

%XXX read about Ruby Gems

%XXX compare to GENESIS


Debian and its Advanced Package Tool (APT) \cite{apt:howto}
are the primary inspiration behind the package universes architecture.
Package universes develop the design of these tools in a few ways:
\begin{itemize}
\item Package universes supports different ways to combine universes.
      APT only supports simple unions.  Particularly interesting are
      name-based union and the name-based filtering, which support
      similar functionality to APT's \emph{pinning} mechanism.

\item Package universes supports uploading packages and package
      descriptions in the core architecture, while APT leaves
      uploading for separate tools.

\item In support of uploading, package universes has a refined,
      policy-neutral approach to access control.
\end{itemize}


Conary is an ambitious system that aims to provide full version
control across distributed ad-hoc repositories \cite{conary:web}.
It shares with
package universes the design goal of allowing new repositories to be
formed from existing ones without needing any kind of permission from
the original ones.

It is an open design question whether a package distribution system
should include version control.  Conary explores the design path of
including version control, thus achieving a system with more features.
Package universes, on the other hand, intentionally rejects
sophisticated versioning.  Instead of branching
within a repository, the approach in the package universes
architecture is to fork a universe into two separate universe.
Instead of having users remain at earlier revisions within a
repository, the package universes approach is to fork the universe
into a fast-moving universe and a slow-moving one.  Instead of modelling
branching within a single repository, branches can be implemented
by forking universes into multiple universes.

Conary pays for internal version control with a large increase in
complexity.  In the package universes architecture, package universes
contain packages which contain files.  To contrast, Conary has four
ways to group files: groups, packages, components, and filesets.  The
Conary documentation gives considerable attention to the specific case
of a local administrator having a ``local shadow'', which is
accomplished in package universes by the simple and generic mechanism of
name-based union.


At the very least, package universes provides a simple conceptual core
for package distribution that does not use version control.  However,
if version control is more complex than worthwhile, package universes
may be a better architecture outright.  Opinions differ over the
desirability of version control within a package-distribution
architecture, and it appears that resolving the issue requires more
time and experience.



\bibliography{architecture.bib}

\end{document}




% LocalWords:  SqueakMap SuperSwikis Etoys formatter focussing DNS Versioned
% LocalWords:  versioning versioned requestor usernames unguessably hoc OSGi
% LocalWords:  FreeBSD's committers scsh APT's Conary modelling filesets libc
% LocalWords:  scala LDAP proxied proven
